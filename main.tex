\documentclass[11pt]{article}
\usepackage{amsmath,amssymb,amsthm,epsf, graphics,enumerate,fancyhdr,marvosym,cancel}
\usepackage[letterpaper, margin=0.9in]{geometry}

% ordered lists have letters instead of numbers
\renewcommand\theenumi{\alph{enumi}}
\renewcommand\labelenumi{(\theenumi)}

% subsections have letters instead of numbers
\renewcommand{\thesubsection}{\thesection (\alph{subsection})}

% use hyperref for links but don't draw the ugly boxes around links
\usepackage[hidelinks]{hyperref}

% command to create underlined link to outside websites
\newcommand{\hrefunderline}[2]{\underline{\href{#1}{#2}}}

% symbols for common sets
\newcommand{\ZZ}{\mathbb{Z}}
\newcommand{\NN}{\mathbb{N}}
\newcommand{\FF}{\mathbb{F}}
\newcommand{\RR}{\mathbb{R}}
\newcommand{\QQ}{\mathbb{Q}}
\newcommand{\CC}{\mathbb{C}}

% use boldface instead of arrows for vectors
%\renewcommand{\vec}{\mathbf}

% order
\newcommand{\Od}[1]{\mathcal{O}{\left(#1\right)}}

% bras and kets
\newcommand{\bra}[1]{\left\langle#1\right|}
\newcommand{\ket}[1]{\left|#1\right\rangle}
\newcommand{\braket}[2]{\left\langle#1|#2\right\rangle}

% quantum operators
\newcommand{\op}[1]{\hat{#1}}

% for congruences
\newcommand{\mmod}[1]{\;(\operatorname{mod} {#1})}

% new theorem style with boldface title and italic content
\newtheoremstyle{step}%            % Name
  {}%                                     % Space above
  {}%                                     % Space below
  {\itshape}%                           % Body font
  {}%                                     % Indent amount
  {\itshape}%                          % Theorem head font
  {:}%                                    % Punctuation after theorem head
  { }%                                    % Space after theorem head, ' ', or \newline
  {}%                                     % Theorem head spec (can be left empty, meaning `normal')

% new theorem style for a gap in the notes
\newtheoremstyle{gap}%            % Name
  {}%                                     % Space above
  {}%                                     % Space below
  {\itshape}%                           % Body font
  {}%                                     % Indent amount
  {\itshape}%                          % Theorem head font
  {!!!}%                                    % Punctuation after theorem head
  {    }%                                    % Space after theorem head, ' ', or \newline
  {}%                                     % Theorem head spec (can be left empty, meaning `normal')

\theoremstyle{theorem}
\newtheorem{claim}{Claim}[section]
\newtheorem*{claim*}{Claim}
\newtheorem{lemma}[claim]{Lemma}
\newtheorem*{lemma*}{Lemma}
\newtheorem{fact}{Fact of Nature}

\theoremstyle{remark}
\newtheorem*{remark}{Remark}
\newtheorem*{notation}{Notation}

\theoremstyle{step}
\newtheorem{step}{Step}[subsection]
\renewcommand{\thestep}{\arabic{step}}

\theoremstyle{gap}
\newtheorem*{gap}{Gap}

\begin{document}


\title{Scattering Notes}
\author{Duncan MacIntyre}
\date{\today}
\maketitle
\tableofcontents
\bigskip
\newpage

\section{Time-dependent perturbation theory in the interaction picture}

{\bf Perturbation theory setup.}
Suppose we have a Hamiltonian of the form \[\op{H} = \op{H}_0 + \op{V}(t)\]
where \(\op{H}_0\) is a well-understood Hamiltonian that does not depend on time and \(\op{V}(t)\) is ``small''. For example, \(\op{H}_0\) might be the free particle Hamiltonian \(\op{H}_0 = \frac{m}{2} \nabla^2\). Our equation of motion is the Schr\"odinger equation
\[i \hbar \frac{\partial}{\partial t} \ket{\Psi, t} = \left(\op{H}_0 + \op{V}(t)\right) \ket{\Psi, t}\]
where \(\ket{\Psi, t}\) is the usual Schr\"odinger-picture state at time \(t\).

Let \(\op{U}(t_0, t) = e^{-i\op{H}_0(t-t_0)/\hbar}\). Then \(\op{U}(t_0, t)\) is the operator that evolves a state from time \(t_0\) to time \(t\) according to \(\op{H}_0\). We define the interaction-picture state to be
\[\ket{\Psi_I, t} = \op{U}(t_0, t)^\dagger \ket{\Psi, t}\]
so \(\ket{\Psi, t} = \op{U}(t_0, t) \ket{\Psi_I, t}\).
Plugging this in to the Schr\"odinger equation,
\begin{align*}
i \hbar \frac{\partial}{\partial t} \op{U}(t_0, t)\ket{\Psi_I, t} &= \left(\op{H}_0 + \op{V}(t)\right) \op{U}(t_0, t)\ket{\Psi, t} \\
i \hbar \left[\frac{\partial}{\partial t} e^{-i\op{H}_0(t-t_0)/\hbar}\right] \ket{\Psi_I, t} + i \hbar e^{-i\op{H}_0(t-t_0)/\hbar} \frac{\partial}{\partial t} \ket{\Psi_I, t} &= \op{H}_0 e^{-i\op{H}_0(t-t_0)/\hbar}\ket{\Psi_I, t} + \op{V}(t) e^{-i\op{H}_0(t-t_0)/\hbar}\ket{\Psi_I, t} \\
\cancel{-i^2 \op{H}_0 e^{i\op{H}_0(t-t_0)/\hbar} \ket{\Psi_I, t}} + i \hbar e^{-i\op{H}_0(t-t_0)/\hbar} \frac{\partial}{\partial t} \ket{\Psi_I, t} &= \cancel{\op{H}_0 e^{-i\op{H}_0(t-t_0)/\hbar}\ket{\Psi_I, t}} + \op{V}(t) e^{-i\op{H}_0(t-t_0)/\hbar}\ket{\Psi_I, t} \\
 i \hbar \frac{\partial}{\partial t} \ket{\Psi_I, t} &= e^{i\op{H}_0(t-t_0)/\hbar}\op{V}(t) e^{-i\op{H}_0(t-t_0)/\hbar}\ket{\Psi_I, t} \\
 i \hbar \frac{\partial}{\partial t} \ket{\Psi_I, t} &= \op{H}_I(t)\ket{\Psi_I, t}
\end{align*}
where we define the interaction Hamiltonian to be
\[\op{H}_I(t) = \op{U}(t_0, t)^\dagger \op{V}(t)\op{U}(t_0, t) = e^{i\op{H}_0(t-t_0)/\hbar}\op{V}(t) e^{-i\op{H}_0(t-t_0)/\hbar}.\]

This sets up the interaction picture. We have rephrased our problem so that we can continue with quantum mechanics normally without having to worry about the time evolution due to \(\op{H}_0\).

We now integrate both sides of our expression.
\begin{align}
\int_{t_0}^t \frac{\partial}{\partial t^\prime} \ket{\Psi_I, t^\prime} &= -\frac{i}{\hbar} \int_{t_0}^t \op{H}_I(t^\prime)\ket{\Psi_I, t^\prime} \,dt^\prime \nonumber\\
\ket{\Psi_I, t} - \ket{\Psi_I, t_0} &= -\frac{i}{\hbar} \int_{t_0}^t \op{H}_I(t^\prime)\ket{\Psi_I, t^\prime} \,dt^\prime \nonumber\\
\ket{\Psi_I, t} &= \ket{\Psi_I, t_0} - \frac{i}{\hbar} \int_{t_0}^t \op{H}_I(t^\prime)\ket{\Psi_I, t^\prime} \,dt^\prime \label{eq.intse}
\end{align}
This is called the integral form of the Schr\"odinger equation.

We can now iteratively calculate perturbative approximations where we assume \(\op{H}_I(t)\) is small. The zero order approximation is simply
\[\ket{\Psi_I, t} = \ket{\Psi_I, t_0} + \Od{\op{H}_I(t)}.\]
Plugging this in for the state inside the integral in (\ref{eq.intse}), we get the first order approximation
\begin{equation}\label{eq.firsttdpe}
\ket{\Psi_I, t} = \ket{\Psi_I, t_0} - \frac{i}{\hbar} \int_{t_0}^t \op{H}_I(t^\prime)\ket{\Psi_I, t_0} \,dt^\prime + \Od{\op{H}_I(t)^2}.
\end{equation}
We can plug in again to get the second order approximation
\[
\ket{\Psi_I, t} = \ket{\Psi_I, t_0} - \frac{i}{\hbar} \int_{t_0}^t \op{H}_I(t^\prime)\left[\ket{\Psi_I, t_0} - \frac{i}{\hbar} \int_{t_0}^t \op{H}_I(t^{\prime\prime})\ket{\Psi_I, t_0} \,dt^{\prime\prime}\right] \,dt^\prime + \Od{\op{H}_I(t)^3}.
\]
In general, we can keep going to achieve higher order approximations. For us, however, the first order approximation (\ref{eq.firsttdpe}) is enough.

{\bf Application to scattering.} We will continue for scattering between two particles. We will assume that the particles are distinguishable so that we don't need to restrict ourselves to symmetric or antisymmetric states. We will also assume that the particles interact in a way that only depends on the distance between them. That is, in the position basis, we can write \[\bra{x_1, x_2}\op{V}(t)\ket{\psi} = \int_{x_1,x_2} \frac{d^3 x_1 \,d^3x_2}{(2\pi)^6} V(x_1 - x_2) \braket{x_1, x_2}{\psi}.\] (Note that this is not time dependent.) Take \(\op{H}_0\) to be the free particle Hamiltonian.

It will be easiest to work in the momentum basis. We will do our calculations for momentum eigenstates---that is, plane waves. Write momentum eigenstates as \(\ket{k_1, k_2}\). In the position basis these have wavefunctions
\[\braket{x_1, x_2}{k_1, k_2} = e^{ik_1\cdot x_1 + ik_2\cdot x_2}.\]
Here \(x_1\) and \(x_2\) are the (3-vector) positions of the first and second particles and \(k_1\) and \(k_2\) are the (3-vector) momenta of the first and second particle. Also, we note that \(\ket{k_1, k_2}\) has energy \(\frac{1}{2m} \left({k_1}^2 + {k_2}^2\right)\) and so
\begin{equation}
\label{eq.timeevonk}
\op{U}(t_0, t) \ket{k_1, k_2} = e^{\frac{i\left(t-t_0\right)\left({k_1}^2 + {k_2}^2\right)}{2m\hbar}} \ket{k_1, k_2}.
\end{equation}

Let \(\widetilde{V}(p)\) be the Fourier transform of \(V(x_1 - x_2)\) so that
\begin{equation}
\label{eq.ftV}
V(x_1 - x_2) = \int_{-\infty}^{\infty} \frac{d^3 p}{(2 \pi)^3} \widetilde{V}(p) e^{ip\cdot(x_1 - x_2)}.
\end{equation}

We now consider (\ref{eq.firsttdpe}) and take the limits \(t_0 \to -\infty\) and \(t \to \infty\) because, in scattering, we measure states long before and long after the scattering occurs. Then, dropping the higher-order terms,
\[
\ket{\Psi_I, \infty} = \ket{\Psi_I, -\infty} - \frac{i}{\hbar} \int_{-\infty}^\infty \op{H}_I(t)\ket{\Psi_I, -\infty} \,dt.
\]
Take the initial state to be \(\ket{\Psi_I, -\infty} = \ket{k_1, k_2}\). Multiplying by the bra \(\bra{k_1^\prime, k_2^\prime}\),
\begin{align}
\braket{k_1^\prime, k_2^\prime}{\Psi_I, \infty} &= \braket{k_1^\prime, k_2^\prime}{k_1, k_2} - \frac{i}{\hbar} \int_{-\infty}^\infty \bra{k_1^\prime, k_2^\prime} \op{H}_I(t)\ket{k_1, k_2} \,dt \nonumber\\
&= \delta^3(k_1^\prime - k_1)\delta^3(k_2^\prime - k_2) - \frac{i}{\hbar} \int_{-\infty}^\infty \bra{k_1^\prime, k_2^\prime} \op{H}_I(t)\ket{k_1, k_2} \,dt. \label{eq.intermed}
\end{align}
This motivates us to compute the matrix element
\[
\bra{k_1^\prime, k_2^\prime} \op{H}_I(t)\ket{k_1, k_2} = \bra{k_1^\prime, k_2^\prime} \op{U}(t_0, t)^\dagger \op{V}(t) \op{U}(t_0, t) \ket{k_1, k_2}.
\]
Plugging in (\ref{eq.timeevonk}),
\begin{align*}
\bra{k_1^\prime, k_2^\prime} \op{H}_I(t)\ket{k_1, k_2} &= \bra{k_1^\prime, k_2^\prime} e^{\frac{i\left(t-t_0\right)\left({k_1^\prime}^2 + {k_2^\prime}^2\right)}{2m\hbar}} \op{V}(t) e^{\frac{-i\left(t-t_0\right)\left({k_1}^2 + {k_2}^2\right)}{2m\hbar}} \ket{k_1, k_2} \\
&= e^{\frac{i\left(t-t_0\right)\left({k_1^\prime}^2 + {k_2^\prime}^2 - {k_1}^2 - {k_2}^2\right)}{2m\hbar}} \bra{k_1^\prime, k_2^\prime} \op{V}(t) \ket{k_1, k_2} \\
&= e^{\frac{i\left(t-t_0\right)\left({k_1^\prime}^2 + {k_2^\prime}^2 - {k_1}^2 - {k_2}^2\right)}{2m\hbar}} \bra{k_1^\prime, k_2^\prime} \left\{\int_{x_1, x_2} \frac{d^3 x_1 d^3 x_2}{(2\pi)^6}\ket{x_1, x_2}\bra{x_1, x_2}\right\}\op{V}(t) \ket{k_1, k_2} \\
&= e^{\frac{i\left(t-t_0\right)\left({k_1^\prime}^2 + {k_2^\prime}^2 - {k_1}^2 - {k_2}^2\right)}{2m\hbar}}  \int_{x_1, x_2} \frac{d^3 x_1 d^3 x_2}{(2\pi)^6}\braket{k_1^\prime, k_2^\prime}{x_1, x_2}\bra{x_1, x_2}\op{V}(t) \ket{k_1, k_2} \\
&= e^{\frac{i\left(t-t_0\right)\left({k_1^\prime}^2 + {k_2^\prime}^2 - {k_1}^2 - {k_2}^2\right)}{2m\hbar}}  \int_{x_1, x_2} \frac{d^3 x_1 d^3 x_2}{(2\pi)^6}e^{-ik_1^\prime\cdot x_1 - ik_2^\prime\cdot x_2}V(x_1 - x_2) e^{ik_1\cdot x_1 + ik_2\cdot x_2} 
\end{align*}
and plugging in (\ref{eq.ftV}),
\begin{align*}
\bra{k_1^\prime, k_2^\prime} \op{H}_I(t)\ket{k_1, k_2} 
&= e^{\frac{i\left(t-t_0\right)\left({k_1^\prime}^2 + {k_2^\prime}^2 - {k_1}^2 - {k_2}^2\right)}{2m\hbar}} \int_{-\infty}^{\infty} \frac{d^3p}{(2\pi)^3}\int_{x_1, x_2} \frac{d^3 x_1 d^3 x_2}{(2\pi)^6}e^{-ik_1^\prime\cdot x_1 - ik_2^\prime\cdot x_2}\widetilde{V}(p)e^{ip\cdot(x_1-x_2)} e^{ik_1\cdot x_1 + ik_2\cdot x_2} \\
&= e^{\frac{i\left(t-t_0\right)\left({k_1^\prime}^2 + {k_2^\prime}^2 - {k_1}^2 - {k_2}^2\right)}{2m\hbar}} \int_{-\infty}^{\infty} \frac{d^3p}{(2\pi)^3}\widetilde{V}(p) \int_{x_1, x_2} \frac{d^3 x_1 d^3 x_2}{(2\pi)^6}e^{i(k_1-k_1^\prime+p)\cdot x_1}e^{i(k_2-k_2^\prime-p)\cdot x_2}\\
&= e^{\frac{i\left(t-t_0\right)\left({k_1^\prime}^2 + {k_2^\prime}^2 - {k_1}^2 - {k_2}^2\right)}{2m\hbar}} \int_{-\infty}^{\infty} \frac{d^3p}{(2\pi)^3}\widetilde{V}(p) \,\delta^3(k_1-k_1^\prime+p)\,\delta^3(k_2-k_2^\prime-p)\\
&= e^{\frac{i\left(t-t_0\right)\left({k_1^\prime}^2 + {k_2^\prime}^2 - {k_1}^2 - {k_2}^2\right)}{2m\hbar}} \widetilde{V}(k_1^\prime-k_1) \;\delta^3(k_1^\prime-k_1-k_2^\prime+k_2).
\end{align*}
Putting this in to (\ref{eq.intermed}) we get
\begin{align}
\braket{k_1^\prime, k_2^\prime}{\Psi_I, \infty} =&\;\; \delta^3(k_1^\prime - k_1)\delta^3(k_2^\prime - k_2) - \frac{i}{\hbar} \int_{-\infty}^\infty dt\;e^{\frac{i\left(t-t_0\right)\left({k_1^\prime}^2 + {k_2^\prime}^2 - {k_1}^2 - {k_2}^2\right)}{2m\hbar}} \widetilde{V}(k_1^\prime-k_1) \;\delta^3(k_1^\prime-k_1-k_2^\prime+k_2) \nonumber\\
=& \;\;\delta^3(k_1^\prime - k_1)\;\delta^3(k_2^\prime - k_2) \nonumber\\&\;\;\;\;- {i}\, \widetilde{V}(k_1^\prime-k_1) \;\delta^3(k_1^\prime-k_1-k_2^\prime+k_2) \;\delta\left(\frac{1}{{2m}}\left({k_1^\prime}^2 + {k_2^\prime}^2 - {k_1}^2 - {k_2}^2\right)\right).
\end{align}
Interestingly, we see that the delta functions cause energy and momentum to be conserved.

{\bf Mixed states and the density operator.} We now consider the same problem but where we allow for mixed states. We will need to describe the system by the density operator \(\op{\rho}(t)\). First, let us derive the time evolution of \(\op{\rho}(t)\) in first order perturbation theory.

Suppose at time \(t_0\) we have a statistical ensemble of interaction-picture states \(\ket{{\psi_I}_n, t_0}\) each with probability \(P_n\). Then
\[\op{\rho}(t_0) = \sum_n P_n \ket{{\psi_I}_n, t_0}\bra{{\psi_I}_n, t_0}.\]
At time \(t\) states will have evolved according to (\ref{eq.firsttdpe}), so
\[\op{\rho}(t) = \sum_n P_n \ket{{\psi_I}_n, t}\bra{{\psi_I}_n, t}\]
where
\begin{align*}
\ket{{\Psi_I}_n, t} &= \ket{{\Psi_I}_n, t_0} - \frac{i}{\hbar} \int_{t_0}^t \op{H}_I(t^\prime)\ket{{\Psi_I}_n, t_0} \,dt^\prime + \Od{\op{H}_I(t)^2} \\
\bra{{\Psi_I}_n, t} &= \bra{{\Psi_I}_n, t_0} + \frac{i}{\hbar} \int_{t_0}^t \bra{{\Psi_I}_n, t_0} \op{H}_I(t^\prime) \,dt^\prime + \Od{\op{H}_I(t)^2} \\
\ket{{\Psi_I}_n, t}\bra{{\Psi_I}_n, t} &= \ket{{\Psi_I}_n, t_0}\bra{{\Psi_I}_n, t_0} - \frac{i}{\hbar} \int_{t_0}^t \left(\op{H}_I(t^\prime) \ket{{\Psi_I}_n, t_0}\bra{{\Psi_I}_n, t_0}  - \ket{{\Psi_I}_n, t_0}\bra{{\Psi_I}_n, t_0} \op{H}_I(t^\prime)\right) \,dt^\prime + \Od{\op{H}_I(t)^2} \\
\ket{{\Psi_I}_n, t}\bra{{\Psi_I}_n, t} &= \ket{{\Psi_I}_n, t_0}\bra{{\Psi_I}_n, t_0} - \frac{i}{\hbar} \int_{t_0}^t \left[\op{H}_I(t^\prime),\; \ket{{\Psi_I}_n, t_0}\bra{{\Psi_I}_n, t_0}\right] \,dt^\prime + \Od{\op{H}_I(t)^2}
\end{align*}
so
\begin{equation}\label{eq.rhotev}
\op{\rho}(t) = \op{\rho}\left(t_0\right) -  \frac{i}{\hbar} \int_{t_0}^t \left[\op{H}_I(t^\prime),\; \op{\rho}\left(t_0\right)\right] \,dt^\prime + \Od{\op{H}_I(t)^2}.
\end{equation}

As before, we drop the higher order terms and take \(t \to \infty\) and \(t_0 \to -\infty\). Then the momentum basis amplitude is
\[\bra{{k_1}^\prime, {k_2}^\prime} \op{\rho}(t) \ket{{k_1}, {k_2}}\]



\end{document}