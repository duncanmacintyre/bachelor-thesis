\documentclass[11pt]{article}
\usepackage{amsmath,amssymb,amsthm,epsf, graphics,enumerate,fancyhdr,marvosym,cancel}
\usepackage[letterpaper, margin=0.9in]{geometry}

% ordered lists have letters instead of numbers
\renewcommand\theenumi{\alph{enumi}}
\renewcommand\labelenumi{(\theenumi)}

% subsections have letters instead of numbers
\renewcommand{\thesubsection}{\thesection (\alph{subsection})}

% use hyperref for links but don't draw the ugly boxes around links
\usepackage[hidelinks]{hyperref}

% command to create underlined link to outside websites
\newcommand{\hrefunderline}[2]{\underline{\href{#1}{#2}}}

% symbols for common sets
\newcommand{\ZZ}{\mathbb{Z}}
\newcommand{\NN}{\mathbb{N}}
\newcommand{\FF}{\mathbb{F}}
\newcommand{\RR}{\mathbb{R}}
\newcommand{\QQ}{\mathbb{Q}}
\newcommand{\CC}{\mathbb{C}}

% use boldface instead of arrows for vectors
%\renewcommand{\vec}{\mathbf}

% order
\newcommand{\Od}[1]{\mathcal{O}{\left(#1\right)}}

% bras and kets
\newcommand{\bra}[1]{\left\langle#1\right|}
\newcommand{\ket}[1]{\left|#1\right\rangle}
\newcommand{\braket}[2]{\left\langle#1|#2\right\rangle}

% quantum operators
\newcommand{\op}[1]{\hat{#1}}

% for congruences
\newcommand{\mmod}[1]{\;(\operatorname{mod} {#1})}

% new theorem style with boldface title and italic content
\newtheoremstyle{step}%            % Name
  {}%                                     % Space above
  {}%                                     % Space below
  {\itshape}%                           % Body font
  {}%                                     % Indent amount
  {\itshape}%                          % Theorem head font
  {:}%                                    % Punctuation after theorem head
  { }%                                    % Space after theorem head, ' ', or \newline
  {}%                                     % Theorem head spec (can be left empty, meaning `normal')

% new theorem style for a gap in the notes
\newtheoremstyle{gap}%            % Name
  {}%                                     % Space above
  {}%                                     % Space below
  {\itshape}%                           % Body font
  {}%                                     % Indent amount
  {\itshape}%                          % Theorem head font
  {!!!}%                                    % Punctuation after theorem head
  {    }%                                    % Space after theorem head, ' ', or \newline
  {}%                                     % Theorem head spec (can be left empty, meaning `normal')

\theoremstyle{theorem}
\newtheorem{claim}{Claim}[section]
\newtheorem*{claim*}{Claim}
\newtheorem{lemma}[claim]{Lemma}
\newtheorem*{lemma*}{Lemma}
\newtheorem{fact}{Fact of Nature}

\theoremstyle{remark}
\newtheorem*{remark}{Remark}
\newtheorem*{notation}{Notation}

\theoremstyle{step}
\newtheorem{step}{Step}[subsection]
\renewcommand{\thestep}{\arabic{step}}

\theoremstyle{gap}
\newtheorem*{gap}{Gap}

\begin{document}


\title{Notes on Entropy Change Over Time}
\author{Duncan MacIntyre}
\date{\today}
\maketitle
\tableofcontents
\bigskip
\newpage

\section{Time-dependent perturbation theory}

Suppose we have a Hamiltonian of the form \[\op{H} = \op{H}_0 + \op{V}(t)\]
where \(\op{H}_0\) is a well-understood Hamiltonian that does not depend on time and \(\op{V}(t)\) is ``small''. For example, \(\op{H}_0\) might be the free particle Hamiltonian \(\op{H}_0 = \frac{m}{2} \nabla^2\). Our equation of motion is the Schr\"odinger equation
\[i \hbar \frac{\partial}{\partial t} \ket{\Psi, t} = \left(\op{H}_0 + \op{V}(t)\right) \ket{\Psi, t}\]
where \(\ket{\Psi, t}\) is the usual Schr\"odinger-picture state at time \(t\).

Let \(\op{U}(t_0, t) = e^{-i\op{H}_0(t-t_0)/\hbar}\). Then \(\op{U}(t_0, t)\) is the operator that evolves a state from time \(t_0\) to time \(t\) according to \(\op{H}_0\). We define the interaction-picture state to be
\[\ket{\Psi_I, t} = \op{U}(t_0, t)^\dagger \ket{\Psi, t}\]
so \(\ket{\Psi, t} = \op{U}(t_0, t) \ket{\Psi_I, t}\).
Plugging this in to the Schr\"odinger equation,
\begin{align*}
i \hbar \frac{\partial}{\partial t} \op{U}(t_0, t)\ket{\Psi_I, t} &= \left(\op{H}_0 + \op{V}(t)\right) \op{U}(t_0, t)\ket{\Psi, t} \\
i \hbar \left[\frac{\partial}{\partial t} e^{-i\op{H}_0(t-t_0)/\hbar}\right] \ket{\Psi_I, t} + i \hbar e^{-i\op{H}_0(t-t_0)/\hbar} \frac{\partial}{\partial t} \ket{\Psi_I, t} &= \op{H}_0 e^{-i\op{H}_0(t-t_0)/\hbar}\ket{\Psi_I, t} + \op{V}(t) e^{-i\op{H}_0(t-t_0)/\hbar}\ket{\Psi_I, t} \\
\cancel{-i^2 \op{H}_0 e^{i\op{H}_0(t-t_0)/\hbar} \ket{\Psi_I, t}} + i \hbar e^{-i\op{H}_0(t-t_0)/\hbar} \frac{\partial}{\partial t} \ket{\Psi_I, t} &= \cancel{\op{H}_0 e^{-i\op{H}_0(t-t_0)/\hbar}\ket{\Psi_I, t}} + \op{V}(t) e^{-i\op{H}_0(t-t_0)/\hbar}\ket{\Psi_I, t} \\
 i \hbar \frac{\partial}{\partial t} \ket{\Psi_I, t} &= e^{i\op{H}_0(t-t_0)/\hbar}\op{V}(t) e^{-i\op{H}_0(t-t_0)/\hbar}\ket{\Psi_I, t} \\
 i \hbar \frac{\partial}{\partial t} \ket{\Psi_I, t} &= \op{H}_I(t)\ket{\Psi_I, t}
\end{align*}
where we define the interaction Hamiltonian to be
\[\op{H}_I(t) = \op{U}(t_0, t)^\dagger \op{V}(t)\op{U}(t_0, t) = e^{i\op{H}_0(t-t_0)/\hbar}\op{V}(t) e^{-i\op{H}_0(t-t_0)/\hbar}.\]

This sets up the interaction picture. We have rephrased our problem so that we can continue with quantum mechanics normally without having to worry about the time evolution due to \(\op{H}_0\).

We now integrate both sides of our expression.
\begin{align}
\int_{t_0}^t \frac{\partial}{\partial t^\prime} \ket{\Psi_I, t^\prime} &= -\frac{i}{\hbar} \int_{t_0}^t \op{H}_I(t^\prime)\ket{\Psi_I, t^\prime} \,dt^\prime \nonumber\\
\ket{\Psi_I, t} - \ket{\Psi_I, t_0} &= -\frac{i}{\hbar} \int_{t_0}^t \op{H}_I(t^\prime)\ket{\Psi_I, t^\prime} \,dt^\prime \nonumber\\
\ket{\Psi_I, t} &= \ket{\Psi_I, t_0} - \frac{i}{\hbar} \int_{t_0}^t \op{H}_I(t^\prime)\ket{\Psi_I, t^\prime} \,dt^\prime \label{eq.intse}
\end{align}
This is called the integral form of the Schr\"odinger equation.

We can now iteratively calculate perturbative approximations where we assume \(\op{H}_I(t)\) is small. The zero order approximation is simply
\[\ket{\Psi_I, t} = \ket{\Psi_I, t_0} + \Od{\op{H}_I(t)}.\]
Plugging this in for the state inside the integral in (\ref{eq.intse}), we get the first order approximation
\begin{equation}\label{eq.firsttdpe}
\ket{\Psi_I, t} = \ket{\Psi_I, t_0} - \frac{i}{\hbar} \int_{t_0}^t \op{H}_I(t^\prime)\ket{\Psi_I, t_0} \,dt^\prime + \Od{\op{H}_I(t)^2}.
\end{equation}
Plugging this in to (\ref{eq.intse}) again we get the second order approximation
\begin{align}\nonumber
\ket{\Psi_I, t} &= \ket{\Psi_I, t_0} - \frac{i}{\hbar} \int_{t_0}^t \op{H}_I(t^\prime)\left[\ket{\Psi_I, t_0} - \frac{i}{\hbar} \int_{t_0}^{t^\prime} \op{H}_I(t^{\prime\prime})\ket{\Psi_I, t_0} \,dt^{\prime\prime}\right] \,dt^\prime + \Od{\op{H}_I(t)^3} \\
&= \ket{\Psi_I, t_0} - \frac{i}{\hbar}\int_{t_0}^t \op{H}_I (t^\prime) \ket{\Psi_I, t_0}\,dt^\prime - \frac{1}{\hbar^2} \int_{t_0}^t \int_{t_0}^{t^\prime} \op{H}_I(t^\prime) \op{H}_I(t^{\prime\prime})\ket{\Psi_I, t_0} \,dt^{\prime\prime}\,dt^\prime + \Od{\op{H}_I(t)^3}.
\label{eq.secondtdpe}\end{align}
In general, we can keep going to achieve higher order approximations. For us, however, the second order approximation (\ref{eq.secondtdpe}) is enough.


\section{Time evolution of the density operator}

We now consider mixed states. We will need to describe the system by the density operator \(\op{\rho}(t)\). Let's derive the time evolution of \(\op{\rho}(t)\) in second order perturbation theory based on (\ref{eq.secondtdpe}).

Suppose at time \(t_0\) we have a statistical ensemble of interaction-picture states \(\ket{{\psi_I}_n, t_0}\) each with probability \(P_n\). Then
\[\op{\rho}(t_0) = \sum_n P_n \ket{{\Psi_I}_n, t_0}\bra{{\Psi_I}_n, t_0}.\]
At time \(t\) states will have evolved according to (\ref{eq.secondtdpe}), so
\[\op{\rho}(t) = \sum_n P_n \ket{{\Psi_I}_n, t}\bra{{\Psi_I}_n, t}.\]
From (\ref{eq.secondtdpe}) we have
\begin{align*}
\ket{{\Psi_I}_n, t} &= \ket{\Psi_I, t_0} - \frac{i}{\hbar}\int_{t_0}^t \op{H}_I (t^\prime) \ket{\Psi_I, t_0}\,dt^\prime - \frac{1}{\hbar^2} \int_{t_0}^t \int_{t_0}^{t^\prime} \op{H}_I(t^\prime) \op{H}_I(t^{\prime\prime})\ket{\Psi_I, t_0} \,dt^{\prime\prime}\,dt^\prime + \Od{\op{H}_I(t)^3} \\
\bra{{\Psi_I}_n, t} &= \bra{\Psi_I, t_0} - \frac{i}{\hbar}\int_{t_0}^t \bra{\Psi_I, t_0}{\op{H}_I} (t^\prime)\,dt^\prime - \frac{1}{\hbar^2} \int_{t_0}^t \int_{t_0}^{t^\prime} \bra{\Psi_I, t_0} {\op{H}_I}(t^{\prime\prime}){\op{H}_I}(t^\prime)\,dt^{\prime\prime}\,dt^\prime + \Od{\op{H}_I(t)^3} 
\end{align*}
so
\begin{align*}
\ket{{\Psi_I}_n, t} \bra{{\Psi_I}_n, t} \;\;=\;\;& \ket{{\Psi_I}_n, t_0} \bra{{\Psi_I}_n, t_0} 
\;\;-\;\; \frac{i}{\hbar} \int_{t_0}^t \left(\op{H}(t^\prime)\ket{{\Psi_I}_n, t_0}\bra{{\Psi_I}_n, t_0} - \ket{{\Psi_I}_n, t_0}\bra{{\Psi_I}_n, t_0} \op{H}_I (t^\prime)\right) \,dt^\prime \\
& -\;\; \frac{1}{\hbar^2} \int_{t_0}^t \int_{t_0}^{t^\prime} \left( \op{H}_I (t^\prime) \op{H}_I(t^{\prime\prime}) \ket{{\Psi_I}_n, t_0}\bra{{\Psi_I}_n, t_0} +\;\; \ket{{\Psi_I}_n, t_0}\bra{{\Psi_I}_n, t_0} \op{H}_I(t^{\prime\prime}) \op{H}_I(t^\prime)\right)\,dt^{\prime\prime}\,dt^\prime \\
&+\;\; \frac{1}{\hbar^2} \int_{t_0}^t \int_{t_0}^{t} \op{H}_I (t^\prime) \ket{{\Psi_I}_n, t_0}\bra{{\Psi_I}_n, t_0} \op{H}_I (t^{\prime\prime})\,dt^{\prime\prime}\,dt^\prime  \\
\;\;=\;\;& \ket{{\Psi_I}_n, t_0} \bra{{\Psi_I}_n, t_0}
\;\;-\;\; \frac{i}{\hbar} \int_{t_0}^t \left[\op{H}(t^\prime),\; \ket{{\Psi_I}_n, t_0}\bra{{\Psi_I}_n, t_0} \right] \,dt^\prime \\
& -\;\; \frac{1}{\hbar^2} \int_{t_0}^t \int_{t_0}^{t^\prime} \left\{ \op{H}_I (t^\prime) \op{H}_I(t^{\prime\prime}),\; \ket{{\Psi_I}_n, t_0}\bra{{\Psi_I}_n, t_0} \right\}\,dt^{\prime\prime}\,dt^\prime \\
&+\;\; \frac{1}{\hbar^2} \int_{t_0}^t \int_{t_0}^{t} \op{H}_I (t^\prime) \ket{{\Psi_I}_n, t_0}\bra{{\Psi_I}_n, t_0} \op{H}_I (t^{\prime\prime})\,dt^{\prime\prime}\,dt^\prime.
\end{align*}
Then
\begin{align}
\label{eq.rhoevol}
\op{\rho}(t) \;\;=\;\;& \op{\rho}\left(t_0\right)
\;\;-\;\; \frac{i}{\hbar} \int_{t_0}^t \left[\op{H}(t^\prime),\; \op{\rho}(t_0) \right] \,dt^\prime \\
& -\;\; \frac{1}{\hbar^2} \int_{t_0}^t \int_{t_0}^{t^\prime} \left\{ \op{H}_I (t^\prime) \op{H}_I(t^{\prime\prime}),\; \op{\rho}(t_0) \right\}\,dt^{\prime\prime}\,dt^\prime \nonumber\\
&+\;\; \frac{1}{\hbar^2} \int_{t_0}^t \int_{t_0}^{t} \op{H}_I (t^\prime) \op{\rho}(t_0) \op{H}_I (t^{\prime\prime})\,dt^{\prime\prime}\,dt^\prime \nonumber.
\end{align}

\section{Eigenvalue corrections from perturbation theory}

The goal of this section is to compute the second-order perturbative corrections to the eigenvalues of an operator. We use the approach known as time-independent perturbation theory.

Suppose \(\op{\rho}\) is an operator with eigenstates \(\ket{\Psi_n}\) and eigenvalues \(\sigma_n\). Suppose we have the asymptotic expansions
\begin{align*}
\op{\rho} &= \op{\rho}_0 + \lambda \op{\rho}_1 + \lambda^2 \op{\rho}_2 + \Od{\lambda^3} \\
\ket{\Psi_n} &= \ket{\Psi_n^{(0)}} + \lambda \ket{\Psi_n^{(1)}} + \lambda^2 \ket{\Psi_n^{(2)}} + \Od{\lambda^3} \\
\sigma_n &= \sigma_n^{(0)} + \lambda \sigma_n^{(1)} + \lambda^2 \sigma_n^{(2)} + \Od{\lambda^3}
\end{align*}
where \(\left\{\ket{\Psi_n^{(0)}}\right\}\) is an orthonormal basis for the Hilbert space.

We have \(\op{\rho} \ket{\Psi_n} = \sigma_n \ket{\Psi_n}\). In the zeroth order of \(\lambda\) this gives \(\op{\rho}_0 \ket{\Psi_n^{(0)}} = \sigma_n^{(0)} \ket{\Psi_n^{(0)}}\), that is, \(\ket{\Psi_n^{(0)}}\) is an eigenstate of \(\op{\rho}_0\) with eigenvalue \(\sigma_n^{(0)}\).

Now, we have
\begin{align*}
\sigma_n \ket{\Psi_n} &= \op{\rho}\ket{\Psi_n}\\
\left(\sigma_n^{(0)} - \sigma_m^{(0)} + \lambda \sigma_n^{(1)} + \lambda^2 \sigma_n^{(2)} + \Od{\lambda^3} \right)\braket{\Psi_m^{(0)}}{\Psi_n} &=\bra{\Psi_m^{(0)}} \lambda \op{\rho}_1 + \lambda^2 \op{\rho}_2 + \Od{\lambda^3} \ket{\Psi_n} \\
\braket{\Psi_m^{(0)}}{\Psi_n} &=\frac{\bra{\Psi_m^{(0)}} \lambda \op{\rho}_1 + \lambda^2 \op{\rho}_2 + \Od{\lambda^3} \ket{\Psi_n}}{\sigma_n^{(0)} - \sigma_m^{(0)} + \lambda \sigma_n^{(1)} + \lambda^2 \sigma_n^{(2)} + \Od{\lambda^3}}.
\end{align*}
We will next proceed to get rid of the denominator by using the Taylor expansion \(\frac{1}{1+x} = 1-x +\Od{x^2}\).

Let \(D_n\) be the indices degenerate with \(n\). That is, let \(D_n = \{m : \sigma_m^{(0)} = \sigma_n^{(0)}\}\).
Consider the case where \(m \in D_n\). Let \(\delta\) be a small real number. In the limit as \(\delta \to 0\),
\begin{align*}
\braket{\Psi_m^{(0)}}{\Psi_n} &=\frac{\bra{\Psi_m^{(0)}} \lambda \op{\rho}_1 + \lambda^2 \op{\rho}_2 + \Od{\lambda^3} \ket{\Psi_n}}{\lambda \sigma_n^{(1)} + \lambda^2 \sigma_n^{(2)} + \Od{\lambda^3} + \lambda\delta} \\
&=\frac{\bra{\Psi_m^{(0)}} \lambda \op{\rho}_1 + \lambda^2 \op{\rho}_2 + \Od{\lambda^3} \ket{\Psi_n}}{\lambda \left(\sigma_n^{(1)}+ \delta\right)\left(1 + \lambda \frac{\sigma_n^{(2)}}{\sigma_n^{(1)}+\delta} + \Od{\lambda^2}\right) } \\
&=\frac{\bra{\Psi_m^{(0)}} \op{\rho}_1 + \lambda \op{\rho}_2 + \Od{\lambda^2} \ket{\Psi_n}}{\sigma_n^{(1)}+ \delta}\left(1 - \lambda \frac{\sigma_n^{(2)}}{\sigma_n^{(1)}+\delta} + \Od{\lambda^2}\right) \\
&= \frac{\bra{\Psi_m^{(0)}} \left(\op{\rho}_1 + \lambda \op{\rho}_2 + \Od{\lambda^2} \right)\left(\ket{\Psi_n^{(0)}} + \lambda \ket{\Psi_n^{(1)}} + \Od{\lambda^2}\right)}{\sigma_n^{(1)}+\delta}
\left(1 - \lambda \frac{\sigma_n^{(2)}}{\sigma_n^{(1)}+\delta} + \Od{\lambda^2}\right) \\
&=\frac{\bra{\Psi_m^{(0)}} \op{\rho}_1 \ket{\Psi_n^{(0)}}}{\sigma_n^{(1)} + \delta} + \lambda \left[\frac{\bra{\Psi_m^{(0)}}\op{\rho}_2\ket{\Psi_n^{(0)}}}{\sigma_n^{(1)} + \delta} + \frac{\bra{\Psi_m^{(0)}}\op{\rho}_2\ket{\Psi_n^{(1)}}}{\sigma_n^{(1)} + \delta} - \frac{\sigma_n^{(2)}\bra{\Psi_m^{(0)}}\op{\rho}_2\ket{\Psi_n^{(0)}}}{\left(\sigma_n^{(1)} + \delta\right)^2}\right] + \Od{\lambda^2}
\end{align*}
(The \(\delta\) ensured that we did not divide by zero if \(\sigma_n^{(1)}=0\).) Now consider the case where \(m \not\in D_n\). Then
\begin{align*}
\braket{\Psi_m^{(0)}}{\Psi_n} &=\frac{\bra{\Psi_m^{(0)}} \lambda \op{\rho}_1 + \Od{\lambda^2} \ket{\Psi_n}}{\left(\sigma_n^{(0)} - \sigma_m^{(0)}\right)\left(1+\Od{\lambda}\right)} \\
&=\frac{\bra{\Psi_m^{(0)}} \lambda \op{\rho}_1 + \Od{\lambda^2} \ket{\Psi_n}}{\sigma_n^{(0)} - \sigma_m^{(0)}}\left(1 + \Od{\lambda}\right) \\
&= \lambda\frac{\bra{\Psi_m^{(0)}} \op{\rho}_1 \ket{\Psi_n}}{\sigma_n^{(0)} - \sigma_m^{(0)}} + \Od{\lambda^2}.
\end{align*}
Putting this all together,
\begin{align}\label{eq.tiptmaineq}
\ket{\Psi_n} &= \sum_{m \in D_n} \ket{\Psi_m^{(0)}} \braket{\Psi_m^{(0)}}{\Psi_n} + \sum_{m \not\in D_n} \ket{\Psi_m^{(0)}} \braket{\Psi_m^{(0)}}{\Psi_n} \\
&= \sum_{m \in D_n} \ket{\Psi_m^{(0)}} \lim_{\delta\to 0} \frac{\bra{\Psi_m^{(0)}} \op{\rho}_1 \ket{\Psi_n^{(0)}}}{\sigma_n^{(1)} + \delta}
\nonumber\\&\hspace{1.5em}+ \lambda\;\;
\Biggl(\;\sum_{m \in D_n} \ket{\Psi_m^{(0)}} \lim_{\delta \to 0}\left[\frac{\bra{\Psi_m^{(0)}}\op{\rho}_2\ket{\Psi_n^{(0)}}}{\sigma_n^{(1)} + \delta} + \frac{\bra{\Psi_m^{(0)}}\op{\rho}_1\ket{\Psi_n^{(1)}}}{\sigma_n^{(1)} + \delta} - \frac{\sigma_n^{(2)}\bra{\Psi_m^{(0)}}\op{\rho}_0\ket{\Psi_n^{(0)}}}{\left(\sigma_n^{(1)} + \delta\right)^2}\right] 
\nonumber\\&\hspace{6em}+ \sum_{m \not\in D_n} \ket{\Psi_m^{(0)}}\frac{\bra{\Psi_m^{(0)}} \op{\rho}_1 \ket{\Psi_n}}{\sigma_n^{(0)} - \sigma_m^{(0)}}
\;\Biggr)
\nonumber\\&\hspace{1.5em}+ \Od{\lambda^2}\nonumber
\end{align}

In the zeroth order of \(\lambda\), equation (\ref{eq.tiptmaineq}) gives
\[
\ket{\Psi_n^{(0)}} = \sum_{m \in D_n} \ket{\Psi_m^{(0)}} \lim_{\delta\to0}\frac{\bra{\Psi_m^{(0)}} \op{\rho}_1 \ket{\Psi_n^{(0)}}}{\sigma_n^{(1)} + \delta}.
\]
Multiplying by \(\bra{\Psi_n^{(0)}}\) we get
\begin{equation}\label{eq.tipt.foval}
\boxed{\sigma_n^{(1)} = \bra{\Psi_n^{(0)}}\op{\rho}_1\ket{\Psi_n^{(0)}}.}
\end{equation}
If we instead multiply by \(\bra{\Psi_k^{(0)}}\) where \(k \in D_n\) but \(k \neq n\) we get
\begin{equation}\label{eq.tipt.fodiag}
\boxed{0 = \bra{\Psi_k^{(0)}}\op{\rho}_1\ket{\Psi_n^{(0)}}, \hspace{3em} k \in D_n,\hspace{3em} m \neq n.}
\end{equation}
In other words, the matrix given by \(\bra{\Psi_k^{(0)}}\op{\rho}_1\ket{\Psi_j^{(0)}}\) is diagonal on \(k,j\in D_n\), for any \(n\).

In the first order of \(\lambda\), equation (\ref{eq.tiptmaineq}) says
\[
\hspace{-2.5em}
\ket{\Psi_n^{(1)}} = \sum_{m \in D_n} \ket{\Psi_m^{(0)}} \lim_{\delta \to 0}\left[\frac{\bra{\Psi_m^{(0)}}\op{\rho}_2\ket{\Psi_n^{(0)}}}{\sigma_n^{(1)} + \delta} + \frac{\bra{\Psi_m^{(0)}}\op{\rho}_1\ket{\Psi_n^{(1)}}}{\sigma_n^{(1)} + \delta} - \frac{\sigma_n^{(2)}\bra{\Psi_m^{(0)}}\op{\rho}_0\ket{\Psi_n^{(0)}}}{\left(\sigma_n^{(1)} + \delta\right)^2}\right] 
+ \sum_{m \not\in D_n} \ket{\Psi_m^{(0)}}\frac{\bra{\Psi_m^{(0)}} \op{\rho}_1 \ket{\Psi_n}}{\sigma_n^{(0)} - \sigma_m^{(0)}}.
\]
Multiplying by \(\bra{\Psi_k^{(0)}}\) where \(k \not\in D_n\) gives
\[
\braket{\Psi_k^{(0)}}{\Psi_n^{(1)}} = \frac{\bra{\Psi_k^{(0)}} \op{\rho}_1 \ket{\Psi_n^{(0)}}}{\sigma_n^{(0)}-\sigma_k^{(0)}}.
\]
If we instead multiply by \(\bra{\Psi_n^{(0)}}\) we get
\[
0 = \braket{\Psi_n^{(0)}}{\Psi_n^{(1)}} = \lim_{\delta \to 0} \left[\frac{\bra{\Psi_n^{(0)}}\op{\rho}_2\ket{\Psi_n^{(0)}}}{\sigma_n^{(1)} + \delta} + \frac{\bra{\Psi_n^{(0)}}\op{\rho}_1\ket{\Psi_n^{(1)}}}{\sigma_n^{(1)} + \delta} - \frac{\sigma_n^{(2)}\left(\sigma_n^{(1)}+\delta\right)}{\left(\sigma_n^{(1)} + \delta\right)^2}\right]
\]
so
\begin{align*}
\hspace{-2em}\sigma_n^{(2)} &= \bra{\Psi_n^{(0)}}\op{\rho}_2\ket{\Psi_n^{(0)}} + \bra{\Psi_n^{(0)}}\op{\rho}_1\ket{\Psi_n^{(1)}} - \braket{\Psi_n^{(0)}}{\Psi_n^{(1)}} \sigma_n^{(1)}\\
&= \bra{\Psi_n^{(0)}}\op{\rho}_2\ket{\Psi_n^{(0)}} + \sum_{m \in D_n} \bra{\Psi_n^{(0)}}\op{\rho}_1\ket{\Psi_m^{(0)}}\braket{\Psi_m^{(0)}}{\Psi_n^{(1)}} + \sum_{m \not\in D_n} \bra{\Psi_n^{(0)}}\op{\rho}_1\ket{\Psi_m^{(0)}}\braket{\Psi_m^{(0)}}{\Psi_n^{(1)}} - \braket{\Psi_n^{(0)}}{\Psi_n^{(1)}} \sigma_n^{(1)}\\
&= \bra{\Psi_n^{(0)}}\op{\rho}_2\ket{\Psi_n^{(0)}} + \sigma_n^{(1)}\braket{\Psi_n^{(0)}}{\Psi_n^{(1)}} +  \sum_{m \not\in D_n} \bra{\Psi_n^{(0)}}\op{\rho}_1\ket{\Psi_m^{(0)}}\frac{\bra{\Psi_m^{(0)}} \op{\rho}_1 \ket{\Psi_n^{(0)}}}{\sigma_n^{(0)}-\sigma_m^{(0)}}- \braket{\Psi_n^{(0)}}{\Psi_n^{(1)}} \sigma_n^{(1)}
\end{align*}
hence
\begin{equation}\label{eq.tipt.soval}
\boxed{\sigma_n^{(2)} = \bra{\Psi_n^{(0)}}\op{\rho}_2\ket{\Psi_n^{(0)}} +  \sum_{m \not\in D_n} \frac{\left|\bra{\Psi_m^{(0)}} \op{\rho}_1 \ket{\Psi_n^{(0)}}\right|^2}{\sigma_n^{(0)}-\sigma_m^{(0)}}.}
\end{equation}


\section{The change in entropy}

Both von Neumann and \(N\)-Tsallis entropy can be calculated from the eigenvalues of the density operator. We therefore study these eigenvalues with perturbation theory.

Suppose that, at some fixed time \(t\), the density operator takes the form
\begin{equation}\label{eq.densdecomp}
\op{\rho}(t) = \sum_n \sigma_n \ket{\Psi_n}\bra{\Psi_n}
\end{equation}
where \(\left\{\ket{\Psi_n^{(0)}}\right\}\) is an orthonormal basis for the Hilbert space.
Suppose we have asymptotic expansions
\begin{align*}
\op{\rho}(t) &= \op{\rho}_0 + \lambda \op{\rho}_1 + \lambda^2 \op{\rho}_2 + \Od{\lambda^3} \\
\ket{\Psi_n} &= \ket{\Psi_n^{(0)}} + \lambda \ket{\Psi_n^{(1)}} + \lambda^2 \ket{\Psi_n^{(2)}} + \Od{\lambda^3} \\
\sigma_n &= \sigma_n^{(0)} + \lambda \sigma_n^{(1)} + \lambda^2 \sigma_n^{(2)} + \Od{\lambda^3}
\end{align*}
The above is indeed true if we take \(\lambda\) to be proportional to \(\op{H}_I(t)\) and take
\begin{align}
\label{eq.rho0}\op{\rho}_0 &= \op{\rho}\left(t_0\right) \\
\label{eq.rho1}\op{\rho}_1 &= -\frac{i}{\hbar} \int_{t_0}^t \left[\op{H}(t^\prime),\; \op{\rho}_0 \right] \,dt^\prime\\
\label{eq.rho2}\op{\rho}_2 &= - \frac{1}{\hbar^2} \int_{t_0}^t \int_{t_0}^{t^\prime} \left\{ \op{H}_I (t^\prime) \op{H}_I(t^{\prime\prime}),\; \op{\rho}_0 \right\}\,dt^{\prime\prime}\,dt^\prime 
+ \frac{1}{\hbar^2} \int_{t_0}^t \int_{t_0}^{t} \op{H}_I (t^\prime) \,\op{\rho}_0\, \op{H}_I (t^{\prime\prime})\,dt^{\prime\prime}\,dt^\prime.
\end{align}
as per equation (\ref{eq.rhoevol}).

\subsection{The first-order correction vanishes}

Combining (\ref{eq.tipt.foval}) and (\ref{eq.rho1}), the first-order eigenvalue correction is
\[
\sigma_n^{(1)} = -\frac{i}{\hbar} \int_{t_0}^t \bra{\Psi_n^{(0)}} \left[\op{H}(t^\prime),\; \op{\rho}_0 \right] \ket{\Psi_n^{(0)}}\,dt^\prime.
\]
Since
\[
\bra{\Psi_n^{(0)}} \op{H}(t^\prime) \op{\rho}_0 \ket{\Psi_n^{(0)}} = 
\sigma_n^{(0)} \bra{\Psi_n^{(0)}} \op{H}(t^\prime) \ket{\Psi_n^{(0)}} =
\bra{\Psi_n^{(0)}} \op{\rho}_0 \op{H}(t^\prime) \ket{\Psi_n^{(0)}}
\]
we have \(\sigma_n^{(1)} = 0\).

\subsection{Von Neumann entropy}

The Von Neumann entropy for the state (\ref{eq.densdecomp}) is defined to be
\[
S = \sum_{n} \sigma_n \log \sigma_n.
\]
Let \(A = \{n : \sigma_n^{(0)} \neq 0\}\). Then
\begin{align*}
S &= -\sum_{n \in A} \sigma_n \log \sigma_n-\sum_{n \not\in A} \sigma_n \log \sigma_n \\
&= - \sum_{n \in A} \sigma_n \log \left(\sigma_n^{(0)} \left(1 + \Od{\lambda^2}\right)\right)  - \sum_{n \not\in A} \sigma_n \log\left(\lambda^2\sigma_n^{(2)}\left(1 + \Od{\lambda}\right)\right) \\
&= - \sum_{n \in A} \left(\sigma_n^{(0)} + \Od{\lambda^2}\right) \left(\log \sigma_n^{(0)}+\Od{\lambda^2}\right) - \sum_{n \not\in A} \sigma_n \left(\log\lambda^2+\log\sigma_n^{(2)}+ \Od{\lambda}\right) \\
&= - \sum_{n \in A} \sigma_n^{(0)} \log \sigma_n^{(0)} -\sum_{n \not\in A} \sigma_n \log\lambda^2+ \Od{\lambda^2}.
\end{align*}
The first term is
\[S_0 = - \sum_{n \in A} \sigma_n^{(0)} \log \sigma_n^{(0)}\]
which is the von Neumann entropy associated with the unperturbed density operator \(\op{\rho}_0 = \op{\rho}\left(t_0\right)\). Thus the change in entropy due to the interaction is
\begin{equation}\label{eq.entropyresultnotA}
{\Delta S = S-S_0 = \log\frac{1}{\lambda^2}\sum_{n \not\in A} \sigma_n + \Od{\lambda^2} = \lambda^2 \log\frac{1}{\lambda^2} \sum_{n \not\in A}\sigma_n^{(2)}+ \Od{\lambda^2}.}
\end{equation}
Each \(\sigma_n\) is non-negative because it is an eigenvalue of a density operator; thus the entire change in entropy is non-negative to the leading order (assuming this term does not vanish).

We can recognize that \(\sum_{n\in A} \sigma_n + \sum_{n \not\in A} \sigma_n = 1\) and so
\[
\Delta S = \left(1- \sum_{n\in A} \left[\sigma_n^{(0)}+\lambda^2 \sigma_n^{(2)}\right]\right)\log\frac{1}{\lambda^2} + \Od{\lambda^2}
\]
and, since \(\sum_{n \in A} \sigma_n^{(0)} = 1\),
\begin{equation}\label{eq.entropyresultwithsigma2}
{\Delta S = -\lambda^2 \log\frac{1}{\lambda^2}\sum_{n \in A} \sigma_n^{(2)} + \Od{\lambda^2}.}
\end{equation}
Substituting in (\ref{eq.tipt.soval}),
\[
\Delta S = -\lambda^2 \log\frac{1}{\lambda^2}\left(\sum_{n \in A} \bra{\Psi_n^{(0)}}\op{\rho}_2\ket{\Psi_n^{(0)}} +  \sum_{n \in A} \sum_{m \not\in D_n} \frac{\left|\bra{\Psi_m^{(0)}} \op{\rho}_1 \ket{\Psi_n^{(0)}}\right|^2}{\sigma_n^{(0)}-\sigma_m^{(0)}} \right)+ \Od{\lambda^2}.
\]
Now, in the second term with the double sum, for every term \(\frac{\left|\bra{\Psi_m^{(0)}} \op{\rho}_1 \ket{\Psi_n^{(0)}}\right|^2}{\sigma_n^{(0)}-\sigma_m^{(0)}}\) there is an equal but opposite term \(\frac{\left|\bra{\Psi_m^{(0)}} \op{\rho}_1 \ket{\Psi_n^{(0)}}\right|^2}{\sigma_m^{(0)}-\sigma_n^{(0)}}\). Thus this double sum evaluates to zero. We end up with
\begin{equation}\label{eq.entropyresultwithrho2}
\boxed{\Delta S = -\lambda^2 \log\frac{1}{\lambda^2}\sum_{n \in A} \bra{\Psi_n^{(0)}}\op{\rho}_2\ket{\Psi_n^{(0)}} + \Od{\lambda^2}.}
\end{equation}
Substituting in (\ref{eq.rho2}) and using that
\[
\bra{\Psi_n^{(0)}} \left\{ \op{H}_I (t^\prime) \op{H}_I(t^{\prime\prime}),\; \op{\rho}_0 \right\} \ket{\Psi_n^{(0)}} = 2 \sigma_n^{(0)} \bra{\Psi_n^{(0)}} \op{H}_I (t^\prime) \op{H}_I(t^{\prime\prime}) \ket{\Psi_n^{(0)}}
\]
we get
\begin{align}\label{entropyresultwithHI}
\Delta S &= \frac{1}{\hbar^2}\lambda^2 \log\frac{1}{\lambda^2}\sum_{n \in A} \Bigg(2 \sigma_n^{(0)} \int_{t_0}^t \int_{t_0}^{t^\prime} \bra{\Psi_n^{(0)}} \op{H}_I (t^\prime) \op{H}_I(t^{\prime\prime}) \ket{\Psi_n^{(0)}}\,dt^{\prime\prime}\,dt^\prime \\
&\hspace{10em}- \int_{t_0}^t \int_{t_0}^{t} \bra{\Psi_n^{(0)}}\op{H}_I (t^\prime) \,\op{\rho}_0\, \op{H}_I (t^{\prime\prime})\ket{\Psi_n^{(0)}}\,dt^{\prime\prime}\,dt^\prime \Bigg) + \Od{\lambda^2}.\nonumber
\end{align}
Putting in (\ref{eq.rho0}),
\begin{align*}
\Delta S &=\frac{1}{\hbar^2}\lambda^2 \log\frac{1}{\lambda^2}\sum_{n \in A} \Bigg(2 \sigma_n^{(0)} \int_{t_0}^t \int_{t_0}^{t^\prime} \bra{\Psi_n^{(0)}} \op{H}_I (t^\prime) \op{H}_I(t^{\prime\prime}) \ket{\Psi_n^{(0)}}\,dt^{\prime\prime}\,dt^\prime \\
&\hspace{10em}-\sum_{m \in A} \sigma_m^{(0)}\int_{t_0}^t \int_{t_0}^{t} \bra{\Psi_n^{(0)}}\op{H}_I (t^\prime) \ket{\Psi_m^{(0)}}\bra{\Psi_m^{(0)}} \op{H}_I (t^{\prime\prime})\ket{\Psi_n^{(0)}}\,dt^{\prime\prime}\,dt^\prime \Bigg) + \Od{\lambda^2}\\
&=\frac{1}{\hbar^2}\lambda^2 \log\frac{1}{\lambda^2} \Bigg(\sum_{n \in A}2 \sigma_n^{(0)} \int_{t_0}^t \int_{t_0}^{t^\prime} \bra{\Psi_n^{(0)}} \op{H}_I (t^\prime) \op{H}_I(t^{\prime\prime}) \ket{\Psi_n^{(0)}}\,dt^{\prime\prime}\,dt^\prime \\
&\hspace{10em}-\sum_{m \in A} \;\sum_{n \in A}\sigma_m^{(0)}\int_{t_0}^t \int_{t_0}^{t} \bra{\Psi_m^{(0)}} \op{H}_I (t^{\prime\prime})\ket{\Psi_n^{(0)}}\bra{\Psi_n^{(0)}}\op{H}_I (t^\prime) \ket{\Psi_m^{(0)}}\,dt^{\prime\prime}\,dt^\prime \Bigg) + \Od{\lambda^2}\\
&=\frac{1}{\hbar^2}\lambda^2 \log\frac{1}{\lambda^2} \Bigg(\sum_{n \in A}2 \sigma_n^{(0)} \int_{t_0}^t \int_{t_0}^{t^\prime} \bra{\Psi_n^{(0)}} \op{H}_I (t^\prime) \op{H}_I(t^{\prime\prime}) \ket{\Psi_n^{(0)}}\,dt^{\prime\prime}\,dt^\prime \\
&\hspace{10em}-\sum_{m \in A} \sigma_m^{(0)}\int_{t_0}^t \int_{t_0}^{t} \bra{\Psi_m^{(0)}} \op{H}_I (t^{\prime\prime})\sum_{n \in A}\ket{\Psi_n^{(0)}}\bra{\Psi_n^{(0)}}\op{H}_I (t^\prime) \ket{\Psi_m^{(0)}}\,dt^{\prime\prime}\,dt^\prime \Bigg) + \Od{\lambda^2}\\
&=\frac{1}{\hbar^2}\lambda^2 \log\frac{1}{\lambda^2} \Bigg(\sum_{n \in A}2 \sigma_n^{(0)} \int_{t_0}^t \int_{t_0}^{t^\prime} \bra{\Psi_n^{(0)}} \op{H}_I (t^\prime) \op{H}_I(t^{\prime\prime}) \ket{\Psi_n^{(0)}}\,dt^{\prime\prime}\,dt^\prime \\
&\hspace{10em}-\sum_{m \in A} \sigma_m^{(0)}\int_{t_0}^t \int_{t_0}^{t} \bra{\Psi_m^{(0)}} \op{H}_I (t^{\prime\prime})\op{H}_I (t^\prime) \ket{\Psi_m^{(0)}}\,dt^{\prime\prime}\,dt^\prime \Bigg) + \Od{\lambda^2} \\
&=\frac{1}{\hbar^2}\lambda^2 \log\frac{1}{\lambda^2} \sum_{n \in A} \sigma_n^{(0)} \Bigg(2\int_{t_0}^t \int_{t_0}^{t^\prime} \bra{\Psi_n^{(0)}} \op{H}_I (t^\prime) \op{H}_I(t^{\prime\prime}) \ket{\Psi_n^{(0)}}\,dt^{\prime\prime}\,dt^\prime  \\
&\hspace{14em}-\int_{t_0}^t \int_{t_0}^{t} \bra{\Psi_n^{(0)}} \op{H}_I (t^{\prime\prime})\op{H}_I (t^{\prime}) \ket{\Psi_n^{(0)}}\,dt^{\prime\prime}\,dt^\prime \Bigg) + \Od{\lambda^2} \\
&=\frac{1}{\hbar^2}\lambda^2 \log\frac{1}{\lambda^2} \sum_{n \in A} \sigma_n^{(0)} \Bigg(2\int_{t_0}^t \int_{t_0}^{t^\prime} \bra{\Psi_n^{(0)}} \op{H}_I (t^\prime) \op{H}_I(t^{\prime\prime}) \ket{\Psi_n^{(0)}}\,dt^{\prime\prime}\,dt^\prime  \\
&\hspace{14em}-\int_{t_0}^t \int_{t_0}^{t^\prime} \bra{\Psi_n^{(0)}} \op{H}_I (t^{\prime\prime})\op{H}_I (t^{\prime}) \ket{\Psi_n^{(0)}}\,dt^{\prime\prime}\,dt^\prime \\
&\hspace{14em}-\int_{t_0}^t \int_{t^\prime}^{t} \bra{\Psi_n^{(0)}} \op{H}_I (t^{\prime\prime})\op{H}_I (t^{\prime}) \ket{\Psi_n^{(0)}}\,dt^{\prime\prime}\,dt^\prime\Bigg) + \Od{\lambda^2} \\
&=\frac{1}{\hbar^2}\lambda^2 \log\frac{1}{\lambda^2} \sum_{n \in A} \sigma_n^{(0)} \Bigg(2\int_{t_0}^t \int_{t_0}^{t^\prime} \bra{\Psi_n^{(0)}} \op{H}_I (t^\prime) \op{H}_I(t^{\prime\prime}) \ket{\Psi_n^{(0)}}\,dt^{\prime\prime}\,dt^\prime  \\
&\hspace{14em}-\int_{t_0}^t \int_{t_0}^{t^\prime} \bra{\Psi_n^{(0)}} \op{H}_I (t^{\prime\prime})\op{H}_I (t^{\prime}) \ket{\Psi_n^{(0)}}\,dt^{\prime\prime}\,dt^\prime \\
&\hspace{14em}-\int_{t_0}^t \int_{t_0}^{t^\prime} \bra{\Psi_n^{(0)}} \op{H}_I (t^{\prime})\op{H}_I (t^{\prime\prime}) \ket{\Psi_n^{(0)}}\,dt^{\prime\prime}\,dt^\prime\Bigg) + \Od{\lambda^2} \\
&=\frac{1}{\hbar^2}\lambda^2 \log\frac{1}{\lambda^2} \sum_{n \in A} \sigma_n^{(0)} \Bigg(\int_{t_0}^t \int_{t_0}^{t^\prime} \bra{\Psi_n^{(0)}} \left[\op{H}_I (t^\prime),\; \op{H}_I(t^{\prime\prime}) \right]\ket{\Psi_n^{(0)}}\,dt^{\prime\prime}\,dt^\prime \Bigg) + \Od{\lambda^2}. \\
\end{align*}
We have derived the rather interesting result that
\begin{equation}
\boxed{\Delta S = \frac{1}{\hbar^2}\lambda^2 \log\frac{1}{\lambda^2} \sum_{n \in A} \sigma_n^{(0)} \Bigg(\int_{t_0}^t \int_{t_0}^{t^\prime} \bra{\Psi_n^{(0)}} \left[\op{H}_I (t^\prime),\; \op{H}_I(t^{\prime\prime}) \right]\ket{\Psi_n^{(0)}}\,dt^{\prime\prime}\,dt^\prime \Bigg) + \Od{\lambda^2}.}
\end{equation}


\subsection{\(N\)-Tsallis entropy}

(to be added)


\appendix
\newpage
\hrule
\vspace{3em}
{\Huge Appendices}
\vspace{3em}
\hrule
\vspace{2em}

These are extra calculations that are not relevant to the above discussion. They should be removed from documents to be shared. (I don't delete them now because they could be useful in future work.)


\section{The first-order approximation for pure states in scattering}

We will continue for scattering between two particles. We will assume that the particles are distinguishable so that we don't need to restrict ourselves to symmetric or antisymmetric states. We will also assume that the particles interact in a way that only depends on the distance between them. That is, in the position basis, we can write
\begin{equation}\label{eq.Vposbasis}
\bra{x_1, x_2}\op{V}(t)\ket{\psi} = \int_{x_1,x_2} {d^3 x_1 \,d^3x_2}\; V(x_1 - x_2) \braket{x_1, x_2}{\psi}.
\end{equation}
(Note that this is not time dependent.) Take \(\op{H}_0\) to be the free particle Hamiltonian.

It will be easiest to work in the momentum basis. We will do our calculations for momentum eigenstates---that is, plane waves. Write momentum eigenstates as \(\ket{k_1, k_2}\). In the position basis these have wavefunctions
\begin{equation}\label{eq.momentuminpos}
\braket{x_1, x_2}{k_1, k_2} = \frac{1}{(2\pi)^3}\;e^{ik_1\cdot x_1 + ik_2\cdot x_2}.
\end{equation}
Here \(x_1\) and \(x_2\) are the (3-vector) positions of the first and second particles and \(k_1\) and \(k_2\) are the (3-vector) momenta of the first and second particle. Also, we note that \(\ket{k_1, k_2}\) has energy \(\frac{1}{2m_1}{k_1}^2 + \frac{1}{2m_2}{k_2}^2\) and so
\begin{equation}
\label{eq.timeevonk}
\op{U}(t_0, t) \ket{k_1, k_2} = e^{-\frac{i}{\hbar}\left(t-t_0\right)\left(\frac{1}{2m_1}{k_1}^2 + \frac{1}{2m_2}{k_2}^2\right)} \ket{k_1, k_2}.
\end{equation}

Let \(\widetilde{V}(p)\) be the Fourier transform of \(V(x_1 - x_2)\) so that
\begin{equation}
\label{eq.ftV}
V(x_1 - x_2) = \int_{-\infty}^{\infty} \frac{d^3 p}{(2 \pi)^3} \widetilde{V}(p) e^{ip\cdot(x_1 - x_2)}.
\end{equation}

We now consider (\ref{eq.firsttdpe}) and take the limits \(t_0 \to -\infty\) and \(t \to \infty\) because, in scattering, we measure states long before and long after the scattering occurs. Then, dropping the higher-order terms,
\[
\ket{\Psi_I, \infty} = \ket{\Psi_I, -\infty} - \frac{i}{\hbar} \int_{-\infty}^\infty \op{H}_I(t)\ket{\Psi_I, -\infty} \,dt.
\]
Take the initial state to be \(\ket{\Psi_I, -\infty} = \ket{k_1, k_2}\). Multiplying by the bra \(\bra{k_1^\prime, k_2^\prime}\),
\begin{align}
\braket{k_1^\prime, k_2^\prime}{\Psi_I, \infty} &= \braket{k_1^\prime, k_2^\prime}{k_1, k_2} - \frac{i}{\hbar} \int_{-\infty}^\infty \bra{k_1^\prime, k_2^\prime} \op{H}_I(t)\ket{k_1, k_2} \,dt \nonumber\\
&= \delta^3(k_1^\prime - k_1)\delta^3(k_2^\prime - k_2) - \frac{i}{\hbar} \int_{-\infty}^\infty \bra{k_1^\prime, k_2^\prime} \op{H}_I(t)\ket{k_1, k_2} \,dt. \label{eq.intermed}
\end{align}
This motivates us to compute the matrix element \(\bra{k_1^\prime, k_2^\prime} \op{H}_I(t)\ket{k_1, k_2}\). We can start by using (\ref{eq.timeevonk}):
\begin{align*}
\bra{k_1^\prime, k_2^\prime} \op{H}_I(t)\ket{k_1, k_2}
&= \bra{k_1^\prime, k_2^\prime} \op{U}(t_0, t)^\dagger \op{V}(t) \op{U}(t_0, t) \ket{k_1, k_2} \nonumber\\
&= \bra{k_1^\prime, k_2^\prime} e^{\frac{i}{\hbar}\left(t-t_0\right)\left(\frac{1}{2m_1}{k_1^\prime}^2 + \frac{1}{2m_2}{k_2^\prime}^2\right)} \op{V}(t) e^{-\frac{i}{\hbar}\left(t-t_0\right)\left(\frac{1}{2m_1}{k_1}^2 + \frac{1}{2m_2}{k_2}^2\right)} \ket{k_1, k_2} \nonumber\\
&= e^{-\frac{i}{\hbar}\left(t-t_0\right)\left(\frac{1}{2m_1}{k_1^\prime}^2 + \frac{1}{2m_2}{k_2^\prime}^2 - \frac{1}{2m_1}{k_1}^2 - \frac{1}{2m_2}{k_2}^2\right)} \bra{k_1^\prime, k_2^\prime} \op{V}(t) \ket{k_1, k_2}.
\end{align*}
We see that
\begin{align*}
\hspace{-1em}\bra{k_1^\prime, k_2^\prime} \op{V}(t) \ket{k_1, k_2} &= \bra{k_1^\prime, k_2^\prime} \left\{\int_{x_1, x_2} {d^3 x_1 d^3 x_2\;} \ket{x_1, x_2}\bra{x_1, x_2}\right\}\op{V}(t) \ket{k_1, k_2} \nonumber\\
&= \int_{x_1, x_2} {d^3 x_1 d^3 x_2\;} \braket{k_1^\prime, k_2^\prime}{x_1, x_2}\bra{x_1, x_2}\op{V}(t) \ket{k_1, k_2} \nonumber\\
&= \int_{x_1, x_2} \frac{d^3 x_1 d^3 x_2}{(2\pi)^6}\; e^{-ik_1^\prime\cdot x_1 - ik_2^\prime\cdot x_2}V(x_1 - x_2) e^{ik_1\cdot x_1 + ik_2\cdot x_2} &&\text{(by (\ref{eq.Vposbasis}) and (\ref{eq.momentuminpos}))}\nonumber\\
&= \int_{-\infty}^{\infty} \frac{d^3p}{(2\pi)^9}\int_{x_1, x_2} {d^3 x_1 d^3 x_2\;\;}e^{-ik_1^\prime\cdot x_1 - ik_2^\prime\cdot x_2}\,\widetilde{V}(p)\,e^{ip\cdot(x_1-x_2)} e^{ik_1\cdot x_1 + ik_2\cdot x_2}
&&\text{(by (\ref{eq.ftV}))}\nonumber\\
&= \int_{-\infty}^{\infty} \frac{d^3p}{(2\pi)^9}\widetilde{V}(p) \int_{x_1, x_2} {d^3 x_1 d^3 x_2\;\;}e^{i(k_1-k_1^\prime+p)\cdot x_1}e^{i(k_2-k_2^\prime-p)\cdot x_2}\nonumber\\
&= \int_{-\infty}^{\infty} \frac{d^3p}{(2\pi)^9}\widetilde{V}(p) \,(2\pi)^6\,\delta^3(k_1-k_1^\prime+p)\,\delta^3(k_2-k_2^\prime-p)\nonumber\\
&= \frac{1}{(2\pi)^3}\;\widetilde{V}(k_1^\prime-k_1) \;\delta^3(k_1^\prime-k_1-k_2^\prime+k_2).
\end{align*}
so 
\[
\bra{k_1^\prime, k_2^\prime} \op{H}_I(t)\ket{k_1, k_2}
=
e^{-\frac{i}{\hbar}\left(t-t_0\right)\left(\frac{1}{2m_1}{k_1^\prime}^2 + \frac{1}{2m_2}{k_2^\prime}^2 - \frac{1}{2m_1}{k_1}^2 - \frac{1}{2m_2}{k_2}^2\right)}\frac{1}{(2\pi)^3}\;\widetilde{V}(k_1^\prime-k_1) \;\delta^3(k_1^\prime-k_1-k_2^\prime+k_2)
\]
If we integrate over all time, we get a delta function:
\begin{equation}\label{eq.HIintegrated}
\hspace{-2em}
\int_{-\infty}^\infty dt \,\bra{k_1^\prime, k_2^\prime} \op{H}_I(t)\ket{k_1, k_2}
=
\frac{\hbar}{(2\pi)^2}\,\widetilde{V}(k_1^\prime-k_1)\,\delta\left({\textstyle\frac{1}{2m_1}{k_1^\prime}^2 + \frac{1}{2m_2}{k_2^\prime}^2 - \frac{1}{2m_1}{k_1}^2 - \frac{1}{2m_2}{k_2}^2}\right) \,\delta^3(k_1^\prime-k_1-k_2^\prime+k_2).
\end{equation}
Putting this into (\ref{eq.intermed}),
\begin{align}
\hspace{-1.5em}\braket{k_1^\prime, k_2^\prime}{\Psi_I, \infty}
%=&\;\; \delta^3(k_1^\prime - k_1)\delta^3(k_2^\prime - k_2) \nonumber\\&\;\;\;\; - \frac{i}{\hbar} \int_{-\infty}^\infty dt\;\frac{1}{(2\pi)^3}\;e^{-\frac{i}{\hbar}\left(t-t_0\right)\left(\frac{1}{2m_1}{k_1^\prime}^2 + \frac{1}{2m_2}{k_2^\prime}^2 - \frac{1}{2m_1}{k_1}^2 - \frac{1}{2m_2}{k_2}^2\right)} \widetilde{V}(k_1^\prime-k_1) \;\delta^3(k_1^\prime-k_1-k_2^\prime+k_2) \nonumber\\
=& \;\;\delta^3(k_1^\prime - k_1)\;\delta^3(k_2^\prime - k_2) \nonumber\\&\;\;\;\;- \frac{i}{(2\pi)^2}\, \widetilde{V}(k_1^\prime-k_1) \;\delta^3(k_1^\prime-k_1-k_2^\prime+k_2) \;\delta\left(\frac{{k_1^\prime}^2}{2m_1} + \frac{{k_2^\prime}^2}{2m_2} - \frac{{k_1}^2}{2m_1} - \frac{{k_2}^2}{2m_2}\right).
\end{align}
Interestingly, we see that the delta functions cause energy and momentum to be conserved.

\section{Reduced density operator}

We are interested in the situation where the second particle exits the system and becomes lost. We therefore consider the reduced density operator
\begin{align*}
\op{\rho}_\text{reduced}(t) &= \int d^3k_2 \bra{k_2} \op{\rho}(t) \ket{k_2} \nonumber\\
&= \op{\rho}_\text{reduced}\left(t_0\right) -  \frac{i}{\hbar} \int_{t_0}^tdt^\prime\int_{-\infty}^\infty d^3k_2\bra{k_2}\left[\op{H}_I(t^\prime),\; \op{\rho}\left(t_0\right)\right]\ket{k_2} + \Od{\op{H}_I(t)^2}.
\end{align*}
In the momentum basis the matrix elements are
\[
\bra{k_1^\prime}\op{\rho}_\text{reduced}(t)\ket{k_1} = \bra{k_1^\prime}\op{\rho}_\text{reduced}\left(t_0\right)\ket{k_1} -  \frac{i}{\hbar} \int_{t_0}^tdt^\prime\int_{-\infty}^\infty d^3k_2 \bra{k_1^\prime, k_2}\left[\op{H}_I(t^\prime),\; \op{\rho}\left(t_0\right)\right]\ket{k_1, k_2} + \Od{\op{H}_I(t)^2}.
\]
As before, we drop the higher order terms and take \(t \to \infty\) and \(t_0 \to -\infty\). Then
\begin{align*}
\hspace{-5em}\bra{{k_1}^\prime}\op{\rho}_\text{reduced}(\infty)\ket{k_1} =\hspace{1.7em}&\hspace{-1.5em} \bra{k_1^\prime}\op{\rho}_\text{reduced}\left(-\infty\right)\ket{k_1} -  \frac{i}{\hbar} \int_{-\infty}^\infty dt \int_{-\infty}^\infty d^3k_2 \bra{k_1^\prime, k_2}\left[\op{H}_I(t),\; \op{\rho}\left(-\infty\right)\right]\ket{k_1, k_2} 
\\=\hspace{1.7em}&\hspace{-1.5em}
\bra{k_1^\prime}\op{\rho}_\text{reduced}\left(-\infty\right)\ket{k_1}
\\&-  \frac{i}{\hbar} \int_{-\infty}^\infty d^3k_2 \,d^3\,\widetilde{k}_1\,d^3\,\widetilde{k}_2 \left\{\int_{-\infty}^\infty dt \bra{k_1^\prime, k_2}\op{H}_I(t) \ket{\widetilde{k}_1, \widetilde{k}_2}
\right\}\bra{\widetilde{k}_1, \widetilde{k}_2}\op{\rho}\left(-\infty\right)\ket{k_1, k_2}
\\&+  \frac{i}{\hbar} \int_{-\infty}^\infty d^3k_2\,d^3\,\widetilde{k}_1\,d^3\,\widetilde{k}_2 \bra{k_1^\prime, k_2} \op{\rho}\left(-\infty\right)
\ket{\widetilde{k}_1, \widetilde{k}_2}\left\{\int_{-\infty}^\infty dt\bra{\widetilde{k}_1, \widetilde{k}_2}\op{H}_I(t)\ket{k_1, k_2}\right\}.
%\\=\hspace{1.7em}&\hspace{-1.5em}
%\bra{k_1^\prime}\op{\rho}_\text{reduced}\left(-\infty\right)\ket{k_1} \\
%\\&-  \frac{i}{\hbar} \int_{-\infty}^\infty dt \int_{-\infty}^\infty d^3k_2 \,d^3\,\widetilde{k}_1\,d^3\,\widetilde{k}_2 \;{\Bigg(}\frac{1}{(2\pi)^3}\; e^{-\frac{i}{\hbar}(t-t_0)\left(\frac{{k_1^\prime}^2}{2m_1} + \frac{{k_2}^2}{2m_2} - \frac{{\widetilde{k}_1}^2}{2m_1} - \frac{{\widetilde{k}_2}^2}{2m_2}\right)} \widetilde{V}\left(k_1^\prime - \widetilde{k}_1\right)\,\\&\hspace{9em}\delta^3(k_1^\prime - \widetilde{k}_1 - k_2 + \widetilde{k}_2)    \bra{\widetilde{k}_1, \widetilde{k}_2}\op{\rho}\left(-\infty\right)\ket{k_1, k_2} {\Bigg)}
%\\&+  \frac{i}{\hbar} \int_{-\infty}^\infty dt \int_{-\infty}^\infty d^3k_2 \,d^3\,\widetilde{k}_1\,d^3\,\widetilde{k}_2 \;{\Bigg(}\frac{1}{(2\pi)^3}\; e^{-\frac{i}{\hbar}(t-t_0)\left(\frac{{\widetilde{k}_1}^2}{2m_1} + \frac{{\widetilde{k}_2}^2}{2m_2} - \frac{{{k}_1}^2}{2m_1} - \frac{{{k}_2}^2}{2m_2}\right)} \widetilde{V}\left(\widetilde{k}_1^\prime - {k}_1\right)\,\\&\hspace{9em}\delta^3(\widetilde{k}_1 - {k}_1 - \widetilde{k}_2 + {k}_2)    \bra{{k}_1^\prime, {k}_2}\op{\rho}\left(-\infty\right)\ket{\widetilde{k}_1, \widetilde{k}_2}{\Bigg)}
\end{align*}
Plugging in our result (\ref{eq.HIintegrated}),
\begin{align*}
\bra{{k_1}^\prime}\op{\rho}_\text{reduced}(\infty)\ket{k_1}
=\hspace{1.7em}&\hspace{-1.5em}
\bra{k_1^\prime}\op{\rho}_\text{reduced}\left(-\infty\right)\ket{k_1}
\\&-  i \int_{-\infty}^\infty d^3k_2 \,d^3\,\widetilde{k}_1\,d^3\,\widetilde{k}_2 \;\;{\Bigg(}\frac{1}{(2\pi)^2}\; \;\widetilde{V}\left(k_1^\prime - \widetilde{k}_1\right)\;\delta\left({\textstyle\frac{{k_1^\prime}^2}{2m_1} + \frac{{k_2}^2}{2m_2} - \frac{{\widetilde{k}_1}^2}{2m_1} - \frac{{\widetilde{k}_2}^2}{2m_2}}\right) \\&\hspace{9em}\delta^3(k_1^\prime - \widetilde{k}_1 - k_2 + \widetilde{k}_2)    \;\;\bra{\widetilde{k}_1, \widetilde{k}_2}\op{\rho}\left(-\infty\right)\ket{k_1, k_2}{\Bigg)}
\\&+  i \int_{-\infty}^\infty d^3k_2 \,d^3\,\widetilde{k}_1\,d^3\,\widetilde{k}_2 \;\;{\Bigg(}\frac{1}{(2\pi)^2}\; \;\widetilde{V}\left(\widetilde{k}_1 - {k}_1\right)\;\delta\left({\textstyle\frac{{\widetilde{k}_1}^2}{2m_1} + \frac{{\widetilde{k}_2}^2}{2m_2} - \frac{{{k}_1}^2}{2m_1} - \frac{{{k}_2}^2}{2m_2}}\right) \\&\hspace{9em}\delta^3(\widetilde{k}_1 - {k}_1 - \widetilde{k}_2 + {k}_2)    \;\;\bra{{k}_1^\prime, {k}_2}\op{\rho}\left(-\infty\right)\ket{\widetilde{k}_1, \widetilde{k}_2}{\Bigg)}.
\end{align*}
Finally, we use the delta functions to get rid of the \(\widetilde{k}_1\) integrals. This gives us
\begin{align*}
\hspace{-4em}\bra{{k_1}^\prime}\op{\rho}_\text{reduced}(\infty)\ket{k_1}
= 
\bra{k_1^\prime}&\op{\rho}_\text{reduced}\left(-\infty\right)\ket{k_1} \\
&-  i \int_{-\infty}^\infty d^3k_2 \,d^3\,\widetilde{k}_2 \;\;\frac{1}{(2\pi)^2}\; \;\widetilde{V}\left(k_2 - 
\widetilde{k}_2\right)\; \bra{k_1^\prime + \widetilde{k}_2 - k_2,\; \widetilde{k}_2}\op{\rho}\left(-\infty\right)\ket{k_1, k_2} \;D(k_1^\prime, k_2, \widetilde{k}_2)
\nonumber\\&+  i \int_{-\infty}^\infty d^3k_2 \,d^3\,\widetilde{k}_2 \;\;\frac{1}{(2\pi)^2}\; \;\widetilde{V}\left(\widetilde{k}_2 - 
{k}_2\right)\; \bra{{k}_1^\prime, {k}_2}\op{\rho}\left(-\infty\right)\ket{k_1 + \widetilde{k}_2 - k_2,\; \widetilde{k}_2} \;D(k_1, k_2, \widetilde{k}_2). \nonumber
\end{align*}
where
\begin{align*}
%\delta_1 &= \delta\left({\textstyle\frac{1}{2}\;{k_2}^2\left(\frac{1}{m_2}-\frac{1}{m_1}\right) - \frac{1}{2}\;{\widetilde{k}_2}^2\left(\frac{1}{m_1}+\frac{1}{m_2}\right) + \frac{1}{2m_1}\left(-k_1^\prime\cdot \widetilde{k}_2 + k_1^\prime \cdot k_2 + \widetilde{k}_2 \cdot k_2\right)}\right) \\
D(k_1, k_2, \widetilde{k}_2) &= \delta\left({\textstyle \frac{1}{2}\;{{k}_2}^2\left(\frac{1}{m_1}-\frac{1}{m_2}\right)+ \frac{1}{2}\;{\widetilde{k}_2}^2\left(\frac{1}{m_1}+\frac{1}{m_2}\right) - \frac{1}{2m_1}\left(-k_1\cdot \widetilde{k}_2 + k_1 \cdot k_2 + \widetilde{k}_2 \cdot k_2\right)}\right).
\end{align*}
Then, assuming \(\widetilde{V}\) is an even function,
\begin{align}\label{eq.reddensevolution}
\hspace{-4em}\bra{{k_1}^\prime}&\op{\rho}_\text{reduced}(\infty)\ket{k_1}
= {\Bigg (}
\bra{k_1^\prime}\op{\rho}_\text{reduced}\left(-\infty\right)\ket{k_1}\;\;+
 \\&\nonumber \int_{-\infty}^\infty \frac{d^3k_2 \,d^3\,\widetilde{k}_2}{(2\pi)^2}\;\widetilde{V}\left(k_2 - 
\widetilde{k}_2\right)\;\left(P\left(k_1,k_1^\prime k_2, \widetilde{k}_2\right)D\left(k_1, k_2, \widetilde{k}_2\right) + P^*\left(k_1^\prime,k_1, k_2, \widetilde{k}_2\right)D\left(k_1^\prime, k_2, \widetilde{k}_2\right)\right){\Bigg )}
\end{align}
where
\[
P(k_1, k_1^\prime, k_2, \widetilde{k}_2) = i\bra{k_1^\prime, k_2} \op{\rho}\left(-\infty\right) \ket{k_1 + \widetilde{k}_2 - k_2,\; \widetilde{k}_2}.
\]



\end{document}