\documentclass[11pt]{article}
\usepackage{amsmath,amssymb,amsthm,epsf, graphics,enumerate,fancyhdr,marvosym,cancel}
\usepackage[letterpaper, margin=0.9in]{geometry}

% ordered lists have letters instead of numbers
\renewcommand\theenumi{\alph{enumi}}
\renewcommand\labelenumi{(\theenumi)}

% subsections have letters instead of numbers
%\renewcommand{\thesubsection}{\thesection (\alph{subsection})}

% use hyperref for links but don't draw the ugly boxes around links
\usepackage[hidelinks]{hyperref}

% add \subsubsubsection command
% taken from https://tex.stackexchange.com/questions/60209/how-to-add-an-extra-level-of-sections-with-headings-below-subsubsection
\usepackage{titlesec}
\titleclass{\subsubsubsection}{straight}[\subsection]
\newcounter{subsubsubsection}[subsubsection]
\renewcommand\thesubsubsubsection{\thesubsubsection.\arabic{subsubsubsection}}
\renewcommand\theparagraph{\thesubsubsubsection.\arabic{paragraph}} % optional; useful if paragraphs are to be numbered
\titleformat{\subsubsubsection}
  {\normalfont\normalsize\bfseries}{\thesubsubsubsection}{1em}{}
\titlespacing*{\subsubsubsection}
{0pt}{3.25ex plus 1ex minus .2ex}{1.5ex plus .2ex}
\makeatletter
\renewcommand\paragraph{\@startsection{paragraph}{5}{\z@}%
  {3.25ex \@plus1ex \@minus.2ex}%
  {-1em}%
  {\normalfont\normalsize\bfseries}}
\renewcommand\subparagraph{\@startsection{subparagraph}{6}{\parindent}%
  {3.25ex \@plus1ex \@minus .2ex}%
  {-1em}%
  {\normalfont\normalsize\bfseries}}
\def\toclevel@subsubsubsection{4}
\def\toclevel@paragraph{5}
\def\toclevel@paragraph{6}
\def\l@subsubsubsection{\@dottedtocline{4}{7em}{4em}}
\def\l@paragraph{\@dottedtocline{5}{10em}{5em}}
\def\l@subparagraph{\@dottedtocline{6}{14em}{6em}}
\makeatother
\setcounter{secnumdepth}{4}
\setcounter{tocdepth}{4}

% command to create underlined link to outside websites
\newcommand{\hrefunderline}[2]{\underline{\href{#1}{#2}}}

% symbols for common sets
\newcommand{\ZZ}{\mathbb{Z}}
\newcommand{\NN}{\mathbb{N}}
\newcommand{\FF}{\mathbb{F}}
\newcommand{\RR}{\mathbb{R}}
\newcommand{\QQ}{\mathbb{Q}}
\newcommand{\CC}{\mathbb{C}}

\renewcommand{\Im}{\operatorname{Im}}
\renewcommand{\Re}{\operatorname{Re}}

% use boldface instead of arrows for vectors
%\renewcommand{\vec}{\mathbf}

% order
\newcommand{\Od}[1]{\mathcal{O}{\left(#1\right)}}

% bras and kets
\newcommand{\bra}[1]{\left\langle#1\right|}
\newcommand{\ket}[1]{\left|#1\right\rangle}
\newcommand{\braket}[2]{\left\langle#1|#2\right\rangle}

% the Hilbert space
\newcommand{\hilb}{\mathcal{H}}

% quantum operators
\newcommand{\op}[1]{\hat{#1}}

% main basis used for bipartite system
\newcommand{\bket}[2]{\ket{#1\;#2}}
\newcommand{\bbra}[2]{\bra{#1\;#2}}

% for congruences
\newcommand{\mmod}[1]{\;(\operatorname{mod} {#1})}

% new theorem style with boldface title and italic content
\newtheoremstyle{step}%            % Name
  {}%                                     % Space above
  {}%                                     % Space below
  {\itshape}%                           % Body font
  {}%                                     % Indent amount
  {\itshape}%                          % Theorem head font
  {:}%                                    % Punctuation after theorem head
  { }%                                    % Space after theorem head, ' ', or \newline
  {}%                                     % Theorem head spec (can be left empty, meaning `normal')

% new theorem style for a gap in the notes
\newtheoremstyle{gap}%            % Name
  {}%                                     % Space above
  {}%                                     % Space below
  {\itshape}%                           % Body font
  {}%                                     % Indent amount
  {\itshape}%                          % Theorem head font
  {!!!}%                                    % Punctuation after theorem head
  {    }%                                    % Space after theorem head, ' ', or \newline
  {}%                                     % Theorem head spec (can be left empty, meaning `normal')

\theoremstyle{theorem}
\newtheorem{claim}{Claim}[section]
\newtheorem*{claim*}{Claim}
\newtheorem{lemma}[claim]{Lemma}
\newtheorem*{lemma*}{Lemma}
\newtheorem{fact}{Fact of Nature}

\theoremstyle{remark}
\newtheorem*{remark}{Remark}
\newtheorem*{notation}{Notation}

\theoremstyle{step}
\newtheorem{step}{Step}[subsection]
\renewcommand{\thestep}{\arabic{step}}

\theoremstyle{gap}
\newtheorem*{gap}{Gap}

\begin{document}


\title{Notes on Entropy Change Over Time}
\author{Duncan MacIntyre}
\date{\today}
\maketitle
\tableofcontents
\bigskip
\newpage

\section{Time-dependent perturbation theory}

Suppose we have a Hamiltonian of the form \[\op{H} = \op{H}_0 + \lambda \op{V}(t)\]
where \(\op{H}_0\) is a well-understood Hamiltonian that does not depend on time and \(\lambda \op{V}(t)\) is ``small''. For example, \(\op{H}_0\) might be the free particle Hamiltonian \(\op{H}_0 = \frac{m}{2} \nabla^2\). Our equation of motion is the Schr\"odinger equation
\[i \hbar \frac{\partial}{\partial t} \ket{\Psi, t} = \left(\op{H}_0 + \lambda\op{V}(t)\right) \ket{\Psi, t}\]
where \(\ket{\Psi, t}\) is the usual Schr\"odinger-picture state at time \(t\).

Let \(\op{U}(t_0, t) = e^{-i\op{H}_0(t-t_0)/\hbar}\). Then \(\op{U}(t_0, t)\) is the operator that evolves a state from time \(t_0\) to time \(t\) according to \(\op{H}_0\). We define the interaction-picture state to be
\[\ket{\Psi_I, t} = \op{U}(t_0, t)^\dagger \ket{\Psi, t}\]
so \(\ket{\Psi, t} = \op{U}(t_0, t) \ket{\Psi_I, t}\).
Plugging this in to the Schr\"odinger equation,
\begin{align*}
i \hbar \frac{\partial}{\partial t} \op{U}(t_0, t)\ket{\Psi_I, t} &= \left(\op{H}_0 + \lambda \op{V}(t)\right) \op{U}(t_0, t)\ket{\Psi, t} \\
i \hbar \left[\frac{\partial}{\partial t} e^{-i\op{H}_0(t-t_0)/\hbar}\right] \ket{\Psi_I, t} + i \hbar e^{-i\op{H}_0(t-t_0)/\hbar} \frac{\partial}{\partial t} \ket{\Psi_I, t} &= \op{H}_0 e^{-i\op{H}_0(t-t_0)/\hbar}\ket{\Psi_I, t} + \lambda\op{V}(t) e^{-i\op{H}_0(t-t_0)/\hbar}\ket{\Psi_I, t} \\
\cancel{-i^2 \op{H}_0 e^{i\op{H}_0(t-t_0)/\hbar} \ket{\Psi_I, t}} + i \hbar e^{-i\op{H}_0(t-t_0)/\hbar} \frac{\partial}{\partial t} \ket{\Psi_I, t} &= \cancel{\op{H}_0 e^{-i\op{H}_0(t-t_0)/\hbar}\ket{\Psi_I, t}} + \lambda\op{V}(t) e^{-i\op{H}_0(t-t_0)/\hbar}\ket{\Psi_I, t} \\
 i \hbar \frac{\partial}{\partial t} \ket{\Psi_I, t} &= e^{i\op{H}_0(t-t_0)/\hbar}\lambda\op{V}(t) e^{-i\op{H}_0(t-t_0)/\hbar}\ket{\Psi_I, t} \\
 i \hbar \frac{\partial}{\partial t} \ket{\Psi_I, t} &= \lambda\op{H}_I(t)\ket{\Psi_I, t}
\end{align*}
where we define the interaction Hamiltonian to be
\[\op{H}_I(t) = \op{U}(t_0, t)^\dagger \op{V}(t)\op{U}(t_0, t) = e^{i\op{H}_0(t-t_0)/\hbar}\op{V}(t) e^{-i\op{H}_0(t-t_0)/\hbar}.\]

This sets up the interaction picture. We have rephrased our problem so that we can continue with quantum mechanics normally without having to worry about the time evolution due to \(\op{H}_0\).

We now integrate both sides of our expression.
\begin{align}
\int_{t_0}^t \frac{\partial}{\partial t^\prime} \ket{\Psi_I, t^\prime} &= -\frac{i}{\hbar} \lambda \int_{t_0}^t \op{H}_I(t^\prime)\ket{\Psi_I, t^\prime} \,dt^\prime \nonumber\\
\ket{\Psi_I, t} - \ket{\Psi_I, t_0} &= -\frac{i}{\hbar} \lambda \int_{t_0}^t \op{H}_I(t^\prime)\ket{\Psi_I, t^\prime} \,dt^\prime \nonumber\\
\ket{\Psi_I, t} &= \ket{\Psi_I, t_0} - \frac{i}{\hbar} \lambda \int_{t_0}^t \op{H}_I(t^\prime)\ket{\Psi_I, t^\prime} \,dt^\prime \label{eq.intse}
\end{align}
This is called the integral form of the Schr\"odinger equation.

We can now iteratively calculate perturbative approximations where we assume \(\op{H}_I(t)\) is small. The zero order approximation is simply
\[\ket{\Psi_I, t} = \ket{\Psi_I, t_0} + \Od{\lambda}.\]
Plugging this in for the state inside the integral in (\ref{eq.intse}), we get the first order approximation
\begin{equation}\label{eq.firsttdpe}
\ket{\Psi_I, t} = \ket{\Psi_I, t_0} - \frac{i}{\hbar} \lambda \int_{t_0}^t \op{H}_I(t^\prime)\ket{\Psi_I, t_0} \,dt^\prime + \Od{\lambda^2}.
\end{equation}
Plugging this in to (\ref{eq.intse}) again we get the second order approximation
\begin{align}\nonumber
\ket{\Psi_I, t} &= \ket{\Psi_I, t_0} - \frac{i}{\hbar} \lambda \int_{t_0}^t \op{H}_I(t^\prime)\left[\ket{\Psi_I, t_0} - \frac{i}{\hbar} \lambda \int_{t_0}^{t^\prime} \op{H}_I(t^{\prime\prime})\ket{\Psi_I, t_0} \,dt^{\prime\prime}\right] \,dt^\prime + \Od{\lambda^3} \\
&= \ket{\Psi_I, t_0} - \frac{i}{\hbar}\lambda \int_{t_0}^t \op{H}_I (t^\prime) \ket{\Psi_I, t_0}\,dt^\prime - \frac{\lambda^2}{\hbar^2} \int_{t_0}^t \int_{t_0}^{t^\prime} \op{H}_I(t^\prime) \op{H}_I(t^{\prime\prime})\ket{\Psi_I, t_0} \,dt^{\prime\prime}\,dt^\prime + \Od{\lambda^3}.
\label{eq.secondtdpe}\end{align}
In general, we can keep going to achieve higher order approximations. For us, however, the second order approximation (\ref{eq.secondtdpe}) is enough.


\section{Time evolution of the density operator}

We now consider mixed states. We will need to describe the system by the density operator \(\op{\rho}(t)\). Let's derive the time evolution of \(\op{\rho}(t)\) in second order perturbation theory based on (\ref{eq.secondtdpe}).

Suppose at time \(t_0\) we have a statistical ensemble of interaction-picture states \(\ket{{\psi_I}_n, t_0}\) each with probability \(P_n\). Then
\[\op{\rho}(t_0) = \sum_n P_n \ket{{\Psi_I}_n, t_0}\bra{{\Psi_I}_n, t_0}.\]
At time \(t\) states will have evolved according to (\ref{eq.secondtdpe}), so
\[\op{\rho}(t) = \sum_n P_n \ket{{\Psi_I}_n, t}\bra{{\Psi_I}_n, t}.\]
From (\ref{eq.secondtdpe}) we have
\begin{align*}
\ket{{\Psi_I}_n, t} &= \ket{\Psi_I, t_0} -\frac{i}{\hbar}\lambda\int_{t_0}^t \op{H}_I (t^\prime) \ket{\Psi_I, t_0}\,dt^\prime - \frac{\lambda^2}{\hbar^2} \int_{t_0}^t \int_{t_0}^{t^\prime} \op{H}_I(t^\prime) \op{H}_I(t^{\prime\prime})\ket{\Psi_I, t_0} \,dt^{\prime\prime}\,dt^\prime + \Od{\lambda^3} \\
\bra{{\Psi_I}_n, t} &= \bra{\Psi_I, t_0} -\frac{i}{\hbar}\lambda\int_{t_0}^t \bra{\Psi_I, t_0}{\op{H}_I} (t^\prime)\,dt^\prime - \frac{\lambda^2}{\hbar^2} \int_{t_0}^t \int_{t_0}^{t^\prime} \bra{\Psi_I, t_0} {\op{H}_I}(t^{\prime\prime}){\op{H}_I}(t^\prime)\,dt^{\prime\prime}\,dt^\prime + \Od{\lambda^3}
\end{align*}
so
\begin{align*}
\ket{{\Psi_I}_n, t} \bra{{\Psi_I}_n, t} \;\;=\;\;& \ket{{\Psi_I}_n, t_0} \bra{{\Psi_I}_n, t_0} 
\;\;-\;\;\frac{i}{\hbar}\lambda \int_{t_0}^t \left(\op{H}(t^\prime)\ket{{\Psi_I}_n, t_0}\bra{{\Psi_I}_n, t_0} - \ket{{\Psi_I}_n, t_0}\bra{{\Psi_I}_n, t_0} \op{H}_I (t^\prime)\right) \,dt^\prime \\
& -\;\; \frac{\lambda^2}{\hbar^2} \int_{t_0}^t \int_{t_0}^{t^\prime} \left( \op{H}_I (t^\prime) \op{H}_I(t^{\prime\prime}) \ket{{\Psi_I}_n, t_0}\bra{{\Psi_I}_n, t_0} +\;\; \ket{{\Psi_I}_n, t_0}\bra{{\Psi_I}_n, t_0} \op{H}_I(t^{\prime\prime}) \op{H}_I(t^\prime)\right)\,dt^{\prime\prime}\,dt^\prime \\
&+\;\; \frac{\lambda^2}{\hbar^2} \int_{t_0}^t \int_{t_0}^{t} \op{H}_I (t^\prime) \ket{{\Psi_I}_n, t_0}\bra{{\Psi_I}_n, t_0} \op{H}_I (t^{\prime\prime})\,dt^{\prime\prime}\,dt^\prime  \;\;+\;\; \Od{\lambda^3}.
\end{align*}
Then
\begin{align}
\label{eq.rhoevol}
\op{\rho}(t) \;\;=\;\;& \op{\rho}\left(t_0\right)
\;\;-\;\;\frac{i}{\hbar}\lambda \int_{t_0}^t \left[\op{H}(t^\prime),\; \op{\rho}(t_0) \right] \,dt^\prime \\
& -\;\; \frac{\lambda^2}{\hbar^2} \int_{t_0}^t \int_{t_0}^{t^\prime} \left( \op{H}_I (t^\prime) \op{H}_I(t^{\prime\prime}) \;\op{\rho}(t_0) + \op{\rho}(t_0)\; \op{H}_I (t^{\prime\prime}) \op{H}_I(t^{\prime}) \right)\,dt^{\prime\prime}\,dt^\prime \nonumber\\
&+\;\; \frac{\lambda^2}{\hbar^2} \int_{t_0}^t \int_{t_0}^{t} \op{H}_I (t^\prime) \op{\rho}(t_0) \op{H}_I (t^{\prime\prime})\,dt^{\prime\prime}\,dt^\prime \;\;+\;\; \Od{\lambda^3}\nonumber.
\end{align}

\section{Eigenvalue corrections from perturbation theory}

The goal of this section is to compute the second-order perturbative corrections to the eigenvalues of an operator. We use the approach known as time-independent perturbation theory.

Suppose \(\op{\rho}\) is an operator with eigenstates \(\ket{\Psi_n}\) and eigenvalues \(\sigma_n\). Suppose we have the asymptotic expansions
\begin{align*}
\op{\rho} &= \op{\rho}_0 + \lambda \op{\rho}_1 + \lambda^2 \op{\rho}_2 + \Od{\lambda^3} \\
\ket{\Psi_n} &= \ket{\Psi_n^{(0)}} + \lambda \ket{\Psi_n^{(1)}} + \lambda^2 \ket{\Psi_n^{(2)}} + \Od{\lambda^3} \\
\sigma_n &= \sigma_n^{(0)} + \lambda \sigma_n^{(1)} + \lambda^2 \sigma_n^{(2)} + \Od{\lambda^3}
\end{align*}
where \(\left\{\ket{\Psi_n^{(0)}}\right\}\) is an orthonormal basis for the Hilbert space.

We have \(\op{\rho} \ket{\Psi_n} = \sigma_n \ket{\Psi_n}\). In the zeroth order of \(\lambda\) this gives \(\op{\rho}_0 \ket{\Psi_n^{(0)}} = \sigma_n^{(0)} \ket{\Psi_n^{(0)}}\), that is, \(\ket{\Psi_n^{(0)}}\) is an eigenstate of \(\op{\rho}_0\) with eigenvalue \(\sigma_n^{(0)}\).

Now, we have
\begin{align*}
\sigma_n \ket{\Psi_n} &= \op{\rho}\ket{\Psi_n}\\
\left(\sigma_n^{(0)} - \sigma_m^{(0)} + \lambda \sigma_n^{(1)} + \lambda^2 \sigma_n^{(2)} + \Od{\lambda^3} \right)\braket{\Psi_m^{(0)}}{\Psi_n} &=\bra{\Psi_m^{(0)}} \lambda \op{\rho}_1 + \lambda^2 \op{\rho}_2 + \Od{\lambda^3} \ket{\Psi_n} \\
\braket{\Psi_m^{(0)}}{\Psi_n} &=\frac{\bra{\Psi_m^{(0)}} \lambda \op{\rho}_1 + \lambda^2 \op{\rho}_2 + \Od{\lambda^3} \ket{\Psi_n}}{\sigma_n^{(0)} - \sigma_m^{(0)} + \lambda \sigma_n^{(1)} + \lambda^2 \sigma_n^{(2)} + \Od{\lambda^3}}.
\end{align*}
We will next proceed to get rid of the denominator by using the Taylor expansion \(\frac{1}{1+x} = 1-x +\Od{x^2}\).

Let \(D_n\) be the indices degenerate with \(n\). That is, let \(D_n = \{m : \sigma_m^{(0)} = \sigma_n^{(0)}\}\).
Consider the case where \(m \in D_n\). Let \(\delta\) be a small real number. In the limit as \(\delta \to 0\),
\begin{align*}
\braket{\Psi_m^{(0)}}{\Psi_n} &=\frac{\bra{\Psi_m^{(0)}} \lambda \op{\rho}_1 + \lambda^2 \op{\rho}_2 + \Od{\lambda^3} \ket{\Psi_n}}{\lambda \sigma_n^{(1)} + \lambda^2 \sigma_n^{(2)} + \Od{\lambda^3} + \lambda\delta} \\
&=\frac{\bra{\Psi_m^{(0)}} \lambda \op{\rho}_1 + \lambda^2 \op{\rho}_2 + \Od{\lambda^3} \ket{\Psi_n}}{\lambda \left(\sigma_n^{(1)}+ \delta\right)\left(1 + \lambda \frac{\sigma_n^{(2)}}{\sigma_n^{(1)}+\delta} + \Od{\lambda^2}\right) } \\
&=\frac{\bra{\Psi_m^{(0)}} \op{\rho}_1 + \lambda \op{\rho}_2 + \Od{\lambda^2} \ket{\Psi_n}}{\sigma_n^{(1)}+ \delta}\left(1 - \lambda \frac{\sigma_n^{(2)}}{\sigma_n^{(1)}+\delta} + \Od{\lambda^2}\right) \\
&= \frac{\bra{\Psi_m^{(0)}} \left(\op{\rho}_1 + \lambda \op{\rho}_2 + \Od{\lambda^2} \right)\left(\ket{\Psi_n^{(0)}} + \lambda \ket{\Psi_n^{(1)}} + \Od{\lambda^2}\right)}{\sigma_n^{(1)}+\delta}
\left(1 - \lambda \frac{\sigma_n^{(2)}}{\sigma_n^{(1)}+\delta} + \Od{\lambda^2}\right) \\
&=\frac{\bra{\Psi_m^{(0)}} \op{\rho}_1 \ket{\Psi_n^{(0)}}}{\sigma_n^{(1)} + \delta} + \lambda \left[\frac{\bra{\Psi_m^{(0)}}\op{\rho}_2\ket{\Psi_n^{(0)}}}{\sigma_n^{(1)} + \delta} + \frac{\bra{\Psi_m^{(0)}}\op{\rho}_2\ket{\Psi_n^{(1)}}}{\sigma_n^{(1)} + \delta} - \frac{\sigma_n^{(2)}\bra{\Psi_m^{(0)}}\op{\rho}_2\ket{\Psi_n^{(0)}}}{\left(\sigma_n^{(1)} + \delta\right)^2}\right] + \Od{\lambda^2}
\end{align*}
(The \(\delta\) ensured that we did not divide by zero if \(\sigma_n^{(1)}=0\).) Now consider the case where \(m \not\in D_n\). Then
\begin{align*}
\braket{\Psi_m^{(0)}}{\Psi_n} &=\frac{\bra{\Psi_m^{(0)}} \lambda \op{\rho}_1 + \Od{\lambda^2} \ket{\Psi_n}}{\left(\sigma_n^{(0)} - \sigma_m^{(0)}\right)\left(1+\Od{\lambda}\right)} \\
&=\frac{\bra{\Psi_m^{(0)}} \lambda \op{\rho}_1 + \Od{\lambda^2} \ket{\Psi_n}}{\sigma_n^{(0)} - \sigma_m^{(0)}}\left(1 + \Od{\lambda}\right) \\
&= \lambda\frac{\bra{\Psi_m^{(0)}} \op{\rho}_1 \ket{\Psi_n}}{\sigma_n^{(0)} - \sigma_m^{(0)}} + \Od{\lambda^2}.
\end{align*}
Putting this all together,
\begin{align}\label{eq.tiptmaineq}
\ket{\Psi_n} &= \sum_{m \in D_n} \ket{\Psi_m^{(0)}} \braket{\Psi_m^{(0)}}{\Psi_n} + \sum_{m \not\in D_n} \ket{\Psi_m^{(0)}} \braket{\Psi_m^{(0)}}{\Psi_n} \\
&= \sum_{m \in D_n} \ket{\Psi_m^{(0)}} \lim_{\delta\to 0} \frac{\bra{\Psi_m^{(0)}} \op{\rho}_1 \ket{\Psi_n^{(0)}}}{\sigma_n^{(1)} + \delta}
\nonumber\\&\hspace{1.5em}+ \lambda\;\;
\Biggl(\;\sum_{m \in D_n} \ket{\Psi_m^{(0)}} \lim_{\delta \to 0}\left[\frac{\bra{\Psi_m^{(0)}}\op{\rho}_2\ket{\Psi_n^{(0)}}}{\sigma_n^{(1)} + \delta} + \frac{\bra{\Psi_m^{(0)}}\op{\rho}_1\ket{\Psi_n^{(1)}}}{\sigma_n^{(1)} + \delta} - \frac{\sigma_n^{(2)}\bra{\Psi_m^{(0)}}\op{\rho}_0\ket{\Psi_n^{(0)}}}{\left(\sigma_n^{(1)} + \delta\right)^2}\right] 
\nonumber\\&\hspace{6em}+ \sum_{m \not\in D_n} \ket{\Psi_m^{(0)}}\frac{\bra{\Psi_m^{(0)}} \op{\rho}_1 \ket{\Psi_n}}{\sigma_n^{(0)} - \sigma_m^{(0)}}
\;\Biggr)
\nonumber\\&\hspace{1.5em}+ \Od{\lambda^2}\nonumber
\end{align}

In the zeroth order of \(\lambda\), equation (\ref{eq.tiptmaineq}) gives
\[
\ket{\Psi_n^{(0)}} = \sum_{m \in D_n} \ket{\Psi_m^{(0)}} \lim_{\delta\to0}\frac{\bra{\Psi_m^{(0)}} \op{\rho}_1 \ket{\Psi_n^{(0)}}}{\sigma_n^{(1)} + \delta}.
\]
Multiplying by \(\bra{\Psi_n^{(0)}}\) we get
\begin{equation}\label{eq.tipt.foval}
\boxed{\sigma_n^{(1)} = \bra{\Psi_n^{(0)}}\op{\rho}_1\ket{\Psi_n^{(0)}}.}
\end{equation}
If we instead multiply by \(\bra{\Psi_k^{(0)}}\) where \(k \in D_n\) but \(k \neq n\) we get
\begin{equation}\label{eq.tipt.fodiag}
\boxed{0 = \bra{\Psi_k^{(0)}}\op{\rho}_1\ket{\Psi_n^{(0)}}, \hspace{3em} k \in D_n,\hspace{3em} m \neq n.}
\end{equation}
In other words, the matrix given by \(\bra{\Psi_k^{(0)}}\op{\rho}_1\ket{\Psi_j^{(0)}}\) is diagonal on \(k,j\in D_n\), for any \(n\).

In the first order of \(\lambda\), equation (\ref{eq.tiptmaineq}) says
\[
\hspace{-2.5em}
\ket{\Psi_n^{(1)}} = \sum_{m \in D_n} \ket{\Psi_m^{(0)}} \lim_{\delta \to 0}\left[\frac{\bra{\Psi_m^{(0)}}\op{\rho}_2\ket{\Psi_n^{(0)}}}{\sigma_n^{(1)} + \delta} + \frac{\bra{\Psi_m^{(0)}}\op{\rho}_1\ket{\Psi_n^{(1)}}}{\sigma_n^{(1)} + \delta} - \frac{\sigma_n^{(2)}\bra{\Psi_m^{(0)}}\op{\rho}_0\ket{\Psi_n^{(0)}}}{\left(\sigma_n^{(1)} + \delta\right)^2}\right] 
+ \sum_{m \not\in D_n} \ket{\Psi_m^{(0)}}\frac{\bra{\Psi_m^{(0)}} \op{\rho}_1 \ket{\Psi_n}}{\sigma_n^{(0)} - \sigma_m^{(0)}}.
\]
Multiplying by \(\bra{\Psi_k^{(0)}}\) where \(k \not\in D_n\) gives
\[
\braket{\Psi_k^{(0)}}{\Psi_n^{(1)}} = \frac{\bra{\Psi_k^{(0)}} \op{\rho}_1 \ket{\Psi_n^{(0)}}}{\sigma_n^{(0)}-\sigma_k^{(0)}}.
\]
If we instead multiply by \(\bra{\Psi_n^{(0)}}\) we get
\[
0 = \braket{\Psi_n^{(0)}}{\Psi_n^{(1)}} = \lim_{\delta \to 0} \left[\frac{\bra{\Psi_n^{(0)}}\op{\rho}_2\ket{\Psi_n^{(0)}}}{\sigma_n^{(1)} + \delta} + \frac{\bra{\Psi_n^{(0)}}\op{\rho}_1\ket{\Psi_n^{(1)}}}{\sigma_n^{(1)} + \delta} - \frac{\sigma_n^{(2)}\left(\sigma_n^{(1)}+\delta\right)}{\left(\sigma_n^{(1)} + \delta\right)^2}\right]
\]
so
\begin{align*}
\hspace{-2em}\sigma_n^{(2)} &= \bra{\Psi_n^{(0)}}\op{\rho}_2\ket{\Psi_n^{(0)}} + \bra{\Psi_n^{(0)}}\op{\rho}_1\ket{\Psi_n^{(1)}} - \braket{\Psi_n^{(0)}}{\Psi_n^{(1)}} \sigma_n^{(1)}\\
&= \bra{\Psi_n^{(0)}}\op{\rho}_2\ket{\Psi_n^{(0)}} + \sum_{m \in D_n} \bra{\Psi_n^{(0)}}\op{\rho}_1\ket{\Psi_m^{(0)}}\braket{\Psi_m^{(0)}}{\Psi_n^{(1)}} + \sum_{m \not\in D_n} \bra{\Psi_n^{(0)}}\op{\rho}_1\ket{\Psi_m^{(0)}}\braket{\Psi_m^{(0)}}{\Psi_n^{(1)}} - \braket{\Psi_n^{(0)}}{\Psi_n^{(1)}} \sigma_n^{(1)}\\
&= \bra{\Psi_n^{(0)}}\op{\rho}_2\ket{\Psi_n^{(0)}} + \sigma_n^{(1)}\braket{\Psi_n^{(0)}}{\Psi_n^{(1)}} +  \sum_{m \not\in D_n} \bra{\Psi_n^{(0)}}\op{\rho}_1\ket{\Psi_m^{(0)}}\frac{\bra{\Psi_m^{(0)}} \op{\rho}_1 \ket{\Psi_n^{(0)}}}{\sigma_n^{(0)}-\sigma_m^{(0)}}- \braket{\Psi_n^{(0)}}{\Psi_n^{(1)}} \sigma_n^{(1)}
\end{align*}
hence
\begin{equation}\label{eq.tipt.soval}
\boxed{\sigma_n^{(2)} = \bra{\Psi_n^{(0)}}\op{\rho}_2\ket{\Psi_n^{(0)}} +  \sum_{m \not\in D_n} \frac{\left|\bra{\Psi_m^{(0)}} \op{\rho}_1 \ket{\Psi_n^{(0)}}\right|^2}{\sigma_n^{(0)}-\sigma_m^{(0)}}.}
\end{equation}


\section{The change in Von Neumann entropy}

Suppose we have a Hilbert space of the form \(\hilb = \hilb_A \otimes \hilb_B\). The \(\hilb_A\)-reduced density operator of a density operator \(\op{\rho}\) is defined to be
\[\overline{\op{\rho}} = \sum_m \left(\cdot\otimes\bra{\beta_m}\right) \op{\rho} \left(\cdot\otimes\ket{\beta_m}\right)\]
where \(\{\ket{\beta_m}\}\) is any orthonormal basis for \(\hilb_B\). (One can verify that all choices of basis give the same \(\overline{\op{\rho}}\).)
We can consider \(\overline{\op{\rho}}\) to act on the Hilbert space \(\hilb_A\).

Define the \(\hilb_A\)-subsystem entropy to be \(S = -\operatorname{Tr} \overline{\op{\rho}} \log \overline{\op{\rho}}\). If \(\overline{\op{\rho}}\) is diagonalized like
\begin{equation}\label{eq.redrhodef}
\overline{\op{\rho}} = \sum_n \overline{\sigma_n} \ket{\alpha_n}\bra{\alpha_n},
\end{equation}
where \(\{\ket{\alpha_n}\}\) is an orthonormal basis of \(\hilb_A\), then
\[S = -\sum_n \overline{\sigma_n} \log \overline{\sigma_n}.\]

Now, looking at equation (\ref{eq.rhoevol}), we see that at some fixed time \(t\) we can form asymptotic expansions
\begin{align*}
\overline{\op{\rho}}(t) &= \overline{\op{\rho}}_0 + \lambda \overline{\op{\rho}}_1 + \lambda^2 \overline{\op{\rho}}_2 + \Od{\lambda^3} \\
\ket{\alpha_n} &= \ket{\alpha_n^{(0)}} + \lambda \ket{\alpha_n^{(1)}} + \lambda^2 \ket{\alpha_n^{(2)}} + \Od{\lambda^3} \\
\overline{\sigma_n} &= \overline{\sigma_n^{(0)}} + \lambda \overline{\sigma_n^{(1)}} + \lambda^2 \overline{\sigma_n^{(2)}} + \Od{\lambda^3}
\end{align*}
by taking
\begin{align}
\nonumber\op{\rho}_0 &= \op{\rho}\left(t_0\right) \\
\nonumber\overline{\op{\rho}}_0 &= \overline{\op{\rho}}\left(t_0\right) =  \sum_m \left(\cdot\otimes\bra{\beta_m}\right) \op{\rho}_0 \left(\cdot\otimes\ket{\beta_m}\right) \\
\label{eq.rho1red}\overline{\op{\rho}}_1 &= -\frac{i}{\hbar} \int_{t_0}^t \sum_m \left(\cdot\otimes\bra{\beta_m}\right) \left[\op{H}_I(t^\prime),\; \op{\rho}_0 \right]\left(\cdot\otimes\ket{\beta_m}\right) \,dt^\prime\\
\label{eq.rho2red}\overline{\op{\rho}}_2 &= - \frac{1}{\hbar^2} \int_{t_0}^t \int_{t_0}^{t^\prime} \sum_m \left(\cdot\otimes\bra{\beta_m}\right) \left( \op{H}_I (t^\prime) \op{H}_I(t^{\prime\prime}) \;\op{\rho}_0 + \op{\rho}_0\; \op{H}_I (t^{\prime\prime}) \op{H}_I(t^{\prime}) \right)\left(\cdot\otimes\ket{\beta_m}\right)\,dt^{\prime\prime}\,dt^\prime \\
\nonumber&\hspace{2em}+ \frac{1}{\hbar^2} \int_{t_0}^t \int_{t_0}^{t} \sum_m \left(\cdot\otimes\bra{\beta_m}\right) \op{H}_I (t^\prime) \,\op{\rho}_0\, \op{H}_I (t^{\prime\prime})\left(\cdot\otimes\ket{\beta_m}\right)\,dt^{\prime\prime}\,dt^\prime.
\end{align}
From (\ref{eq.redrhodef}) we see that
\begin{equation}\label{eq.rho0red}
\overline{\op{\rho}}_0 = \sum_n \overline{\sigma_{n}^{(0)}} \ket{\alpha_n^{(0)}}\bra{\alpha_n^{(0)}}.
\end{equation}
Therefore we can take \(\left\{\ket{\alpha_n^{(0)}}\right\}\) to be an orthonormal basis for \(\hilb_A\) and we must have \(\sum_n \overline{\sigma_n^{(0)}} = 1\).

It will be useful to work in the basis \(\left\{\ket{\alpha_n^{(0)}}\otimes\ket{\beta_m}\right\}\) of \(\hilb\). To simplify notation, let
\[
\bket{n}{m} = \ket{\alpha_n^{(0)}}\otimes\ket{\beta_m}.
\]
Let's write \(\op{\rho}_0\) in this basis. Take
\[\tau_{n^\prime, m^\prime, \tilde{n}, \tilde{m}} = \bbra{n^\prime}{m^\prime} \op{\rho}_0 \bket{\tilde{n}}{\tilde{m}}\]
so that
\begin{equation}\label{eq.rho0.withtau}
\op{\rho}_0 = \sum_{n^\prime, m^\prime, \tilde{n}, \tilde{m}} \tau_{n^\prime, m^\prime, \tilde{n}, \tilde{m}} \bket{n^\prime}{m^\prime} \bbra{\tilde{n}}{\tilde{m}}.
\end{equation}
Then
\[
\overline{\op{\rho}}_0 = \sum_m \bra{\beta_m} \op{\rho}_0 \ket{\beta_m} = \sum_{n^\prime, \tilde{n}, m} \tau_{n^\prime, m, \tilde{n}, m} \ket{\alpha_{n^\prime}^{(0)}} \bra{\alpha_{\tilde{n}}^{(0)}}.
\]
This should be diagonal with respect to \(n^\prime\) and \(\tilde{n}\); the off-diagonal terms must be zero. Thus
\begin{align*}
\overline{\op{\rho}}_0 &= \sum_{n, m} \tau_{n, m, n, m} \ket{\alpha_{n}^{(0)}} \bra{\alpha_{n}^{(0)}} &
&\text{and} &
\overline{\sigma_n^{(0)}} &= \sum_{m} \tau_{n, m, n, m}.
\end{align*}



We can now compute the change in entropy. We will consider several cases.

\subsection{Case where all \(\overline{\sigma_n^{(0)}} \neq 0\)}

Let \(k\) be the first non-zero order for which some \(\overline{\sigma_n^{(k)}}\) does not vanish. Then
\begin{align*}
S &= -\sum_{n} \overline{\sigma_n} \log \overline{\sigma_n}\\
&= - \sum_{n} \left(\overline{\sigma_n^{(0)}} + \lambda^k \overline{\sigma_n^{(k)}} + \Od{\lambda^{k+1}}\right) \log \left(\overline{\sigma_n^{(0)}} \left(1 + \frac{\lambda^k \overline{\sigma_n^{(k)}}}{\overline{\sigma_n^{(0)}}} + \Od{\lambda^{k+1}}\right)\right) \\
&= - \sum_{n} \left(\overline{\sigma_n^{(0)}} + \lambda^k \overline{\sigma_n^{(k)}}\right) \left(\log \overline{\sigma_n^{(0)}} + \frac{\lambda^k \overline{\sigma_n^{(k)}}}{\overline{\sigma_n^{(0)}}}\right) + \Od{\lambda^{k+1}}\\
&= - \sum_{n} \overline{\sigma_n^{(0)}} \log \overline{\sigma_n^{(0)}} - \lambda^k \sum_n \overline{\sigma_n^{(k)}} \left(1+ \log\overline{\sigma_n^{(0)}}\right) + \Od{\lambda^{k+1}}.
\end{align*}
Now, by \(\sum_n \overline{\sigma_n} = 1\) we should always have \(\sum_n \overline{\sigma_n^{(k)}} = 0\). Thus
\[
S = - \sum_{n} \overline{\sigma_n^{(0)}} \log \overline{\sigma_n^{(0)}} - \lambda^k \sum_n \overline{\sigma_n^{(k)}} \log\overline{\sigma_n^{(0)}} + \Od{\lambda^{k+1}}.
\]
The first term, \(- \sum_{n \in I} \overline{\sigma_n^{(0)}} \log \overline{\sigma_n^{(0)}}\), would be the entropy if \(\lambda=0\), that is, with no perturbation. Thus the change in entropy due to the perturbation is
\begin{equation}
\boxed{\Delta S = 
\lambda^k \sum_n \overline{\sigma_n^{(k)}} \log\frac{1}{\;\overline{\sigma_n^{(0)}}\;} + \Od{\lambda^{k+1}}.}
\end{equation}


\subsection{Case where some \(\overline{\sigma_n^{(0)}} = 0\).}

Let \(k_n\) be the first non-zero order for which \(\overline{\sigma_n^{(k_n)}}\) does not vanish. For example, if \(\overline{\sigma_n^{(1)}} \neq 0\), then \(k_n = 1\). If \(\overline{\sigma_n^{(1)}} = 0\) but \(\overline{\sigma_n^{(2)}} \neq 0\), then \(k_n = 2\). Let \(k\) be the minimal \(k_n\). Now let
\[
I = \left\{n : \overline{\sigma_n^{(0)}} \neq 0\right\}.
\]
Then
\begin{align*}
S &= -\sum_{n \in I} \overline{\sigma_n} \log \overline{\sigma_n}-\sum_{n \not\in I} \overline{\sigma_n} \log \overline{\sigma_n} \\
&= - \sum_{n \in I} \overline{\sigma_n} \log \left(\overline{\sigma_n^{(0)}} \left(1 + \Od{\lambda^{k}}\right)\right)  - \sum_{n \not\in I} \overline{\sigma_n} \log\left(\lambda^{k_n}\overline{\sigma_n^{(k_n)}}\left(1 + \Od{\lambda}\right)\right) \\
&= - \sum_{n \in I} \left(\overline{\sigma_n^{(0)}} + \Od{\lambda^{k_n}}\right) \left(\log \overline{\sigma_n^{(0)}}+\Od{\lambda^{k_n}}\right) - \sum_{n \not\in I} \overline{\sigma_n} \left(\log\lambda^{k_n}+\log\overline{\sigma_n^{(k_n)}}+ \Od{\lambda}\right) \\
&= - \sum_{n \in I} \left(\overline{\sigma_n^{(0)}} \log \overline{\sigma_n^{(0)}} + \Od{\lambda^{k_n}}\right) - \sum_{n \not\in I} \left(\overline{\sigma_n}\log\lambda^{k_n} + \Od{\lambda^{k_n}}\right)\\
&= - \sum_{n \in I} \left(\overline{\sigma_n^{(0)}} \log \overline{\sigma_n^{(0)}} + \Od{\lambda^{k}}\right) - \sum_{n \not\in I} \left(\overline{\sigma_n}\log\lambda^{k} + \Od{\lambda^{k}}\right)\\
&= - \sum_{n \in I} \overline{\sigma_n^{(0)}} \log \overline{\sigma_n^{(0)}} - \log\lambda^{k} \sum_{n \not\in I} \overline{\sigma_n} + \Od{\lambda^{k}}\\
&= - \sum_{n \in I} \overline{\sigma_n^{(0)}} \log \overline{\sigma_n^{(0)}} -\log\lambda^{k} \left(1 - \sum_{n \in I} \overline{\sigma_n}\right) + \Od{\lambda^k} \\
&= - \sum_{n \in I} \overline{\sigma_n^{(0)}} \log \overline{\sigma_n^{(0)}} -\log\lambda^{k} \left(1 - \sum_{n \in I} \overline{\sigma_n^{(0)}} - \sum_{n \in I} \lambda^k \overline{\sigma_n^{(k)}}\right) + \Od{\lambda^k} \\
&= - \sum_{n \in I} \overline{\sigma_n^{(0)}} \log \overline{\sigma_n^{(0)}} -\lambda^k\log\frac{1}{\lambda^{k}} \sum_{n \in I} \overline{\sigma_n^{(k)}} + \Od{\lambda^k}
\end{align*}
Here we used that \(\sum_n \overline{\sigma_n^{(0)}} = 1\).
\emph{[The above should be checked. In particular, I need to figure out whether to use big-O or little-o notation.]}
Now, the first term, \(- \sum_{n \in I} \overline{\sigma_n^{(0)}} \log \overline{\sigma_n^{(0)}}\), would be the entropy if \(\lambda=0\), that is, with no perturbation. Thus the change in entropy due to the perturbation is
\begin{equation}\label{eq.deltaS}
\boxed{\Delta S = -\left(\lambda^k\log\frac{1}{\lambda^{k}} \right)\sum_{n \in I} \overline{\sigma_n^{(k)}} + \Od{\lambda^k}.}
\end{equation}

\subsubsection{Case where \(k=1\) (first order correction leads)}

We can compute \(\overline{\sigma_n^{(1)}}\) by combining equations (\ref{eq.tipt.foval}) and (\ref{eq.rho1red}). We get
\begin{equation}\label{eq.focor}
\overline{\sigma_n^{(1)}} = -\frac{i}{\hbar} \sum_m\int_{t_0}^t \bbra{n}{m} \left[\op{H}_I(t^\prime),\; \op{\rho}_0 \right] \bket{n}{m}\,dt^\prime.
\end{equation}
If this is nonzero for some \(n\), (\ref{eq.deltaS}) gives
\[
\Delta S = \left(-\frac{i}{\hbar}\lambda\log\frac{1}{\lambda}\right) \sum_{n \in I} \sum_m \int_{t_0}^t \bbra{n}{m} \left[\op{H}_I(t^\prime),\; \op{\rho}_0 \right] \bket{n}{m}\,dt^\prime + \Od{\lambda}.
\]
Putting in (\ref{eq.rho0.withtau}),
\begin{align*}
\Delta S &= \left(-\frac{i}{\hbar}\lambda\log\frac{1}{\lambda}\right) \int_{t_0}^t \sum_{n \in I} \;\sum_{m,n^\prime, m^\prime, \tilde{n}, \tilde{m}}\;\Bigg(\braket{\alpha_n^{(0)}}{\alpha_{n^\prime}^{(0)}}\braket{\beta_m}{\beta_{m^\prime}} \bbra{\tilde{n}}{\tilde{m}} \op{H}_I\left(t^\prime\right) \bket{n}{m} \tau_{n^\prime, m^\prime, \tilde{n}, \tilde{m}}\\
&\hspace{5em} - \bbra{n}{m} \op{H}_I\left(t^\prime\right) \bket{n^\prime}{m^\prime} \braket{\alpha_{\tilde{n}}^{(0)}}{\alpha_n^{(0)}}\braket{\beta_{\tilde{m}}}{\beta_{m}} \tau_{n^\prime, m^\prime, \tilde{n}, \tilde{m}}\Bigg) \,dt^\prime + \Od{\lambda} \\
&= \left(\frac{i}{\hbar}\lambda\log\frac{1}{\lambda}\right) \int_{t_0}^t\Bigg( -\sum_{n \in I} \;\sum_{m,\tilde{n}, \tilde{m}}\; \bbra{\tilde{n}}{\tilde{m}} \op{H}_I\left(t^\prime\right) \bket{n}{m} \tau_{n, m, \tilde{n}, \tilde{m}}\\
&\hspace{5em} + \sum_{n \in I} \;\sum_{m,\,n^\prime, m^\prime}\;\bbra{n}{m} \op{H}_I\left(t^\prime\right) \bket{n^\prime}{m^\prime} \tau_{n^\prime, m^\prime, {n}, {m}} \Bigg)\,dt^\prime + \Od{\lambda}.
\end{align*}
Since \(\op{\rho}_0\) is Hermitian, we have the property that \(\tau_{n,m,\tilde{n},\tilde{m}} = {\tau_{\tilde{n},\tilde{m},n,m}}^*\). Thus
\[
\Delta S = \left(\frac{i}{\hbar}\lambda\log\frac{1}{\lambda}\right) \sum_{n \in I} \;\sum_{m,\,n^\prime, m^\prime}\; \int_{t_0}^t\,2i\;\Im\left( \bbra{n}{m}\op{H}_I\left(t^\prime\right) \bket{n^\prime}{m^\prime} \tau_{n^\prime, m^\prime, {n}, {m}} \right)\,dt^\prime.
\]
Now, if \(n^\prime \in I\), for each term like \(\Im\left( \bbra{n}{m} \op{H}_I\left(t^\prime\right) \bket{n^\prime}{m^\prime} \tau_{n^\prime, m^\prime, {n}, {m}} \right)\) in the sum, there is an equal but oppositely signed term \(\Im\left( \bbra{n^\prime}{m^\prime} \op{H}_I\left(t^\prime\right) \bket{n}{m} \tau_{n, m, n^\prime, m^\prime} \right)\) in the sum. These terms add to zero. Hence we are left with
\begin{equation}\label{eq.deltaS.fo}
\boxed{\Delta S = \left(-\frac{2}{\hbar}\lambda\log\frac{1}{\lambda}\right) \sum_{n \in I}\sum_{n^\prime \not\in I} \sum_{m,m^\prime}\; \int_{t_0}^t\,\Im\left( \bbra{n}{m} \op{H}_I\left(t^\prime\right) \bket{n^\prime}{m^\prime} \tau_{n^\prime, m^\prime, {n}, {m}} \right)\,dt^\prime.}
\end{equation}
Fascinatingly, we end up with matrix elements that represent transition amplitudes from states with non-zero initial probability (\(n \in I\)) to states with zero initial probability (\(n^\prime \not\in I\)). The entropy growth seems dominated by transitions to these zero-initial-probability states.



\subsubsection{Case where \(k=2\) (second order correction leads)}

On the other hand, (\ref{eq.focor}) could be zero for all \(n\). Then we turn to the second order correction. Let us assume that \(\overline{\sigma_n^{(2)}} \neq 0\) for some \(n\), that is, \(k=2\).

Substituting (\ref{eq.tipt.soval}) into (\ref{eq.deltaS}),
\[
\Delta S = -\lambda^2 \log\frac{1}{\lambda^2}\left(\sum_{n \in I} \bra{\alpha_n^{(0)}}\overline{\op{\rho}_2}\ket{\alpha_n^{(0)}} +  \sum_{n \in I} \sum_{{n^\prime} \not\in D_n} \frac{\left|\bra{\alpha_{n^\prime}^{(0)}} \overline{\op{\rho}_1} \ket{\alpha_n^{(0)}}\right|^2}{\overline{\sigma_n^{(0)}}-\overline{\sigma_{n^\prime}^{(0)}}} \right)+ \Od{\lambda^2}.
\]
Now, in the second term with the double sum, for every term \(\frac{\left|\bra{\alpha_{n^\prime}^{(0)}} \overline{\op{\rho}_1} \ket{\alpha_n^{(0)}}\right|^2}{\overline{\sigma_n^{(0)}}-\overline{\sigma_{n^\prime}^{(0)}}}\) with \(n^\prime \in I\) there is an equal but oppositely signed term \(\frac{\left|\bra{\alpha_{n^\prime}^{(0)}} \overline{\op{\rho}_1} \ket{\alpha_n^{(0)}}\right|^2}{\overline{\sigma_{n^\prime}^{(0)}}-\overline{\sigma_n^{(0)}}}\) in the sum. The portion of the sum over \(n^\prime \in I\) evaluates to zero. We end up with
\begin{equation}\label{eq.deltaSso.intermediate}
\Delta S = -\lambda^2 \log\frac{1}{\lambda^2}\left(\sum_{n \in I} \bra{\alpha_n^{(0)}}\overline{\op{\rho}_2}\ket{\alpha_n^{(0)}} +  \sum_{n \in I} \sum_{n^\prime \not\in I} \frac{\left|\bra{\alpha_{n}^{(0)}} \overline{\op{\rho}_1} \ket{\alpha_{n^\prime}^{(0)}}\right|^2}{\overline{\sigma_n^{(0)}}} \right)+ \Od{\lambda^2}.
\end{equation}

%Let us begin by computing the first term. By (\ref{eq.rho2red}) we get
%\begin{align}\label{eq.sointermediate2}
%\sum_{n \in I} \bra{\alpha_n^{(0)}}\overline{\op{\rho}_2}\ket{\alpha_n^{(0)}} = &\sum_{n \in I}  \sum_m\int_{t_0}^t \int_{t_0}^{t^\prime} \bbra{n}{m} \left( \op{H}_I (t^\prime) \op{H}_I(t^{\prime\prime}) \;\op{\rho}_0 + \op{\rho}_0\; \op{H}_I (t^{\prime\prime}) \op{H}_I(t^{\prime}) \right) \bket{n}{m}\,dt^{\prime\prime}\,dt^\prime \nonumber\\
%&- \sum_{n \in I}\sum_m\int_{t_0}^t \int_{t_0}^{t} \bbra{n}{m}\op{H}_I (t^\prime) \,\op{\rho}_0\, \op{H}_I (t^{\prime\prime})\bket{n}{m}\,dt^{\prime\prime}\,dt^\prime.
%\end{align}


Putting in (\ref{eq.rho1red}) and (\ref{eq.rho2red}),
\begin{align}\label{eq.entropyresultwithrho2}
\Delta S = \frac{1}{\hbar^2}&\lambda^2 \log\frac{1}{\lambda^2}\sum_{n \in I} \Bigg( \\
& \sum_m\int_{t_0}^t \int_{t_0}^{t^\prime} \bbra{n}{m} \left( \op{H}_I (t^\prime) \op{H}_I(t^{\prime\prime}) \;\op{\rho}_0 + \op{\rho}_0\; \op{H}_I (t^{\prime\prime}) \op{H}_I(t^{\prime}) \right) \bket{n}{m}\,dt^{\prime\prime}\,dt^\prime \nonumber\\
-& \sum_m\int_{t_0}^t \int_{t_0}^{t} \bbra{n}{m}\op{H}_I (t^\prime) \,\op{\rho}_0\, \op{H}_I (t^{\prime\prime})\bket{n}{m}\,dt^{\prime\prime}\,dt^\prime \nonumber\\
-& \sum_{n^\prime \not \in I} \frac{1}{\;\overline{\sigma_n^{(0)}}\;} \left|\sum_m\int_{t_0}^t \bbra{n}{m} \left[\op{H}_I(t^\prime),\; \op{\rho}_0 \right] \bket{n^\prime}{m}\;dt^\prime \right|^2
\Bigg) + \Od{\lambda^2}. \nonumber
\end{align}

\subsubsection{Case of separable initial state}

We will now consider the special case where, at the time \(t_0\), the density operator takes the form
\begin{equation}\label{eq.rho0fullinitial}
\op{\rho}_0 = \op{\rho}\left(t_0\right) = \sum_{n,m} \sigma_{n,m}^{(0)} \left(\ket{\alpha_n^{(0)}}\otimes\ket{\beta_m}\right) \left(\bra{\alpha_n^{(0)}}\otimes\bra{\beta_m}\right)
\end{equation}
where \(\left\{\ket{\alpha_n^{(0)}}\right\}\) is \emph{any} orthonormal basis for \(\hilb_A\) and \(\left\{\ket{\beta_m}\right\}\) is \emph{any} orthonormal basis for \(\hilb_B\). We call such states ``separable states''.

(\ref{eq.rho0fullinitial}) implies that the reduced density operator at time \(t_0\) is
\[
\overline{\op{\rho}}_0 = \overline{\op{\rho}}\left(t_0\right) = \sum_n \overline{\sigma_{n}^{(0)}} \ket{\alpha_n^{(0)}}\bra{\alpha_n^{(0)}}
\]
where
\[
\overline{\sigma_n^{(0)}} = \sum_m \sigma_{n,m}^{(0)}.
\]
Now, if we fix a time \(t\), there should be some basis \(\{\ket{\alpha_n}\}\) of \(\hilb_A\) and some quantities \(\overline{\sigma_n}\) such that
\[
\overline{\op{\rho}}(t) = \sum_n \overline{\sigma_n} \ket{\alpha_n}\bra{\alpha_n}.
\]
The eigenstate/eigenvalue decomposition of an operator is unique up to relabeling. Therefore, comparing our equations to (\ref{eq.redrhodef}) and (\ref{eq.rho0red}), we see that our above analysis must hold with the notation unchanged. Indeed, now
\[\tau_{n^\prime, m^\prime, \tilde{n}, \tilde{m}} = \delta_{n^\prime, \tilde{n}}\;\delta_{m^\prime,\tilde{m}}\;\;\sigma_{n^\prime,m^\prime}^{(0)}\]
and
\begin{equation}
\label{eq.rho0full}
\op{\rho}_0 = \sum_{n,m} \sigma_{n,m}^{(0)} \bket{n}{m}\bbra{n}{m}.
\end{equation}

We have
\begin{align*}
\bbra{n}{m} \op{H}_I(t^\prime) \;\op{\rho}_0 \bket{n}{m} 
&= 
\sigma_{n,m}^{(0)} \bbra{n}{m} \op{H}_I(t^\prime) \bket{n}{m} 
\\&=
 \bbra{n}{m} \op{\rho}_0 \;\op{H}_I(t^\prime) \bket{n}{m} 
\end{align*}
so the first-order eigenvalue correction (\ref{eq.focor}) vanishes. We will assume that the second-order eigenvalue correction \(\overline{\sigma_n^{(2)}}\) does not vanish. Then we are in the \(k=2\) case so we consider (\ref{eq.entropyresultwithrho2}).

The first two terms in (\ref{eq.entropyresultwithrho2}) are
\begin{align*}
&\hspace{-4em} \sum_{n \in I} \sum_m\int_{t_0}^t \int_{t_0}^{t^\prime} \bbra{n}{m} \left( \op{H}_I (t^\prime) \op{H}_I(t^{\prime\prime}) \;\op{\rho}_0 + \op{\rho}_0\; \op{H}_I (t^{\prime\prime}) \op{H}_I(t^{\prime}) \right) \bket{n}{m}\,dt^{\prime\prime}\,dt^\prime \\
&\hspace{-4em}- \sum_{n \in I}\sum_m\int_{t_0}^t \int_{t_0}^{t} \bbra{n}{m}\op{H}_I (t^\prime) \,\op{\rho}_0\, \op{H}_I (t^{\prime\prime})\bket{n}{m}\,dt^{\prime\prime}\,dt^\prime \\
&= \sum_{n \in I}\sum_m\sigma_{n,m}^{(0)} \int_{t_0}^t \int_{t_0}^{t^\prime} \bbra{n}{m} \left\{\op{H}_I (t^\prime),\; \op{H}_I(t^{\prime\prime}) \right\} \bket{n}{m}\,dt^{\prime\prime}\,dt^\prime \\
&\hspace{2em} -\sum_m\sum_{n^\star} \sum_{m^\star} \sigma_{n^\star,m^\star}^{(0)}\int_{t_0}^t \int_{t_0}^{t} \bbra{n}{m}\op{H}_I (t^\prime) \bket{n^\star}{m^\star}\bbra{n^\star}{m^\star} \op{H}_I (t^{\prime\prime})\bket{n}{m}\,dt^{\prime\prime}\,dt^\prime \\
&=\sum_{n \in I}\sum_m \sigma_{n,m}^{(0)} \int_{t_0}^t \int_{t_0}^{t} \bbra{n}{m}\op{H}_I (t^\prime) \op{H}_I(t^{\prime\prime}) \bket{n}{m}\,dt^{\prime\prime}\,dt^\prime \\
&\hspace{2em}-\sum_{n \in I}\sum_m\sum_{n^\star} \sum_{m^\star} \sigma_{n^\star,m^\star}^{(0)}\int_{t_0}^t \int_{t_0}^{t} \bbra{n^\star}{m^\star} \op{H}_I (t^{\prime\prime})\bket{n}{m} \bbra{n}{m}\op{H}_I (t^\prime) \bket{n^\star}{m^\star}\,dt^{\prime\prime}\,dt^\prime \\
&= \sum_{n \in I}\sum_m\sigma_{n,m}^{(0)} \int_{t_0}^t \int_{t_0}^{t} \bbra{n}{m}\op{H}_I (t^\prime) \op{H}_I(t^{\prime\prime}) \bket{n}{m}\,dt^{\prime\prime}\,dt^\prime \\
&\hspace{2em}-\sum_{n^\star} \sum_{m^\star} \sigma_{n^\star,m^\star}^{(0)}\int_{t_0}^t \int_{t_0}^{t} \bbra{n^\star}{m^\star} \op{H}_I (t^{\prime\prime})\left[\left(\sum_{n\in I} \ket{\alpha_n^{(0)}}\bra{\alpha_n^{(0)}}\right)\otimes\cdot\right]\op{H}_I (t^\prime) \bket{n^\star}{m^\star}\,dt^{\prime\prime}\,dt^\prime \\
&=\sum_{n \in I}\sum_m \sigma_{n,m}^{(0)} \int_{t_0}^t \int_{t_0}^{t} \bbra{n}{m}\op{H}_I (t^\prime) \op{H}_I(t^{\prime\prime}) \bket{n}{m}\,dt^{\prime\prime}\,dt^\prime \\
&\hspace{2em}-\sum_{n} \sum_{m} \sigma_{n,m}^{(0)}\int_{t_0}^t \int_{t_0}^{t} \bbra{n}{m} \op{H}_I (t^{\prime\prime})\left[\left(1-\sum_{n^\prime\not\in I} \ket{\alpha_{n^\prime}^{(0)}}\bra{\alpha_{n^\prime}^{(0)}}\right)\otimes\cdot\right]\op{H}_I (t^\prime) \bket{n}{m}\,dt^{\prime\prime}\,dt^\prime \\
&= \sum_{n \in I}\sum_m \sigma_{n,m}^{(0)} \left[\int_{t_0}^t \int_{t_0}^{t} \bbra{n}{m}\op{H}_I (t^\prime) \op{H}_I(t^{\prime\prime}) \bket{n}{m}\,dt^{\prime\prime}\,dt^\prime - \int_{t_0}^t \int_{t_0}^{t} \bbra{n}{m}\op{H}_I (t^\prime) \op{H}_I(t^{\prime\prime}) \bket{n}{m}\,dt^{\prime\prime}\,dt^\prime\right]\\
&\hspace{2em}+ \sum_{n} \sum_m \sigma_{n,m}^{(0)} \sum_{n^\prime \not\in I} \sum_{m^\prime} \int_{t_0}^t \bbra{n}{m} \op{H}_I (t^{\prime\prime})\bket{n^\prime}{m^\prime} \,dt^{\prime\prime}\int_{t_0}^t \bbra{n^\prime}{m^\prime} \op{H}_I (t^{\prime})\bket{n}{m}\,dt^\prime \\
&=\sum_{n \in I} \sum_{n^\prime \not\in I} \sum_m \sigma_{n,m}^{(0)} 
 \sum_{m^\prime} \left|\int_{t_0}^t \bbra{n}{m} \op{H}_I (t^{\prime})\bket{n^\prime}{m^\prime} \,dt^{\prime}\right|^2.
\end{align*}

Meanwhile, the third term in (\ref{eq.entropyresultwithrho2}) is
\begin{align*}
-\sum_{n \in I}\sum_{n^\prime \not \in I} &\frac{1}{\;\overline{\sigma_n^{(0)}}\;} \left|\sum_m\int_{t_0}^t \bbra{n}{m} \left[\op{H}_I(t^\prime),\; \op{\rho}_0 \right] \bket{n^\prime}{m}\;dt^\prime \right|^2 \\
&= -\sum_{n \in I}\sum_{n^\prime \not \in I} \frac{1}{\;\overline{\sigma_n^{(0)}}\;} \left|\sum_m\int_{t_0}^t \bbra{n}{m} \op{H}_I(t^\prime) 
 \bket{n^\prime}{m}\;\left({\sigma_{n,m}^{(0)}} - \cancel{{\sigma_{n^\prime,m}^{(0)}}}\right)\;dt^\prime \right|^2\Bigg) + \Od{\lambda^2} \\
 &= -\sum_{n \in I}\sum_{n^\prime \not \in I} \frac{1}{\sum_{\tilde{m}} \sigma_{n,\tilde{m}}^{(0)}} \left|\sum_m{\sigma_{n,m}^{(0)}}\int_{t_0}^t \bbra{n}{m} \op{H}_I(t^\prime) 
 \bket{n^\prime}{m}\;dt^\prime \right|^2.
\end{align*}

Putting this all back into (\ref{eq.entropyresultwithrho2}), we obtain the rather interesting result that
\begin{align}\label{eq.Sredresult.separable}
\Delta S =\frac{1}{\hbar^2}\lambda^2 \log\frac{1}{\lambda^2} \sum_{n\in I}\sum_{n^\prime \not\in I} \Bigg(& \sum_m \sigma_{n,m}^{(0)} \sum_{m^\prime} \left|\int_{t_0}^t \bbra{n}{m}\op{H}_I (t^{\prime})\bket{n^\prime}{m^\prime} \,dt^{\prime}\right|^2 \\
&- \frac{1}{\sum_{\tilde{m}} \sigma_{n,\tilde{m}}^{(0)}} \left|\sum_m{\sigma_{n,m}^{(0)}}\int_{t_0}^t \bbra{n}{m} \op{H}_I(t^\prime) 
 \bket{n^\prime}{m}\;dt^\prime \right|^2\Bigg) + \Od{\lambda^2}. \nonumber
\end{align}
Assuming that these terms do not vanish, we see that the entropy grows proportionally to \(\left(t-t_0\right)^2\).

We also observe that this expression vanishes if \(\hilb_B\) has only a single possible state. This situation is equivalent to taking the entropy of the whole system rather than the subsystem entropy. We have thus shown that the entropy of the whole system is conserved (at the order of \(\lambda^2 \log\frac{1}{\lambda^2}\), for separable states).

\subsubsubsection{Case of pure, separable initial state}

If the initial state is pure and separable, let \(\bket{n}{m}\) be the initial state so that \(\op{\rho}_0 = \bket{n}{m}\bbra{n}{m}\). Then (\ref{eq.Sredresult.separable}) simplifies to
\begin{equation}\label{eq.Sredresult.separablepure}
\Delta S =\frac{1}{\hbar^2}\lambda^2 \log\frac{1}{\lambda^2} \sum_{n^\prime\neq n} \;\sum_{m^\prime\neq m} \left|\int_{t_0}^t \bbra{n}{m} \op{H}_I (t^{\prime})\bket{n^\prime}{m^\prime} \,dt^{\prime}\right|^2 + \Od{\lambda^2}.
\end{equation}
We observe a crucial exclusion principle: in our expression for entropy, we only have transition amplitudes where the state changes in both subsystems, never transition amplitudes for a change in only one subsystem. Furthermore, since this expression is symmetric with respect to \(n\) and \(m\), the change in subsystem entropy for \(\hilb_A\) is the same as the change in subsystem entropy for \(\hilb_B\).

\subsubsubsection{Upper and lower bounds for entropy growth}

We can apply the triangle inequality to (\ref{eq.Sredresult.separable}) to obtain an illuminating lower bound on \(\Delta S\) for separable initial states. We have
\begin{align*}
\frac{1}{\sum_m \sigma_{n,m}^{(0)}} \left|\sum_m{\sigma_{n,m}^{(0)}}\int_{t_0}^t \bbra{n}{m} \op{H}_I(t^\prime) 
 \bket{n^\prime}{m}\;dt^\prime \right|^2
& \leq \frac{1}{\sum_{\tilde{m}} \sigma_{n,\tilde{m}}^{(0)}} \sum_{m} \left|{\sigma_{n,m}^{(0)}}\int_{t_0}^t \bbra{n}{m} \op{H}_I(t^\prime) 
 \bket{n^\prime}{m}\;dt^\prime \right|^2 \\
 &= \sum_m \frac{\sigma_{n,m}^{(0)}}{\sum_{\tilde{m}} \sigma_{n,\tilde{m}}^{(0)}} \;\sigma_{n,m}^{(0)} \left|\int_{t_0}^t \bbra{n}{m} \op{H}_I(t^\prime) 
 \bket{n^\prime}{m}\;dt^\prime \right|^2 \\
 &\leq \sum_m \sigma_{n,m}^{(0)} \left|\int_{t_0}^t \bbra{n}{m} \op{H}_I(t^\prime) 
 \bket{n^\prime}{m}\;dt^\prime \right|^2.
\end{align*}
Applying this to (\ref{eq.Sredresult.separable}), we see at order \(\lambda^2 \log \frac{1}{\lambda^2}\) that
\begin{equation}\label{eq.dSlowerbound}
\Delta S \geq \frac{1}{\hbar^2}\lambda^2 \log\frac{1}{\lambda^2} \sum_{n\in I}\sum_{n^\prime \not\in I} \sum_m \sigma_{n,m}^{(0)} \sum_{m^\prime \neq m} \left|\int_{t_0}^t \bbra{n}{m}\op{H}_I (t^{\prime})\bket{n^\prime}{m^\prime} \,dt^{\prime}\right|^2.
\end{equation}
This result is profound for two reasons. First, it shows that the change in entropy for a separable state is non-negative. This matches our intuition. A separable state is an unentangled state---in this sense, it already has minimal entropy, so it seems natural that its entropy can only grow. Second, comparing (\ref{eq.dSlowerbound}) to (\ref{eq.Sredresult.separablepure}), we see that the lower bound (\ref{eq.dSlowerbound}) is exactly what we would expect if entropy was linear with respect to \(\op{\rho}\) (i.e., if we could compute entropy by the superposition principle). That is, we can rewrite (\ref{eq.dSlowerbound}) as
\begin{equation}\label{eq.dSlowerbound.linearcomparison}
\Delta S\, \bigg|_{\op{\rho}_0 = \sum_{n,m} \sigma_{n,m}^{(0)} \bket{n}{m}\bbra{n}{m}} \geq \sum_{n,m} \sigma_{n,m}^{(0)} \; \Delta S\, \bigg|_{\op{\rho}_0 = \bket{n}{m}\bbra{n}{m}}
\end{equation}
Evidently the non-linearity of entropy due to \(\log\) causes entropy to be higher than if it was linear. (\ref{eq.dSlowerbound}) is a kind of \emph{linear approximation} to the entropy. More precisely, it is an approximation assuming that linear superposition of the density operator applies for entropy.

\emph{The second ``profound result'' above is actually erroneous, because the sum is over \(n^\prime \not\in I\), not \(n^\prime \neq n\). This discussion should be removed.}

We can also obtain an upper bound for (\ref{eq.Sredresult.separable}) by keeping only the first term. Then, at order \(\lambda^2 \log \frac{1}{\lambda^2}\),
\begin{align}\label{eq.dSbounds}
\frac{1}{\hbar^2}\lambda^2 &\log\frac{1}{\lambda^2} \sum_{n\in I}\sum_{n^\prime \not\in I} \sum_m \sigma_{n,m}^{(0)} \sum_{m^\prime \neq m} \left|\int_{t_0}^t \bbra{n}{m}\op{H}_I (t^{\prime})\bket{n^\prime}{m^\prime} \,dt^{\prime}\right|^2 \\
&\leq \;\;\Delta S \;\;\leq\;\; \frac{1}{\hbar^2}\lambda^2 \log\frac{1}{\lambda^2} \sum_{n\in I}\sum_{n^\prime \not\in I} \sum_m \sigma_{n,m}^{(0)} \sum_{m^\prime} \left|\int_{t_0}^t \bbra{n}{m}\op{H}_I (t^{\prime})\bket{n^\prime}{m^\prime} \,dt^{\prime}\right|^2. \nonumber
\end{align}


\section{Example: Interaction between Qubits}

In this section we will study a simple bipartite system and calculate the growth in subsystem entropy due to a weak interaction. Consider the Hilbert space \(\{\ket{0}, \ket{1}\}\otimes\{\ket{0}, \ket{1}\}\) describing two qubits. To simplify notation, write \(\ket{ab}\) to mean \(\ket{a}\otimes\ket{b}\).

Suppose the qubits interact according to the Hamiltonian
\[
\op{H} = \op{S}_z^{A}\otimes\op{S}_z^B + \lambda \op{S}_x^A \otimes \op{S}_x^B.
\]
One can calculate that this Hamiltonian has eigenstates and eigenvalues
\begin{align*}
\ket{E_0} &= \frac{1}{\sqrt{2}} \left(\ket{10}-\ket{01}\right) &
E_0 &= \frac{\hbar}{2} \left(-1-\lambda\right) \\
\ket{E_1} &= \frac{1}{\sqrt{2}} \left(\ket{10}+\ket{01}\right) &
E_1 &= \frac{\hbar}{2} \left(-1+\lambda\right) \\
\ket{E_2} &= \frac{1}{\sqrt{2}} \left(\ket{11}-\ket{00}\right) &
E_2 &= \frac{\hbar}{2} \left(1-\lambda\right) \\
\ket{E_3} &= \frac{1}{\sqrt{2}} \left(\ket{11}+\ket{00}\right) &
E_3 &= \frac{\hbar}{2} \left(1+\lambda\right).
\end{align*}
Let's start by calculating the exact change in entropy; then we can calculate the perturbative result and compare.

Suppose at \(t=0\) we start in the initial state \(\ket{11} = \frac{1}{\sqrt{2}}\left(\ket{E_2}+\ket{E_3}\right)\). At time \(t\), the state is then
\[
\ket{\Psi(t)} = \frac{1}{\sqrt{2}}\left(e^{-itE_2/\hbar}\ket{E_2} + e^{-itE_3/\hbar} \ket{E_3}\right)
\]
and the density operator is
\[
\op{\rho} = \frac{1}{2}\ket{E_2}\bra{E_2} + \frac{1}{2}\ket{E_3}\bra{E_3} + \frac{1}{2} e^{it\left(E_3-E_2\right)/\hbar} \ket{E_2}\bra{E_3} + \frac{1}{2} e^{-it\left(E_3 - E_2\right)/\hbar} \ket{E_3}\bra{E_2}.
\]
Then the reduced density operator for the first qubit is
\begin{align*}
\overline{\op{\rho}} &= \cdot\otimes\bra{0} \op{\rho} \cdot\otimes\ket{0} + \cdot\otimes\bra{1} \op{\rho} \cdot\otimes\ket{1} \\
&= \frac{1}{2}\left(1 + \cos \left[\left(E_3-E_2\right)\frac{t}{\hbar}\right] \right)\ket{1}\bra{1} + \frac{1}{2}\left(1 -\cos \left[\left(E_3-E_2\right)\frac{t}{\hbar}\right] \right)\ket{0}\bra{0}.
\end{align*}
Recognizing that \(\frac{E_3-E_2}{\hbar} = \lambda\), we see that the Von Neumann subsystem entropy is
\begin{align*}
S &= \frac{1}{2}\left(1 + \cos \lambda t \right)\log \left(1+\cos \frac{\lambda t}{2}\right)
+ \frac{1}{2}\left(1 - \cos \lambda t \right)\log \left(1-\cos \frac{\lambda t}{2}\right).
\end{align*}
Approximating \(\cos x \approx 1-x^2\), this becomes
\begin{align*}
S &\approx \frac{1}{2}\left(2-\lambda^2 t^2\right) \log \left(2-\lambda^2 t^2/4\right) + \frac{1}{2}\lambda^2 t^2 \log \left(\lambda^2 t^2\right) \\
&\approx \log 2 - \frac{1}{2}\lambda^2 t^2 \log \left(\frac{1}{\lambda^2 t^2}\right).
\end{align*}
We observe that the leading order term for the change in entropy is of order \(\lambda^2t^2 \log \frac{1}{\lambda^2 t^2}\). Interestingly, the entropy decreases with time. This makes sense given that we started in a maximally entangled state (i.e. a Bell state). \emph{[Actually, I'm not sure that this is a good explanation of the decreasing entropy...]}

We did the derivation above by solving the Schr\"odinger equation exactly. Let's now apply our perturbative calculation and see that we get the same result. The Schr\"odinger picture Hamiltonian is \(\op{H} = \op{H}_0 + \lambda \op{V}\) where
\begin{align*}
\op{H}_0 &= \frac{\hbar}{2} \left(-\ket{E_0}\bra{E_0} - \ket{E_1}\bra{E_1} + \ket{E_2}\bra{E_2} + \ket{E_3}\bra{E_3}\right) \\
\op{V} &= \frac{\hbar}{2} \left(-\ket{E_0}\bra{E_0} + \ket{E_1}\bra{E_1} - \ket{E_2}\bra{E_2} + \ket{E_3}\bra{E_3}\right).
\end{align*}
Because \([\op{H}_0, \op{V}] = 0\) we have \([\op{U}(0, t), \op{V}] = 0\). Thus the interaction picture Hamiltonian is just
\[\op{H}_I = \op{V} = \frac{\hbar}{2} \left(-\ket{E_0}\bra{E_0} + \ket{E_1}\bra{E_1} - \ket{E_2}\bra{E_2} + \ket{E_3}\bra{E_3}\right).\]
We must now diagonalize the initial density operator in a basis that easily separates into bases for the two subspaces. Indeed, we get
\begin{align*}
\op{\rho} &= \frac{1}{2}\ket{E_2}\bra{E_2} + \frac{1}{2}\ket{E_3}\bra{E_3} + \frac{1}{2} e^{it\lambda} \ket{E_2}\bra{E_3} + \frac{1}{2} e^{-it\lambda} \ket{E_3}\bra{E_2} \\
&= \left(\frac{1}{\sqrt{2}} e^{it\lambda}{\ket{E_2}} + \frac{1}{\sqrt{2}}\ket{E_3}\right) \left(\frac{1}{\sqrt{2}} e^{-it\lambda}{\bra{E_2}} + \frac{1}{\sqrt{2}}\bra{E_3}\right) \\
&= \left(\frac{1}{2} \left(1-e^{it\lambda}\right){\ket{00}} + \frac{1}{2}\left(1+e^{it\lambda}\right)\ket{11}\right) \left(\frac{1}{2} \left(1+e^{-it\lambda}\right){\bra{00}} + \frac{1}{2}\left(1+e^{-it\lambda}\right)\bra{11}\right) \\
\end{align*}
\emph{[This is not a separable state, so we cannot use the perturbation formula. Should maybe change the example to a separable state.]}



\appendix
\newpage
\hrule
\vspace{3em}
{\Huge Appendices}
\vspace{3em}
\hrule
\vspace{2em}

These are extra calculations that are not relevant to the above discussion. They should be removed from documents to be shared. (I don't delete them now because they could be useful in future work.)


\section{The first-order approximation for pure states in scattering}

We will continue for scattering between two particles. We will assume that the particles are distinguishable so that we don't need to restrict ourselves to symmetric or antisymmetric states. We will also assume that the particles interact in a way that only depends on the distance between them. That is, in the position basis, we can write
\begin{equation}\label{eq.Vposbasis}
\bra{x_1, x_2}\op{V}(t)\ket{\psi} = \int_{x_1,x_2} {d^3 x_1 \,d^3x_2}\; V(x_1 - x_2) \braket{x_1, x_2}{\psi}.
\end{equation}
(Note that this is not time dependent.) Take \(\op{H}_0\) to be the free particle Hamiltonian.

It will be easiest to work in the momentum basis. We will do our calculations for momentum eigenstates---that is, plane waves. Write momentum eigenstates as \(\ket{k_1, k_2}\). In the position basis these have wavefunctions
\begin{equation}\label{eq.momentuminpos}
\braket{x_1, x_2}{k_1, k_2} = \frac{1}{(2\pi)^3}\;e^{ik_1\cdot x_1 + ik_2\cdot x_2}.
\end{equation}
Here \(x_1\) and \(x_2\) are the (3-vector) positions of the first and second particles and \(k_1\) and \(k_2\) are the (3-vector) momenta of the first and second particle. Also, we note that \(\ket{k_1, k_2}\) has energy \(\frac{1}{2m_1}{k_1}^2 + \frac{1}{2m_2}{k_2}^2\) and so
\begin{equation}
\label{eq.timeevonk}
\op{U}(t_0, t) \ket{k_1, k_2} = e^{-\frac{i}{\hbar}\left(t-t_0\right)\left(\frac{1}{2m_1}{k_1}^2 + \frac{1}{2m_2}{k_2}^2\right)} \ket{k_1, k_2}.
\end{equation}

Let \(\widetilde{V}(p)\) be the Fourier transform of \(V(x_1 - x_2)\) so that
\begin{equation}
\label{eq.ftV}
V(x_1 - x_2) = \int_{-\infty}^{\infty} \frac{d^3 p}{(2 \pi)^3} \widetilde{V}(p) e^{ip\cdot(x_1 - x_2)}.
\end{equation}

We now consider (\ref{eq.firsttdpe}) and take the limits \(t_0 \to -\infty\) and \(t \to \infty\) because, in scattering, we measure states long before and long after the scattering occurs. Then, dropping the higher-order terms,
\[
\ket{\Psi_I, \infty} = \ket{\Psi_I, -\infty} - \frac{i}{\hbar} \int_{-\infty}^\infty \op{H}_I(t)\ket{\Psi_I, -\infty} \,dt.
\]
Take the initial state to be \(\ket{\Psi_I, -\infty} = \ket{k_1, k_2}\). Multiplying by the bra \(\bra{k_1^\prime, k_2^\prime}\),
\begin{align}
\braket{k_1^\prime, k_2^\prime}{\Psi_I, \infty} &= \braket{k_1^\prime, k_2^\prime}{k_1, k_2} - \frac{i}{\hbar} \int_{-\infty}^\infty \bra{k_1^\prime, k_2^\prime} \op{H}_I(t)\ket{k_1, k_2} \,dt \nonumber\\
&= \delta^3(k_1^\prime - k_1)\delta^3(k_2^\prime - k_2) - \frac{i}{\hbar} \int_{-\infty}^\infty \bra{k_1^\prime, k_2^\prime} \op{H}_I(t)\ket{k_1, k_2} \,dt. \label{eq.intermed}
\end{align}
This motivates us to compute the matrix element \(\bra{k_1^\prime, k_2^\prime} \op{H}_I(t)\ket{k_1, k_2}\). We can start by using (\ref{eq.timeevonk}):
\begin{align*}
\bra{k_1^\prime, k_2^\prime} \op{H}_I(t)\ket{k_1, k_2}
&= \bra{k_1^\prime, k_2^\prime} \op{U}(t_0, t)^\dagger \op{V}(t) \op{U}(t_0, t) \ket{k_1, k_2} \nonumber\\
&= \bra{k_1^\prime, k_2^\prime} e^{\frac{i}{\hbar}\left(t-t_0\right)\left(\frac{1}{2m_1}{k_1^\prime}^2 + \frac{1}{2m_2}{k_2^\prime}^2\right)} \op{V}(t) e^{-\frac{i}{\hbar}\left(t-t_0\right)\left(\frac{1}{2m_1}{k_1}^2 + \frac{1}{2m_2}{k_2}^2\right)} \ket{k_1, k_2} \nonumber\\
&= e^{-\frac{i}{\hbar}\left(t-t_0\right)\left(\frac{1}{2m_1}{k_1^\prime}^2 + \frac{1}{2m_2}{k_2^\prime}^2 - \frac{1}{2m_1}{k_1}^2 - \frac{1}{2m_2}{k_2}^2\right)} \bra{k_1^\prime, k_2^\prime} \op{V}(t) \ket{k_1, k_2}.
\end{align*}
We see that
\begin{align*}
\hspace{-1em}\bra{k_1^\prime, k_2^\prime} \op{V}(t) \ket{k_1, k_2} &= \bra{k_1^\prime, k_2^\prime} \left\{\int_{x_1, x_2} {d^3 x_1 d^3 x_2\;} \ket{x_1, x_2}\bra{x_1, x_2}\right\}\op{V}(t) \ket{k_1, k_2} \nonumber\\
&= \int_{x_1, x_2} {d^3 x_1 d^3 x_2\;} \braket{k_1^\prime, k_2^\prime}{x_1, x_2}\bra{x_1, x_2}\op{V}(t) \ket{k_1, k_2} \nonumber\\
&= \int_{x_1, x_2} \frac{d^3 x_1 d^3 x_2}{(2\pi)^6}\; e^{-ik_1^\prime\cdot x_1 - ik_2^\prime\cdot x_2}V(x_1 - x_2) e^{ik_1\cdot x_1 + ik_2\cdot x_2} &&\text{(by (\ref{eq.Vposbasis}) and (\ref{eq.momentuminpos}))}\nonumber\\
&= \int_{-\infty}^{\infty} \frac{d^3p}{(2\pi)^9}\int_{x_1, x_2} {d^3 x_1 d^3 x_2\;\;}e^{-ik_1^\prime\cdot x_1 - ik_2^\prime\cdot x_2}\,\widetilde{V}(p)\,e^{ip\cdot(x_1-x_2)} e^{ik_1\cdot x_1 + ik_2\cdot x_2}
&&\text{(by (\ref{eq.ftV}))}\nonumber\\
&= \int_{-\infty}^{\infty} \frac{d^3p}{(2\pi)^9}\widetilde{V}(p) \int_{x_1, x_2} {d^3 x_1 d^3 x_2\;\;}e^{i(k_1-k_1^\prime+p)\cdot x_1}e^{i(k_2-k_2^\prime-p)\cdot x_2}\nonumber\\
&= \int_{-\infty}^{\infty} \frac{d^3p}{(2\pi)^9}\widetilde{V}(p) \,(2\pi)^6\,\delta^3(k_1-k_1^\prime+p)\,\delta^3(k_2-k_2^\prime-p)\nonumber\\
&= \frac{1}{(2\pi)^3}\;\widetilde{V}(k_1^\prime-k_1) \;\delta^3(k_1^\prime-k_1-k_2^\prime+k_2).
\end{align*}
so 
\[
\bra{k_1^\prime, k_2^\prime} \op{H}_I(t)\ket{k_1, k_2}
=
e^{-\frac{i}{\hbar}\left(t-t_0\right)\left(\frac{1}{2m_1}{k_1^\prime}^2 + \frac{1}{2m_2}{k_2^\prime}^2 - \frac{1}{2m_1}{k_1}^2 - \frac{1}{2m_2}{k_2}^2\right)}\frac{1}{(2\pi)^3}\;\widetilde{V}(k_1^\prime-k_1) \;\delta^3(k_1^\prime-k_1-k_2^\prime+k_2)
\]
If we integrate over all time, we get a delta function:
\begin{equation}\label{eq.HIintegrated}
\hspace{-2em}
\int_{-\infty}^\infty dt \,\bra{k_1^\prime, k_2^\prime} \op{H}_I(t)\ket{k_1, k_2}
=
\frac{\hbar}{(2\pi)^2}\,\widetilde{V}(k_1^\prime-k_1)\,\delta\left({\textstyle\frac{1}{2m_1}{k_1^\prime}^2 + \frac{1}{2m_2}{k_2^\prime}^2 - \frac{1}{2m_1}{k_1}^2 - \frac{1}{2m_2}{k_2}^2}\right) \,\delta^3(k_1^\prime-k_1-k_2^\prime+k_2).
\end{equation}
Putting this into (\ref{eq.intermed}),
\begin{align}
\hspace{-1.5em}\braket{k_1^\prime, k_2^\prime}{\Psi_I, \infty}
=& \;\;\delta^3(k_1^\prime - k_1)\;\delta^3(k_2^\prime - k_2) \nonumber\\&\;\;\;\;- \frac{i}{(2\pi)^2}\, \widetilde{V}(k_1^\prime-k_1) \;\delta^3(k_1^\prime-k_1-k_2^\prime+k_2) \;\delta\left(\frac{{k_1^\prime}^2}{2m_1} + \frac{{k_2^\prime}^2}{2m_2} - \frac{{k_1}^2}{2m_1} - \frac{{k_2}^2}{2m_2}\right).
\end{align}
Interestingly, we see that the delta functions cause energy and momentum to be conserved.

\section{Reduced density operator}

We are interested in the situation where the second particle exits the system and becomes lost. We therefore consider the reduced density operator
\begin{align*}
\op{\rho}_\text{reduced}(t) &= \int d^3k_2 \bra{k_2} \op{\rho}(t) \ket{k_2} \nonumber\\
&= \op{\rho}_\text{reduced}\left(t_0\right) -  \frac{i}{\hbar} \int_{t_0}^tdt^\prime\int_{-\infty}^\infty d^3k_2\bra{k_2}\left[\op{H}_I(t^\prime),\; \op{\rho}\left(t_0\right)\right]\ket{k_2} + \Od{\op{H}_I(t)^2}.
\end{align*}
In the momentum basis the matrix elements are
\[
\bra{k_1^\prime}\op{\rho}_\text{reduced}(t)\ket{k_1} = \bra{k_1^\prime}\op{\rho}_\text{reduced}\left(t_0\right)\ket{k_1} -  \frac{i}{\hbar} \int_{t_0}^tdt^\prime\int_{-\infty}^\infty d^3k_2 \bra{k_1^\prime, k_2}\left[\op{H}_I(t^\prime),\; \op{\rho}\left(t_0\right)\right]\ket{k_1, k_2} + \Od{\op{H}_I(t)^2}.
\]
As before, we drop the higher order terms and take \(t \to \infty\) and \(t_0 \to -\infty\). Then
\begin{align*}
\hspace{-5em}\bra{{k_1}^\prime}\op{\rho}_\text{reduced}(\infty)\ket{k_1} =\hspace{1.7em}&\hspace{-1.5em} \bra{k_1^\prime}\op{\rho}_\text{reduced}\left(-\infty\right)\ket{k_1} -  \frac{i}{\hbar} \int_{-\infty}^\infty dt \int_{-\infty}^\infty d^3k_2 \bra{k_1^\prime, k_2}\left[\op{H}_I(t),\; \op{\rho}\left(-\infty\right)\right]\ket{k_1, k_2} 
\\=\hspace{1.7em}&\hspace{-1.5em}
\bra{k_1^\prime}\op{\rho}_\text{reduced}\left(-\infty\right)\ket{k_1}
\\&-  \frac{i}{\hbar} \int_{-\infty}^\infty d^3k_2 \,d^3\,\widetilde{k}_1\,d^3\,\widetilde{k}_2 \left\{\int_{-\infty}^\infty dt \bra{k_1^\prime, k_2}\op{H}_I(t) \ket{\widetilde{k}_1, \widetilde{k}_2}
\right\}\bra{\widetilde{k}_1, \widetilde{k}_2}\op{\rho}\left(-\infty\right)\ket{k_1, k_2}
\\&+  \frac{i}{\hbar} \int_{-\infty}^\infty d^3k_2\,d^3\,\widetilde{k}_1\,d^3\,\widetilde{k}_2 \bra{k_1^\prime, k_2} \op{\rho}\left(-\infty\right)
\ket{\widetilde{k}_1, \widetilde{k}_2}\left\{\int_{-\infty}^\infty dt\bra{\widetilde{k}_1, \widetilde{k}_2}\op{H}_I(t)\ket{k_1, k_2}\right\}.\end{align*}
Plugging in our result (\ref{eq.HIintegrated}),
\begin{align*}
\bra{{k_1}^\prime}\op{\rho}_\text{reduced}(\infty)\ket{k_1}
=\hspace{1.7em}&\hspace{-1.5em}
\bra{k_1^\prime}\op{\rho}_\text{reduced}\left(-\infty\right)\ket{k_1}
\\&-  i \int_{-\infty}^\infty d^3k_2 \,d^3\,\widetilde{k}_1\,d^3\,\widetilde{k}_2 \;\;{\Bigg(}\frac{1}{(2\pi)^2}\; \;\widetilde{V}\left(k_1^\prime - \widetilde{k}_1\right)\;\delta\left({\textstyle\frac{{k_1^\prime}^2}{2m_1} + \frac{{k_2}^2}{2m_2} - \frac{{\widetilde{k}_1}^2}{2m_1} - \frac{{\widetilde{k}_2}^2}{2m_2}}\right) \\&\hspace{9em}\delta^3(k_1^\prime - \widetilde{k}_1 - k_2 + \widetilde{k}_2)    \;\;\bra{\widetilde{k}_1, \widetilde{k}_2}\op{\rho}\left(-\infty\right)\ket{k_1, k_2}{\Bigg)}
\\&+  i \int_{-\infty}^\infty d^3k_2 \,d^3\,\widetilde{k}_1\,d^3\,\widetilde{k}_2 \;\;{\Bigg(}\frac{1}{(2\pi)^2}\; \;\widetilde{V}\left(\widetilde{k}_1 - {k}_1\right)\;\delta\left({\textstyle\frac{{\widetilde{k}_1}^2}{2m_1} + \frac{{\widetilde{k}_2}^2}{2m_2} - \frac{{{k}_1}^2}{2m_1} - \frac{{{k}_2}^2}{2m_2}}\right) \\&\hspace{9em}\delta^3(\widetilde{k}_1 - {k}_1 - \widetilde{k}_2 + {k}_2)    \;\;\bra{{k}_1^\prime, {k}_2}\op{\rho}\left(-\infty\right)\ket{\widetilde{k}_1, \widetilde{k}_2}{\Bigg)}.
\end{align*}
Finally, we use the delta functions to get rid of the \(\widetilde{k}_1\) integrals. This gives us
\begin{align*}
\hspace{-4em}\bra{{k_1}^\prime}\op{\rho}_\text{reduced}(\infty)\ket{k_1}
= 
\bra{k_1^\prime}&\op{\rho}_\text{reduced}\left(-\infty\right)\ket{k_1} \\
&-  i \int_{-\infty}^\infty d^3k_2 \,d^3\,\widetilde{k}_2 \;\;\frac{1}{(2\pi)^2}\; \;\widetilde{V}\left(k_2 - 
\widetilde{k}_2\right)\; \bra{k_1^\prime + \widetilde{k}_2 - k_2,\; \widetilde{k}_2}\op{\rho}\left(-\infty\right)\ket{k_1, k_2} \;D(k_1^\prime, k_2, \widetilde{k}_2)
\nonumber\\&+  i \int_{-\infty}^\infty d^3k_2 \,d^3\,\widetilde{k}_2 \;\;\frac{1}{(2\pi)^2}\; \;\widetilde{V}\left(\widetilde{k}_2 - 
{k}_2\right)\; \bra{{k}_1^\prime, {k}_2}\op{\rho}\left(-\infty\right)\ket{k_1 + \widetilde{k}_2 - k_2,\; \widetilde{k}_2} \;D(k_1, k_2, \widetilde{k}_2). \nonumber
\end{align*}
where
\begin{align*}
D(k_1, k_2, \widetilde{k}_2) &= \delta\left({\textstyle \frac{1}{2}\;{{k}_2}^2\left(\frac{1}{m_1}-\frac{1}{m_2}\right)+ \frac{1}{2}\;{\widetilde{k}_2}^2\left(\frac{1}{m_1}+\frac{1}{m_2}\right) - \frac{1}{2m_1}\left(-k_1\cdot \widetilde{k}_2 + k_1 \cdot k_2 + \widetilde{k}_2 \cdot k_2\right)}\right).
\end{align*}
Then, assuming \(\widetilde{V}\) is an even function,
\begin{align}\label{eq.reddensevolution}
\hspace{-4em}\bra{{k_1}^\prime}&\op{\rho}_\text{reduced}(\infty)\ket{k_1}
= {\Bigg (}
\bra{k_1^\prime}\op{\rho}_\text{reduced}\left(-\infty\right)\ket{k_1}\;\;+
 \\&\nonumber \int_{-\infty}^\infty \frac{d^3k_2 \,d^3\,\widetilde{k}_2}{(2\pi)^2}\;\widetilde{V}\left(k_2 - 
\widetilde{k}_2\right)\;\left(P\left(k_1,k_1^\prime k_2, \widetilde{k}_2\right)D\left(k_1, k_2, \widetilde{k}_2\right) + P^*\left(k_1^\prime,k_1, k_2, \widetilde{k}_2\right)D\left(k_1^\prime, k_2, \widetilde{k}_2\right)\right){\Bigg )}
\end{align}
where
\[
P(k_1, k_1^\prime, k_2, \widetilde{k}_2) = i\bra{k_1^\prime, k_2} \op{\rho}\left(-\infty\right) \ket{k_1 + \widetilde{k}_2 - k_2,\; \widetilde{k}_2}.
\]



\end{document}