We have derived expressions for the change in Von Neumann entropy due to a perturbation, including two beautiful and new expressions (\ref{eq.deltaS.noassumptions}) and (\ref{eq.Sredresult.separable}). We have seen that the leading-order perturbative correction to entropy is non-negative if the density operator's eigenvalue corrections for states in the initial density operator's kernel are of similar or lower order as for states not in the kernel. In particular, entropy will not decrease if we start in a diagonally separable state such that some states in the accessible space have zero initial probability but can be transitioned into. In this case, the entropy growth is dependent on the transition amplitudes to these states. We also applied our formulae in a few examples.

Future work could examine whether it is possible to use the density of accessible kernel states to predict the rate at which systems become entangled. In particular, a system’s energy spectrum and selection rules might dictate the rate of decoherence.

In Chapter \ref{ch.intro}, I hinted at a probability-theory formalism of quantum mechanics based on the many worlds interpretation. This should be formalized---doing so would give insight into the meaning of the reduced density operator.

Concerning Chapter \ref{ch.derivation}, it is possible that the first order-probability corrections \(\overline{\sigma_n^{(1)}}\) vanish if and only if the initial state is diagonally separable. I have proven the ``if'' direction. The ``only if'' direction should also be proven or falsified by counterexample. It would also be interesting to explore the entropy evolution of states that are separable but not diagonally separable. One might also study whether a perturbation always leads to an increase in entropy when new states can be entered, even if the systems is not separable to begin with. Furthermore, I considered asymptotic expansions in the limit \(\lambda \to 0\). It would be interesting to also consider expansions in the limits \(\left(t-t_0\right) \to 0\) or \(\lambda\left(t-t_0\right) \to 0\).

One could also investigate applications of entropy. For example, how useful is Von Neumann entropy for quantifying decoherence in quantum computing, and can the formulae derived here be used in such applications? 

The approach here could also be replicated for other kinds of entropy, for example, \(N\)-Tsallis entropy. Ref. \cite{cheung} shows that \(N\)-Tsallis entropy can decrease when one starts with diagonally separable states. We could ask: do these cases where \(N\)-Tsallis entropy decreases have lower-order corrections in the initial state space than in the zero-initial-probability space? If not, my results would show Von Neumann entropy to increase for the system, giving rise to an important difference between Von Neumann and \(N\)-Tsallis entropy.

Finally, physicists should continue to investigate how quantum entropy might give rise to thermodynamical properties. Perhaps one could derive all of thermodynamics starting with only the Schrödinger equation. Ref. \cite{bracken} and \cite{heusler} begin this work. It should be continued.
