\section{The general case}

Suppose we have a Hilbert space of the form \(\hilb = \hilb_A \otimes \hilb_B\). The \(\hilb_A\)-reduced density operator of a density operator \(\op{\rho}\) is defined to be
\[\overline{\op{\rho}} = \sum_m \left(\id\otimes\bra{\beta_m}\right) \op{\rho} \left(\id\otimes\ket{\beta_m}\right)\]
where \(\{\ket{\beta_m}\}\) is any orthonormal basis for \(\hilb_B\) and \(\id\) is the identity operator. (One can verify that all choices of basis give the same \(\overline{\op{\rho}}\).)
We can consider \(\overline{\op{\rho}}\) to act on the Hilbert space \(\hilb_A\).

Define the \(\hilb_A\)-subsystem entropy to be \(S = -\Tr{\overline{\op{\rho}} \log \overline{\op{\rho}}}\). If \(\overline{\op{\rho}}\) is diagonalized like
\begin{equation}\label{eq.redrhodef}
\overline{\op{\rho}} = \sum_n \overline{\sigma_n} \ket{\alpha_n}\bra{\alpha_n},
\end{equation}
where \(\{\ket{\alpha_n}\}\) is an orthonormal basis of \(\hilb_A\), then
\begin{equation}\label{eq.entropydefinition}
S = -\sum_n \overline{\sigma_n} \log \overline{\sigma_n}.
\end{equation}

Now, looking at equation (\ref{eq.rhoevol}), we see that at some fixed time \(t\) we can form asymptotic expansions
\begin{align*}
\overline{\op{\rho}}(t) &= \overline{\op{\rho}}_0 + \lambda \overline{\op{\rho}}_1 + \lambda^2 \overline{\op{\rho}}_2 + \Od{\lambda^3} \\
\ket{\alpha_n} &= \ket{\alpha_n^{(0)}} + \lambda \ket{\alpha_n^{(1)}} + \lambda^2 \ket{\alpha_n^{(2)}} + \Od{\lambda^3} \\
\overline{\sigma_n} &= \overline{\sigma_n^{(0)}} + \lambda \overline{\sigma_n^{(1)}} + \lambda^2 \overline{\sigma_n^{(2)}} + \Od{\lambda^3}
\end{align*}
by taking
\begin{align}
\nonumber\op{\rho}_0 &= \op{\rho}\left(t_0\right) \\
\nonumber\overline{\op{\rho}}_0 &= \overline{\op{\rho}}\left(t_0\right) =  \sum_m \left(\id\otimes\bra{\beta_m}\right) \op{\rho}_0 \left(\id\otimes\ket{\beta_m}\right) \\
\label{eq.rho1red}\overline{\op{\rho}}_1 &= -\frac{i}{\hbar} \int_{t_0}^t \sum_m \left(\id\otimes\bra{\beta_m}\right) \left[\op{H}_I(t^\prime),\; \op{\rho}_0 \right]\left(\id\otimes\ket{\beta_m}\right) \,dt^\prime\\
\label{eq.rho2red}\overline{\op{\rho}}_2 &= - \frac{1}{\hbar^2} \int_{t_0}^t \int_{t_0}^{t^\prime} \sum_m \left(\id\otimes\bra{\beta_m}\right) \left( \op{H}_I (t^\prime) \op{H}_I(t^{\prime\prime}) \;\op{\rho}_0 + \op{\rho}_0\; \op{H}_I (t^{\prime\prime}) \op{H}_I(t^{\prime}) \right)\left(\id\otimes\ket{\beta_m}\right)\,dt^{\prime\prime}\,dt^\prime \\
\nonumber&\hspace{2em}+ \frac{1}{\hbar^2} \int_{t_0}^t \int_{t_0}^{t} \sum_m \left(\id\otimes\bra{\beta_m}\right) \op{H}_I (t^\prime) \,\op{\rho}_0\, \op{H}_I (t^{\prime\prime})\left(\id\otimes\ket{\beta_m}\right)\,dt^{\prime\prime}\,dt^\prime.
\end{align}
From (\ref{eq.redrhodef}) we see that
\begin{equation}\label{eq.rho0red}
\overline{\op{\rho}}_0 = \sum_n \overline{\sigma_{n}^{(0)}} \ket{\alpha_n^{(0)}}\bra{\alpha_n^{(0)}}.
\end{equation}
Therefore we can take \(\left\{\ket{\alpha_n^{(0)}}\right\}\) to be an orthonormal basis for \(\hilb_A\) and \(\sum_n \overline{\sigma_n^{(0)}} = 1\).

It will be useful to work in the basis \(\left\{\ket{\alpha_n^{(0)}}\otimes\ket{\beta_m}\right\}\) of \(\hilb\). To simplify notation, let
\[
\bket{n}{m} = \ket{\alpha_n^{(0)}}\otimes\ket{\beta_m}.
\]
Also let
\[
I = \left\{n : \overline{\sigma_n^{(0)}} \neq 0\right\}.
\]

We can determine \(\overline{\sigma_{n}^{(1)}}\) and \(\overline{\sigma_{n}^{(2)}}\) by combining (\ref{eq.tipt.foval}), (\ref{eq.tipt.soval}), (\ref{eq.rho1red}), and (\ref{eq.rho2red}). We get
\begin{equation}\label{eq.foprobcor}
\overline{\sigma_n^{(1)}} = -\frac{i}{\hbar} \int_{t_0}^t \sum_m \bbra{n}{m} \left[\op{H}_I(t^\prime),\; \op{\rho}_0 \right]\bket{n}{m} \,dt^\prime
\end{equation}
and
\begin{align}
\overline{\sigma_n^{(2)}} =
\sum_m\int_{t_0}^t &\int_{t_0}^{t^\prime} \bbra{n}{m} \left( \op{H}_I (t^\prime) \op{H}_I(t^{\prime\prime}) \;\op{\rho}_0 + \op{\rho}_0\; \op{H}_I (t^{\prime\prime}) \op{H}_I(t^{\prime}) \right) \bket{n}{m}\,dt^{\prime\prime}\,dt^\prime \nonumber\\
&- \sum_m\int_{t_0}^t \int_{t_0}^{t} \bbra{n}{m}\op{H}_I (t^\prime) \,\op{\rho}_0\, \op{H}_I (t^{\prime\prime})\bket{n}{m}\,dt^{\prime\prime}\,dt^\prime \nonumber\\
&- \sum_{n^\prime \in C_n} \frac{1}{\;\overline{\sigma_n^{(0)}}-\overline{\sigma_{n^\prime}^{(0)}}\;} \left|\sum_m\int_{t_0}^t \bbra{n}{m} \left[\op{H}_I(t^\prime),\; \op{\rho}_0 \right] \bket{n^\prime}{m}\;dt^\prime \right|^2
\label{eq.soprobcor}
\end{align}
where \(C_n = \left\{n^\prime : \overline{\sigma_{n^\prime}^{(0)}} = \overline{\sigma_{n}^{(0)}}\right\}\).

We now compute the subsystem entropy at time \(t\). Let us examine the terms in (\ref{eq.entropydefinition}).
If \(\sigma_{n}^{(0)} \neq 0\) (i.e. \(n \in I\)), let \(l_n\) be the lowest positive integer such that \(\overline{\sigma_n^{(l_n)}} \neq 0\). Let \(l\) be the minimal \(l_n\). (If all corrections are zero, we could say \(l\) doesn't exist, but this situation is uninteresting because the system just stays in the initial state. We may as well assume \(l\) exists.) We see that
\begin{align*}
\overline{\sigma_n} \log \overline{\sigma_n}
&= \left(\overline{\sigma_n^{(0)}} + \lambda^{l_n} \overline{\sigma_n^{(l_n)}} + \Od{\lambda^{l_n+1}}\right) \log \left(\overline{\sigma_n^{(0)}} \left(1 + \frac{\lambda^{l_n} \overline{\sigma_n^{(l_n)}}}{\overline{\sigma_n^{(0)}}}+ \Od{\lambda^{l_n+1}}\right)\right) \\
&= \left(\overline{\sigma_n^{(0)}} + \lambda^{l_n} \overline{\sigma_n^{(l_n)}}\right) \left(\log \overline{\sigma_n^{(0)}} + \frac{\lambda^{l_n} \overline{\sigma_n^{(l_n)}}}{\overline{\sigma_n^{(0)}}}\right) + \Od{\lambda^{l_n+1}}\\
&= \left(\overline{\sigma_n^{(0)}} + \lambda^{l} \overline{\sigma_n^{(l)}}\right) \left(\log \overline{\sigma_n^{(0)}} + \frac{\lambda^{l} \overline{\sigma_n^{(l)}}}{\overline{\sigma_n^{(0)}}}\right) + \Od{\lambda^{l+1}}\\
&=  \overline{\sigma_n^{(0)}} \log \overline{\sigma_n^{(0)}} + \lambda^l\, \overline{\sigma_n^{(l)}} + \lambda^l\, \overline{\sigma_n^{(l)}}\log \frac{1}{\overline{\sigma_n^{(0)}}} + \Od{\lambda^{l+1}}.
\end{align*}
On the other hand, if \(n\) has \(\sigma_{n}^{(0)} = 0\) (i.e. \(n \not\in I\)) and \(\overline{\sigma_n}\neq 0\), let \(k_n\) be the lowest positive integer such that \(\overline{\sigma_n^{(k_n)}} \neq 0\). Let \(k\) be the minimal \(k_n\). Then
\begin{align*}
\overline{\sigma_n} \log \overline{\sigma_n}
&= \left(\lambda^{k_n}\overline{\sigma_n^{(k_n)}} + \Od{\lambda^{k_n+1}}\right) \log\left(\lambda^{k_n}\overline{\sigma_n^{(k_n)}}\left(1 + \Od{\lambda}\right)\right) \\
&= \lambda^{k_n}\overline{\sigma_n^{(k_n)}} \log\left(\lambda^{k_n}\overline{\sigma_n^{(k_n)}}\right)+ \Od{\lambda^{k_n+1}}\\
&= \lambda^{k}\overline{\sigma_n^{(k)}} \log\left(\lambda^{k}\overline{\sigma_n^{(k)}}\right)+ \Od{\lambda^{k+1}} \\
&= -\left(\lambda^{k}\log\frac{1}{\lambda^k}\right)\overline{\sigma_n^{(k)}} + \Od{\lambda^{k}}.
\end{align*}
Putting this all into (\ref{eq.entropydefinition}),
\begin{align*}
\hspace{-1em} S = -\sum_{n \in I} \overline{\sigma_n^{(0)}} &\log \overline{\sigma_n^{(0)}} - \lambda^l\, \sum_{n \in I}\overline{\sigma_n^{(l)}}+\lambda^l\, \sum_{n \in I}\overline{\sigma_n^{(l)}}\log\frac{1}{\overline{\sigma_n^{(0)}}} \\&+\left(\lambda^{k}\log\frac{1}{\lambda^k}\right)\sum_{n\not\in I} \overline{\sigma_n^{(k)}} + \Od{\lambda^{l+1}} + \Od{\lambda^{k}}.
\end{align*}
The first term, \(- \sum_{n \in I} \overline{\sigma_n^{(0)}} \log \overline{\sigma_n^{(0)}}\), would be the entropy if \(\lambda=0\), that is, with no perturbation. Also, because \(\sum_n\overline{\sigma_n} = 1\) we must have \(\sum_{n\in I}\overline{\sigma_n^{(l)}} = -\sum_{n\not\in I} \overline{\sigma_n^{(l)}}\). If \(l < k\) this vanishes; if \(l \geq k\) it can be absorbed into the \(\Od{\lambda^k}\). Thus the change in entropy due to the perturbation is
\begin{equation}\label{eq.deltaS.noassumptions}
\boxed{\Delta S = \lambda^l\, \sum_{n \in I}\overline{\sigma_n^{(l)}}\log\frac{1}{\overline{\sigma_n^{(0)}}} +\left(\lambda^{k}\log\frac{1}{\lambda^k}\right)\sum_{n\not\in I} \overline{\sigma_n^{(k)}} + \Od{\lambda^{l+1}} + \Od{\lambda^{k}}.}
\end{equation}
Here \(l\) is the order of the lowest non-vanishing correction \(\overline{\sigma_n^{(l)}}\) for \(n \in I\) and \(k\) is the order of the lowest non-vanishing correction \(\overline{\sigma_n^{(k)}}\) for \(n \not\in I\). In general, \(l\) and \(k\) will be \(1\) or \(2\). We can find them and compute the eigenvalue corrections with (\ref{eq.foprobcor}) and (\ref{eq.soprobcor}).

Now, what does \(\Delta S\) mean? Recall our Hamiltonian \(\op{H} = \op{H}_0 + \lambda \op{V}\).
If \(\op{H}_0 = \op{H}_0^A + \op{H}_0^B\), where \(\op{H}_0^A\) acts only on \(\hilb_A\) and \(\op{H}_0^B\) acts only on \(\hilb_B\), then \(\op{H}_0\) does not cause mixing between the two subsystems. In this case \(\Delta S\) is the change in entropy from time \(t_0\) to time \(t\).
Conversely, if \(\op{H}_0\) is not of this form, \(\op{H}_0\) might cause mixing between subsystems. Then there could be some zeroth-order change in entropy due to \(\op{H}_0\). In this case \(\Delta S\) is not the \emph{total} change in entropy from time \(t_0\) to time \(t\) but rather the change in entropy from time \(t_0\) to time \(t\) due to the perturbation \(\lambda \op{V}\).

(\ref{eq.deltaS.noassumptions}) is a rather beautiful result. The \(\lambda^l\) term is the leading correction for entropy change within the subspace of \(\hilb_A\) that the system already occupied. The \(\lambda^k \log \frac{1}{\lambda^k}\) term is the leading correction for entropy generated due to transitioning to new states.

Because all \(\overline{\sigma_n} > 0\), the leading correction to \(\overline{\sigma_n}\) must be nonnegative for \(n \in I\). Therefore, if \(k \leq l\), the second term dominates and the change in entropy is non-negative. {\bf The perturbation may only cause entropy to decrease if \(l < k\), that is, if there are lower-order corrections in the occupied space (\(n \in I\)) than in the kernel of \(\overline{\op{\rho}}_0\) (\(n \not\in I\)).} In particular, if all states have the same order of correction that leads (often true), and if there exist states with zero initial probability that the system can transition to (also often true), then the entropy can only increase.

We will find in Section \ref{sec.separablestate} that \(l=k=2\) for separable states. Therefore, if a separable state can transition into new states that initially had probability zero, the entropy will never decrease due to the perturbation. This conclusion is consistent with Ref. \cite{bracken} which proves that the Von Neumann entropy of separable states does not decrease.

\section{Case where first-order corrections vanish and new states can be entered}\label{sec.secoddom}

Suppose that some \(n\) have \(\overline{\sigma_n^{(0)}} = 0\) and \(\overline{\sigma_n} \neq 0\). Suppose also that all \(\sigma_n^{(1)} = 0\). (In other words, we suppose that \(l=k=2\) and that there are accessible states in the kernel of \(\overline{\op{\rho}}_0\).) Then  (\ref{eq.deltaS.noassumptions}) becomes
\begin{equation}\label{eq.deltaS.notinI}
\Delta S = \left(\lambda^2\log\frac{1}{\lambda^{2}} \right)\sum_{n \not\in I} \overline{\sigma_n^{(2)}} + \Od{\lambda^2}.
\end{equation}
Now, since \(\sum_{n} \overline{\sigma_n^{(2)}}= 0\) we can rewrite (\ref{eq.deltaS.notinI}) as
\begin{equation}\label{eq.deltaS}
\Delta S = -\left(\lambda^2\log\frac{1}{\lambda^{2}} \right)\sum_{n \in I} \overline{\sigma_n^{(2)}} + \Od{\lambda^2}.
\end{equation}
Substituting in (\ref{eq.tipt.soval}),
\[
\Delta S = -\left(\lambda^2 \log\frac{1}{\lambda^2}\right)\left(\sum_{n \in I} \bra{\alpha_n^{(0)}}\overline{\op{\rho}_2}\ket{\alpha_n^{(0)}} +  \sum_{n \in I} \sum_{{n^\prime} \not\in D_n} \frac{\left|\bra{\alpha_{n^\prime}^{(0)}} \overline{\op{\rho}_1} \ket{\alpha_n^{(0)}}\right|^2}{\overline{\sigma_n^{(0)}}-\overline{\sigma_{n^\prime}^{(0)}}} \right)+ \Od{\lambda^2}.
\]
In the second term with the double sum, for every term \(\frac{\left|\bra{\alpha_{n^\prime}^{(0)}} \overline{\op{\rho}_1} \ket{\alpha_n^{(0)}}\right|^2}{\overline{\sigma_n^{(0)}}-\overline{\sigma_{n^\prime}^{(0)}}}\) with \(n^\prime \in I\) there is an equal but oppositely signed term \(\frac{\left|\bra{\alpha_{n^\prime}^{(0)}} \overline{\op{\rho}_1} \ket{\alpha_n^{(0)}}\right|^2}{\overline{\sigma_{n^\prime}^{(0)}}-\overline{\sigma_n^{(0)}}}\) in the sum. The portion of the sum over \(n^\prime \in I\) evaluates to zero. We end up with
\begin{equation}\label{eq.deltaSso.intermediate}
\Delta S = -\left(\lambda^2 \log\frac{1}{\lambda^2}\right)\left(\sum_{n \in I} \bra{\alpha_n^{(0)}}\overline{\op{\rho}_2}\ket{\alpha_n^{(0)}} +  \sum_{n \in I} \sum_{n^\prime \not\in I} \frac{\left|\bra{\alpha_{n}^{(0)}} \overline{\op{\rho}_1} \ket{\alpha_{n^\prime}^{(0)}}\right|^2}{\overline{\sigma_n^{(0)}}} \right)+ \Od{\lambda^2}.
\end{equation}

Putting in (\ref{eq.rho1red}) and (\ref{eq.rho2red}),
\begin{align}\label{eq.entropyresultwithrho2}
\Delta S = \frac{1}{\hbar^2}&\left(\lambda^2 \log\frac{1}{\lambda^2}\right)\sum_{n \in I} \Bigg( \\
& \sum_m\int_{t_0}^t \int_{t_0}^{t^\prime} \bbra{n}{m} \left( \op{H}_I (t^\prime) \op{H}_I(t^{\prime\prime}) \;\op{\rho}_0 + \op{\rho}_0\; \op{H}_I (t^{\prime\prime}) \op{H}_I(t^{\prime}) \right) \bket{n}{m}\,dt^{\prime\prime}\,dt^\prime \nonumber\\
-& \sum_m\int_{t_0}^t \int_{t_0}^{t} \bbra{n}{m}\op{H}_I (t^\prime) \,\op{\rho}_0\, \op{H}_I (t^{\prime\prime})\bket{n}{m}\,dt^{\prime\prime}\,dt^\prime \nonumber\\
-& \sum_{n^\prime \not \in I} \frac{1}{\;\overline{\sigma_n^{(0)}}\;} \left|\sum_m\int_{t_0}^t \bbra{n}{m} \left[\op{H}_I(t^\prime),\; \op{\rho}_0 \right] \bket{n^\prime}{m}\;dt^\prime \right|^2
\Bigg) + \Od{\lambda^2}. \nonumber
\end{align}
This is as far as we will go without making further assumptions. We can, however, relax our requirements about what \(\bket{n}{m}\) are. Normally, perturbation theory requires us to choose \(\ket{\alpha_n^{(0)}}\) so that the leading eigenvalue correction matrices are diagonal on degenerate subspaces. We see, however, that we can write \ref{eq.entropyresultwithrho2} as 
\[
\Delta S = \frac{1}{\hbar^2}\left(\lambda^2 \log\frac{1}{\lambda^2}\right)\Tr{M_{n,\tilde{n}}} + \Od{\lambda^2}
\]
where we take \(n,\tilde{n} \in I\) and
\begin{align*}
M_{n,\tilde{n}} = &\sum_m\int_{t_0}^t \int_{t_0}^{t^\prime} \bbra{n}{m} \left( \op{H}_I (t^\prime) \op{H}_I(t^{\prime\prime}) \;\op{\rho}_0 + \op{\rho}_0\; \op{H}_I (t^{\prime\prime}) \op{H}_I(t^{\prime}) \right) \bket{\tilde{n}}{m}\,dt^{\prime\prime}\,dt^\prime \nonumber\\
&- \sum_m\int_{t_0}^t \int_{t_0}^{t} \bbra{n}{m}\op{H}_I (t^\prime) \,\op{\rho}_0\, \op{H}_I (t^{\prime\prime})\bket{\tilde{n}}{m}\,dt^{\prime\prime}\,dt^\prime \nonumber\\
&- \sum_{n^\prime \not \in I} \frac{1}{\;\overline{\sigma_n^{(0)}}\;} \sum_m\int_{t_0}^t \bbra{n}{m} \left[\op{H}_I(t^\prime),\; \op{\rho}_0 \right] \bket{n^\prime}{m}\;dt^\prime \sum_{m^\prime}\int_{t_0}^t \bbra{n^\prime}{m^\prime} \left[\op{H}_I(t^\prime),\; \op{\rho}_0 \right] \bket{\tilde{n}}{m^\prime}\;dt^\prime.
\end{align*}
Because the trace of a matrix is invariant under a change of basis, we do not need to use the perturbation-diagonalized basis.

\section{Case of diagonally separable initial state where new states can be entered}\label{sec.separablestate}

We begin with a definition.

\begin{definition}\label{def.diagsep}
A \emph{diagonally separable state} on a product Hilbert space \(\hilb_A \otimes \hilb_B\) is a state with density operator
\begin{equation}\label{eq.defdiagsep}
\op{\rho} = \sum_{n,m} \sigma_{n,m} \left(\ket{\Psi_n^A}\otimes\ket{\Psi_m^B}\right)\left(\bra{\Psi_n^A}\otimes\bra{\Psi_m^B}\right)
\end{equation}
where \(\{\ket{\Psi_n^A}\}\) is some orthonormal basis for \(\hilb_A\), \(\{\ket{\Psi_m^B}\}\) is some orthonormal basis for \(\hilb_B\), and \(\sigma_{n,m}\in[0,1]\) with \(\sum_{n,m}\sigma_{n,m} =1\).
\end{definition}
It turns out that diagonally separable states will have entropy that evolves in a characteristic way.

From the definition, it is immediately clear that product states \(\op{\rho}_A\otimes\op{\rho}_B\) are diagonally separable states. (Indeed, product states are the diagonally separable states where \(\sigma_{n,m} = \sigma_n^A\sigma_m^B\) for some \(\sigma_n^A\) and \(\sigma_m^B\).) It is also clear that diagonally separable states are separable states. (Separable states have density operators like \(\sum_{n,m} \sigma_{n,m} \op{\rho}_n^A\otimes\op{\rho}_m^B\) where the \(\op{\rho}_n^A\) and \(\op{\rho}_m^B\) are themselves density operators on \(\hilb_A\) and \(\hilb_B\).) The converses are not true; there exist separable states that are not diagonally separable and there exist diagonally separable states that are not product states.

We now return to our study of entropy. We consider the case where, at the time \(t_0\), the state is diagonally separable; write
\begin{equation}\label{eq.rho0fullinitial}
\op{\rho}_0 = \op{\rho}\left(t_0\right) = \sum_{n,m} \sigma_{n,m}^{(0)} \left(\ket{\Psi_n^{A}}\otimes\ket{\Psi_m^B}\right) \left(\bra{\Psi_n^{A}}\otimes\bra{\Psi_m^B}\right)
\end{equation}
where \(\left\{\ket{\Psi_n^{A}}\right\}\) is an orthonormal basis for \(\hilb_A\) and \(\left\{\ket{\Psi_m^B}\right\}\) is an orthonormal basis for \(\hilb_B\).

(\ref{eq.rho0fullinitial}) implies that the reduced density operator at time \(t_0\) is
\[
\overline{\op{\rho}}_0 = \overline{\op{\rho}}\left(t_0\right) = \sum_n \overline{\sigma_{n}^{(0)}} \ket{\Psi_n^{A}}\bra{\Psi_n^{A}}
\]
where
\[
\overline{\sigma_n^{(0)}} = \sum_m \sigma_{n,m}^{(0)}.
\]
Now, if we fix a different time \(t\), there should be some basis \(\{\ket{\alpha_n}\}\) of \(\hilb_A\) and some quantities \(\overline{\sigma_n}\) such that
\[
\overline{\op{\rho}}(t) = \sum_n \overline{\sigma_n} \ket{\alpha_n}\bra{\alpha_n}.
\]
The eigenstate/eigenvalue decomposition of an operator is unique up to relabeling once we choose a basis that is appropriately diagonalized on degenerate spaces.
Comparing our equations to (\ref{eq.redrhodef}) and (\ref{eq.rho0red}), we see that our previous analysis must hold with
\begin{align*}
\ket{\alpha_n^{(0)}} &= \ket{\Psi_n^A} &
\ket{\beta_m} &= \ket{\Psi_m^B} & 
\bket{n}{m} &= \ket{\alpha_n^{(0)}} \otimes \ket{\beta_m} = \ket{\Psi_n^A}\otimes \ket{\Psi_m^B}.
\end{align*}
Then
\begin{equation}
\label{eq.rho0full}
\op{\rho}_0 = \sum_{n,m} \sigma_{n,m}^{(0)} \bket{n}{m}\bbra{n}{m}.
\end{equation}

We have
\begin{align*}
\bbra{n}{m} \op{H}_I(t^\prime) \;\op{\rho}_0 \bket{n}{m} 
&= 
\sigma_{n,m}^{(0)} \bbra{n}{m} \op{H}_I(t^\prime) \bket{n}{m} 
\\&=
 \bbra{n}{m} \op{\rho}_0 \;\op{H}_I(t^\prime) \bket{n}{m} 
\end{align*}
so the first-order eigenvalue correction (\ref{eq.foprobcor}) vanishes. We will assume that the second-order eigenvalue corrections \(\overline{\sigma_n^{(2)}}\) do not vanish and also that there exist states in \(\hilb_A\) with zero initial probability that can be transitioned into. Therefore we are in the case of Section \ref{sec.secoddom} and use (\ref{eq.entropyresultwithrho2}). (We don't need to worry about diagonalizing on degenerate subspaces for the same reasons as in Section \ref{sec.secoddom}.)

The first two terms in (\ref{eq.entropyresultwithrho2}) are
\begin{align*}
&\hspace{-4em} \sum_{n \in I} \sum_m\int_{t_0}^t \int_{t_0}^{t^\prime} \bbra{n}{m} \left( \op{H}_I (t^\prime) \op{H}_I(t^{\prime\prime}) \;\op{\rho}_0 + \op{\rho}_0\; \op{H}_I (t^{\prime\prime}) \op{H}_I(t^{\prime}) \right) \bket{n}{m}\,dt^{\prime\prime}\,dt^\prime \\
&\hspace{-4em}- \sum_{n \in I}\sum_m\int_{t_0}^t \int_{t_0}^{t} \bbra{n}{m}\op{H}_I (t^\prime) \,\op{\rho}_0\, \op{H}_I (t^{\prime\prime})\bket{n}{m}\,dt^{\prime\prime}\,dt^\prime \\
&= \sum_{n \in I}\sum_m\sigma_{n,m}^{(0)} \int_{t_0}^t \int_{t_0}^{t^\prime} \bbra{n}{m} \left\{\op{H}_I (t^\prime),\; \op{H}_I(t^{\prime\prime}) \right\} \bket{n}{m}\,dt^{\prime\prime}\,dt^\prime \\
&\hspace{2em} -\sum_m\sum_{n^\star} \sum_{m^\star} \sigma_{n^\star,m^\star}^{(0)}\int_{t_0}^t \int_{t_0}^{t} \bbra{n}{m}\op{H}_I (t^\prime) \bket{n^\star}{m^\star}\bbra{n^\star}{m^\star} \op{H}_I (t^{\prime\prime})\bket{n}{m}\,dt^{\prime\prime}\,dt^\prime \\
&=\sum_{n \in I}\sum_m \sigma_{n,m}^{(0)} \int_{t_0}^t \int_{t_0}^{t} \bbra{n}{m}\op{H}_I (t^\prime) \op{H}_I(t^{\prime\prime}) \bket{n}{m}\,dt^{\prime\prime}\,dt^\prime \\
&\hspace{2em}-\sum_{n \in I}\sum_m\sum_{n^\star} \sum_{m^\star} \sigma_{n^\star,m^\star}^{(0)}\int_{t_0}^t \int_{t_0}^{t} \bbra{n^\star}{m^\star} \op{H}_I (t^{\prime\prime})\bket{n}{m} \bbra{n}{m}\op{H}_I (t^\prime) \bket{n^\star}{m^\star}\,dt^{\prime\prime}\,dt^\prime \\
&= \sum_{n \in I}\sum_m\sigma_{n,m}^{(0)} \int_{t_0}^t \int_{t_0}^{t} \bbra{n}{m}\op{H}_I (t^\prime) \op{H}_I(t^{\prime\prime}) \bket{n}{m}\,dt^{\prime\prime}\,dt^\prime \\
&\hspace{2em}-\sum_{n^\star} \sum_{m^\star} \sigma_{n^\star,m^\star}^{(0)}\int_{t_0}^t \int_{t_0}^{t} \bbra{n^\star}{m^\star} \op{H}_I (t^{\prime\prime})\left[\left(\sum_{n\in I} \ket{\alpha_n^{(0)}}\bra{\alpha_n^{(0)}}\right)\otimes\id\right]\op{H}_I (t^\prime) \bket{n^\star}{m^\star}\,dt^{\prime\prime}\,dt^\prime \\
&=\sum_{n \in I}\sum_m \sigma_{n,m}^{(0)} \int_{t_0}^t \int_{t_0}^{t} \bbra{n}{m}\op{H}_I (t^\prime) \op{H}_I(t^{\prime\prime}) \bket{n}{m}\,dt^{\prime\prime}\,dt^\prime \\
&\hspace{2em}-\sum_{n} \sum_{m} \sigma_{n,m}^{(0)}\int_{t_0}^t \int_{t_0}^{t} \bbra{n}{m} \op{H}_I (t^{\prime\prime})\left[\left(1-\sum_{n^\prime\not\in I} \ket{\alpha_{n^\prime}^{(0)}}\bra{\alpha_{n^\prime}^{(0)}}\right)\otimes\id\right]\op{H}_I (t^\prime) \bket{n}{m}\,dt^{\prime\prime}\,dt^\prime \\
&= \sum_{n \in I}\sum_m \sigma_{n,m}^{(0)} \left[\int_{t_0}^t \int_{t_0}^{t} \bbra{n}{m}\op{H}_I (t^\prime) \op{H}_I(t^{\prime\prime}) \bket{n}{m}\,dt^{\prime\prime}\,dt^\prime - \int_{t_0}^t \int_{t_0}^{t} \bbra{n}{m}\op{H}_I (t^\prime) \op{H}_I(t^{\prime\prime}) \bket{n}{m}\,dt^{\prime\prime}\,dt^\prime\right]\\
&\hspace{2em}+ \sum_{n} \sum_m \sigma_{n,m}^{(0)} \sum_{n^\prime \not\in I} \sum_{m^\prime} \int_{t_0}^t \bbra{n}{m} \op{H}_I (t^{\prime\prime})\bket{n^\prime}{m^\prime} \,dt^{\prime\prime}\int_{t_0}^t \bbra{n^\prime}{m^\prime} \op{H}_I (t^{\prime})\bket{n}{m}\,dt^\prime \\
&=\sum_{n \in I} \sum_{n^\prime \not\in I} \sum_m \sigma_{n,m}^{(0)} 
 \sum_{m^\prime} \left|\int_{t_0}^t \bbra{n}{m} \op{H}_I (t^{\prime})\bket{n^\prime}{m^\prime} \,dt^{\prime}\right|^2.
\end{align*}

Meanwhile, the third term in (\ref{eq.entropyresultwithrho2}) is
\begin{align*}
-\sum_{n \in I}\sum_{n^\prime \not \in I} &\frac{1}{\;\overline{\sigma_n^{(0)}}\;} \left|\sum_m\int_{t_0}^t \bbra{n}{m} \left[\op{H}_I(t^\prime),\; \op{\rho}_0 \right] \bket{n^\prime}{m}\;dt^\prime \right|^2 \\
&= -\sum_{n \in I}\sum_{n^\prime \not \in I} \frac{1}{\;\overline{\sigma_n^{(0)}}\;} \left|\sum_m\int_{t_0}^t \bbra{n}{m} \op{H}_I(t^\prime) 
 \bket{n^\prime}{m}\;\left({\sigma_{n,m}^{(0)}} - \cancel{{\sigma_{n^\prime,m}^{(0)}}}\right)\;dt^\prime \right|^2\Bigg) \\
 &= -\sum_{n \in I}\sum_{n^\prime \not \in I} \frac{1}{\sum_{\tilde{m}} \sigma_{n,\tilde{m}}^{(0)}} \left|\sum_m{\sigma_{n,m}^{(0)}}\int_{t_0}^t \bbra{n}{m} \op{H}_I(t^\prime) 
 \bket{n^\prime}{m}\;dt^\prime \right|^2.
\end{align*}

Putting this all back into (\ref{eq.entropyresultwithrho2}), we obtain the rather interesting result that
\begin{empheq}[box=\fbox]{align}\label{eq.Sredresult.separable}
\Delta S =\frac{1}{\hbar^2}\left(\lambda^2 \log\frac{1}{\lambda^2}\right)& \sum_{n\in I}\sum_{n^\prime \not\in I} \Bigg( \sum_m \sigma_{n,m}^{(0)} \sum_{m^\prime} \left|\int_{t_0}^t \bbra{n}{m}\op{H}_I (t^{\prime})\bket{n^\prime}{m^\prime} \,dt^{\prime}\right|^2 \\
&- \frac{1}{\sum_{\tilde{m}} \sigma_{n,\tilde{m}}^{(0)}} \left|\sum_m{\sigma_{n,m}^{(0)}}\int_{t_0}^t \bbra{n}{m} \op{H}_I(t^\prime) 
 \bket{n^\prime}{m}\;dt^\prime \right|^2\Bigg) + \Od{\lambda^2}. \nonumber
\end{empheq}
Entropy just depends on the Born transition amplitudes to states that started with zero initial probability.

In \(\ref{eq.Sredresult.separable}\), the first term \(\sum_{n\in I}\sum_{n^\prime \not\in I}\sum_m \sigma_{n,m}^{(0)} \sum_{m^\prime} \left|\int_{t_0}^t \bbra{n}{m}\op{H}_I (t^{\prime})\bket{n^\prime}{m^\prime} \,dt^{\prime}\right|^2\) can be interpreted as the total probability that a transition occurs to a state that had zero probability to start. The second term is harder to interpret; perhaps it is some kind of correction to account for double-counting.

We observe that \(\ref{eq.Sredresult.separable}\) vanishes if \(\hilb_B\) has only a single possible state. This situation is equivalent to taking the entropy of the whole system rather than the subsystem entropy. We have thus shown that the entropy of the whole system is conserved (at the order of \(\lambda^2 \log\frac{1}{\lambda^2}\), for separable states).

I am not aware of any others having derived (\ref{eq.Sredresult.separable}). I think it is a new result.


\section{Case of pure, separable initial state}\label{sec.purestate}

If the initial state is pure and separable, let \(\bket{n}{m}\) be the initial state so that \(\op{\rho}_0 = \bket{n}{m}\bbra{n}{m}\). Then (\ref{eq.Sredresult.separable}) simplifies to
\begin{equation}\label{eq.Sredresult.separablepure}
\boxed{\Delta S =\frac{1}{\hbar^2}\left(\lambda^2 \log\frac{1}{\lambda^2}\right) \sum_{n^\prime\neq n} \;\sum_{m^\prime\neq m} \left|\int_{t_0}^t \bbra{n}{m} \op{H}_I (t^{\prime})\bket{n^\prime}{m^\prime} \,dt^{\prime}\right|^2 + \Od{\lambda^2}.}
\end{equation}
We observe a crucial exclusion principle: in our expression for entropy, we only have transition amplitudes where the state changes in both subsystems, never transition amplitudes for a change in only one subsystem. Furthermore, since this expression is symmetric with respect to \(n\) and \(m\), the change in subsystem entropy for \(\hilb_A\) is the same as the change in subsystem entropy for \(\hilb_B\).

Equation (\ref{eq.Sredresult.separablepure}) replicates the result of Ref. \cite{seki}, though we derived it by a different method.
