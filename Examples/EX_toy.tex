In this section, we will study a simple bipartite system and calculate the growth in subsystem entropy due to a weak interaction. Consider the Hilbert space \(\{\ket{0}, \ket{1}\}\otimes\{\ket{0}, \ket{1}\}\) describing two qubits. To simplify notation, write \(\ket{ab}\) to mean \(\ket{a}\otimes\ket{b}\).

Suppose the qubits interact according to the Hamiltonian
\[
\op{H} = \op{S}_z^{A}\otimes\op{S}_z^B + \lambda \op{S}_x^A \otimes \op{S}_x^B.
\]
One can calculate that this Hamiltonian has eigenstates and eigenvalues
\begin{align*}
\ket{E_0} &= \frac{1}{\sqrt{2}} \left(\ket{10}-\ket{01}\right) &
E_0 &= \frac{\hbar}{2} \left(-1-\lambda\right) \\
\ket{E_1} &= \frac{1}{\sqrt{2}} \left(\ket{10}+\ket{01}\right) &
E_1 &= \frac{\hbar}{2} \left(-1+\lambda\right) \\
\ket{E_2} &= \frac{1}{\sqrt{2}} \left(\ket{11}-\ket{00}\right) &
E_2 &= \frac{\hbar}{2} \left(1-\lambda\right) \\
\ket{E_3} &= \frac{1}{\sqrt{2}} \left(\ket{11}+\ket{00}\right) &
E_3 &= \frac{\hbar}{2} \left(1+\lambda\right).
\end{align*}
Then we can write \(\op{H} = \op{H}_0 + \lambda \op{V}\) where
\begin{align*}
\op{H}_0 &= \frac{\hbar}{2} \left(-\ket{E_0}\bra{E_0} - \ket{E_1}\bra{E_1} + \ket{E_2}\bra{E_2} + \ket{E_3}\bra{E_3}\right) \\
\op{V} &= \frac{\hbar}{2} \left(-\ket{E_0}\bra{E_0} + \ket{E_1}\bra{E_1} - \ket{E_2}\bra{E_2} + \ket{E_3}\bra{E_3}\right).
\end{align*}
Because \([\op{H}_0, \op{V}] = 0\) we have \([\op{U}(0, t), \op{V}] = 0\). Thus the interaction picture Hamiltonian is just
\[\op{H}_I = \op{V} = \frac{\hbar}{2} \left(-\ket{E_0}\bra{E_0} + \ket{E_1}\bra{E_1} - \ket{E_2}\bra{E_2} + \ket{E_3}\bra{E_3}\right)\]
or, in our product basis,
\[
\op{H}_I = \frac{\hbar}{2} \left(\ket{11}\bra{00} + \ket{00}\bra{11} + \ket{10}\bra{01} + \ket{01}\bra{10}\right).
\]

Let's now consider two different initial states. In both cases, let's start by calculating the exact change in entropy; then we can calculate the perturbative result and compare.

\subsection{Initial state \(\ket{11}\)}

Suppose at \(t=0\) we start in the initial state \(\ket{11} = \frac{1}{\sqrt{2}}\left(\ket{E_2}+\ket{E_3}\right)\). At time \(t\), the state is then
\[
\ket{\Psi(t)} = \frac{1}{\sqrt{2}}\left(e^{-itE_2/\hbar}\ket{E_2} + e^{-itE_3/\hbar} \ket{E_3}\right)
\]
and the density operator is
\[
\op{\rho} = \frac{1}{2}\ket{E_2}\bra{E_2} + \frac{1}{2}\ket{E_3}\bra{E_3} + \frac{1}{2} e^{it\left(E_3-E_2\right)/\hbar} \ket{E_2}\bra{E_3} + \frac{1}{2} e^{-it\left(E_3 - E_2\right)/\hbar} \ket{E_3}\bra{E_2}.
\]
Then the reduced density operator for the first qubit is
\begin{align*}
\overline{\op{\rho}} &= \cdot\otimes\bra{0} \op{\rho} \cdot\otimes\ket{0} + \cdot\otimes\bra{1} \op{\rho} \cdot\otimes\ket{1} \\
&= \left(\frac{1}{2} + \frac{1}{4}e^{\left(E_3-E_2\right)\frac{t}{\hbar}} + \frac{1}{4}e^{-\left(E_3-E_2\right)\frac{t}{\hbar}} \right)\ket{1}\bra{1} + \left(\frac{1}{2} -\frac{1}{4}e^{\left(E_3-E_2\right)\frac{t}{\hbar}} -\frac{1}{4}e^{-\left(E_3-E_2\right)\frac{t}{\hbar}} \right)\ket{0}\bra{0} \\
% &= \frac{1}{2}\left(1 + \cos \left[\left(E_3-E_2\right)\frac{t}{\hbar}\right] \right)\ket{1}\bra{1} + \frac{1}{2}\left(1 -\cos \left[\left(E_3-E_2\right)\frac{t}{\hbar}\right] \right)\ket{0}\bra{0} \\
&= \frac{1}{2}\left(1 + \cos \lambda t \right)\ket{1}\bra{1} + \frac{1}{2}\left(1 -\cos \lambda t \right)\ket{0}\bra{0}.
\end{align*}
where we used that \(\frac{E_3-E_2}{\hbar} = \lambda\). Thus the Von Neumann subsystem entropy is
\begin{align*}
S &= -\frac{1}{2}\left(1 + \cos \lambda t \right)\log \left(\frac{1}{2}\left[1+\cos \lambda\right]\right)
- \frac{1}{2}\left(1 - \cos \lambda t \right)\log \left(\frac{1}{2}\left[1-\cos\lambda t\right]\right).
\end{align*}
Using \(\cos x = 1-\frac{1}{2} x^2 + \Od{x^4}\) we get
\begin{align*}
S &=- \left(1-\frac{1}{4}\lambda^2 t^2\right) \log \left(1-\frac{1}{4}\lambda^2 t^2\right) - \frac{1}{4}\lambda^2 t^2 \log \left(\frac{1}{4}\lambda^2 t^2\right) + \Od{\lambda^4 t^4}\\
&=\frac{1}{4}\lambda^2 t^2 - \frac{1}{4}\lambda^2 t^2 \log \left(\frac{1}{4}\lambda^2 t^2\right) + \Od{\lambda^4 t^4}\\
&= \left(\lambda^2 \log \frac{1}{\lambda^2}\right) \frac{1}{4} t^2 + \Od{\lambda^2}.
\end{align*}

We did the derivation above by solving the Schr\"odinger equation exactly. Let's now apply our perturbative calculation and see that we get the same result. 
The first-order corrections vanish because \(\op{\rho}_0\) commutes with \(\op{H}_I\), so we have \(l=k=2\). We therefore will use (\ref{eq.Sredresult.separablepure}).

The only transition amplitude in the sum (\ref{eq.Sredresult.separablepure}) will be \(\bra{11} \op{H}_I \ket{00}\). Indeed (\ref{eq.Sredresult.separablepure}) becomes
\begin{align*}
S &= \frac{1}{\hbar^2} \left(\lambda^2 \log \frac{1}{\lambda^2}\right) \left|\int_0^t \bra{11} \op{H}_I \ket{00}\,dt^\prime\right|^2 + \Od{\lambda^2} \\
&= \frac{1}{\hbar^2} \left(\lambda^2 \log \frac{1}{\lambda^2}\right) \left|\int_0^t \frac{\hbar}{2}\,dt^\prime\right|^2 + \Od{\lambda^2} \\
&= \left(\lambda^2 \log \frac{1}{\lambda^2}\right) \frac{1}{4} t^2
\end{align*}
which is the same as what we found in our exact calculation.

\subsection{Initial state \(\frac{1}{2}\left(\ket{10} + \ket{01} + \ket{11} - \ket{00}\right)\)}

Now suppose the system begins in the state
\[\frac{1}{2}\left(\ket{10} + \ket{01} + \ket{11} - \ket{00}\right) = \frac{1}{\sqrt{2}} \left(\ket{E_1} + \ket{E_2}\right)\]
at time \(t=0\). Note that this is \emph{not} a separable state of the form (\ref{eq.rho0fullinitial}), though it is a pure state.

At time \(t\), the state is
\[
\ket{\Psi(t)} = \frac{1}{\sqrt{2}}\left(e^{-itE_1/\hbar}\ket{E_1} + e^{-itE_2/\hbar} \ket{E_2}\right)
\]
and the density operator is
\[
\op{\rho} = \frac{1}{2}\ket{E_1}\bra{E_1} + \frac{1}{2}\ket{E_2}\bra{E_2} + \frac{1}{2} e^{it\left(E_2-E_1\right)/\hbar} \ket{E_1}\bra{E_2} + \frac{1}{2} e^{-it\left(E_2 - E_1\right)/\hbar} \ket{E_2}\bra{E_1}.
\]
We can compute that reduced density operator for the first qubit is
\begin{align*}
\overline{\op{\rho}} &= \cdot\otimes\bra{0} \op{\rho} \cdot\otimes\ket{0} + \cdot\otimes\bra{1} \op{\rho} \cdot\otimes\ket{1} \\
&= \frac{1}{2} \ket{1}\bra{1} + \frac{1}{2}\ket{0}\bra{0} + \frac{1}{4} e^{it\left(E_2-E_1\right)/\hbar}\left(-\ket{1}\bra{0} + \ket{0}\bra{1}\right) + \frac{1}{4} e^{-it\left(E_2-E_1\right)/\hbar}\left(\ket{1}\bra{0} - \ket{0}\bra{1}\right) \\
&= \frac{1}{2}\ket{1}\bra{1} + \frac{1}{2}\ket{0}\bra{0} + \frac{i}{2} \left(\sin \phi\right) \ket{0}\bra{1} - \frac{i}{2} \left(\sin \phi\right) \ket{1}\bra{0}
\end{align*}
where \(\phi = \frac{E_3-E_2}{\hbar}\,t = (1-\lambda)\,t\).
This \(\overline{\op{\rho}}\) has eigenvalues and eigenvectors
\begin{align*}
\sigma_{\pm} &= \frac{1 \pm \sin \phi}{2}, &
\ket{\psi_\pm} &= \frac{\pm i \ket{0} + \ket{1}}{\sqrt{2}}.
\end{align*}
Hence the Von Neumann subsystem entropy is
\[
S = - \frac{1}{2}\left(1-\sin\phi\right) \log \left(\frac{1}{2}\left(1-\sin\phi\right)\right) - \frac{1}{2}\left(1+\sin\phi\right) \log \left(\frac{1}{2}\left(1+\sin\phi\right)\right).
\]
Using \(\sin x = x + \Od{x^3}\) and \(\log\left(1+x\right) = x + \Od{x^2}\) we get
\begin{align*}
S &= \log 2 - \phi^2 + \Od{\phi^3} \\
&= \log 2 - t^2 + 2 \lambda t^2 - \lambda^2 t^2 + \Od{(1-\lambda)^3\,t^3}.
\end{align*}

Let us now consider the perturbative approach. Since the initial state is not separable, we cannot use (\ref{eq.Sredresult.separablepure}). Instead we will use (\ref{eq.foprobcor}) and (\ref{eq.deltaS.noassumptions}). We compute
\[
\left[\op{H}_I, \;\op{\rho}\right] = \frac{\hbar}{2} \left(\ket{00}\bra{10}- \ket{10}\bra{00} + \ket{00}\bra{01} - \ket{01}\bra{00} - \ket{11}\bra{10} + \ket{10}\bra{11} - \ket{11}\bra{01} +\ket{01}\bra{11}\right).
\]
Upon reduction this becomes
\begin{align*}
\overline{\left[\op{H}_I, \;\op{\rho}\right]} &= \cdot\otimes\bra{0} \left[\op{H}_I, \;\op{\rho}\right] \cdot\otimes\ket{0} + \cdot\otimes\bra{1} \left[\op{H}_I, \;\op{\rho}\right] \cdot\otimes\ket{1} \\
&= \hbar\left(\ket{0}\bra{1} - \ket{1}\bra{0}\right).
\end{align*}
By (\ref{eq.foprobcor}) the first-order correction to \(\sigma_-\) is
\begin{align*}
\sigma_-^{(1)} &= -\frac{i}{\hbar} \int_{0}^t \bra{\psi_-} \overline{\left[\op{H}_I, \;\op{\rho}\right]} \ket{\psi_-} dt^\prime \\
&= -\frac{i}{2} \int_{0}^t \left(i\bra{0} + \bra{1}\right) \left(\ket{0}\bra{1} - \ket{1}\bra{0}\right) \left(-i\ket{0} + \ket{1}\right) dt^\prime \\
&= t
\end{align*}
and similarly the first-order correction to \(\sigma_+\) is
\begin{align*}
\sigma_+^{(1)} &= -\frac{i}{\hbar} \int_{0}^t \bra{\psi_+} \overline{\left[\op{H}_I, \;\op{\rho}\right]} \ket{\psi_+} dt^\prime \\
&= -\frac{i}{2} \int_{0}^t \left(-i\bra{0} + \bra{1}\right) \left(\ket{0}\bra{1} - \ket{1}\bra{0}\right) \left(i\ket{0} + \ket{1}\right) dt^\prime \\
&= -t.
\end{align*}
Also, by taking \(\lambda=0\) in the definition of \(\sigma_\pm\) we get the zero-order corrections
\[
\sigma_\pm^{(0)} = \frac{1\pm \sin t}{2}.
\]
Then by (\ref{eq.deltaS.noassumptions}) the change in entropy due to the perturbation is
\begin{align*}
\Delta S &= \lambda \left(t \log (1 - \sin t) - t\log(1+\sin t)\right) \\
&= -2\lambda t^2 + \Od{t^4}
\end{align*}
which agrees with our above result up to sign. \emph{[I probably made a sign error somewhere. This should be fixed.]}

This example emphasizes that our perturbative results from Chapter \ref{ch.derivation} work even when \(\op{H}_0\) causes mixing between subsystems. We must keep in mind, however, that if \(\op{H}_0\) causes mixing between subsystems, there will be some time dependence in entropy at order \(\lambda^0\)---but this time dependence is not due to the perturbation.

