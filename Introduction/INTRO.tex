
In quantum mechanics, physical laws are written down as expressions describing total energy called ``Hamiltonians.'' I will use a set of techniques called perturbation theory to study Hamiltonians of the form \(\op{H} = \op{H}_0 + \lambda \op{V}\) where \(\op{H}_0\) is a Hamiltonian that is well understood and \(\lambda\) is very small. I will ask: how much does the entropy change due to \(\lambda \op{V}\)?

I start in this chapter by explaining what entropy is. In Chapter \ref{ch.theory}, I develop the perturbation theory tools that I will need later on. In Chapter \ref{ch.derivation}, I apply these tools to derive general formulae for the change in entropy due to the perturbation \(\lambda V\). In Chapter \ref{ch.ex}, I discuss a few examples of entropy evolution. Finally, in Chapter \ref{ch.conclusion}, I summarize the results and speculate about future research directions.

The physicist reading this thesis may want to jump straight to Chapter \ref{ch.derivation} because that is where the important results are to be found. The philosopher or casual reader may be more interested in Chapters \ref{ch.intro} and \ref{ch.conclusion}.
